% Options for packages loaded elsewhere
\PassOptionsToPackage{unicode}{hyperref}
\PassOptionsToPackage{hyphens}{url}
%
\documentclass[
]{article}
\usepackage{amsmath,amssymb}
\usepackage{iftex}
\ifPDFTeX
  \usepackage[T1]{fontenc}
  \usepackage[utf8]{inputenc}
  \usepackage{textcomp} % provide euro and other symbols
\else % if luatex or xetex
  \usepackage{unicode-math} % this also loads fontspec
  \defaultfontfeatures{Scale=MatchLowercase}
  \defaultfontfeatures[\rmfamily]{Ligatures=TeX,Scale=1}
\fi
\usepackage{lmodern}
\ifPDFTeX\else
  % xetex/luatex font selection
\fi
% Use upquote if available, for straight quotes in verbatim environments
\IfFileExists{upquote.sty}{\usepackage{upquote}}{}
\IfFileExists{microtype.sty}{% use microtype if available
  \usepackage[]{microtype}
  \UseMicrotypeSet[protrusion]{basicmath} % disable protrusion for tt fonts
}{}
\makeatletter
\@ifundefined{KOMAClassName}{% if non-KOMA class
  \IfFileExists{parskip.sty}{%
    \usepackage{parskip}
  }{% else
    \setlength{\parindent}{0pt}
    \setlength{\parskip}{6pt plus 2pt minus 1pt}}
}{% if KOMA class
  \KOMAoptions{parskip=half}}
\makeatother
\usepackage{xcolor}
\usepackage[margin=1in]{geometry}
\usepackage{color}
\usepackage{fancyvrb}
\newcommand{\VerbBar}{|}
\newcommand{\VERB}{\Verb[commandchars=\\\{\}]}
\DefineVerbatimEnvironment{Highlighting}{Verbatim}{commandchars=\\\{\}}
% Add ',fontsize=\small' for more characters per line
\usepackage{framed}
\definecolor{shadecolor}{RGB}{248,248,248}
\newenvironment{Shaded}{\begin{snugshade}}{\end{snugshade}}
\newcommand{\AlertTok}[1]{\textcolor[rgb]{0.94,0.16,0.16}{#1}}
\newcommand{\AnnotationTok}[1]{\textcolor[rgb]{0.56,0.35,0.01}{\textbf{\textit{#1}}}}
\newcommand{\AttributeTok}[1]{\textcolor[rgb]{0.13,0.29,0.53}{#1}}
\newcommand{\BaseNTok}[1]{\textcolor[rgb]{0.00,0.00,0.81}{#1}}
\newcommand{\BuiltInTok}[1]{#1}
\newcommand{\CharTok}[1]{\textcolor[rgb]{0.31,0.60,0.02}{#1}}
\newcommand{\CommentTok}[1]{\textcolor[rgb]{0.56,0.35,0.01}{\textit{#1}}}
\newcommand{\CommentVarTok}[1]{\textcolor[rgb]{0.56,0.35,0.01}{\textbf{\textit{#1}}}}
\newcommand{\ConstantTok}[1]{\textcolor[rgb]{0.56,0.35,0.01}{#1}}
\newcommand{\ControlFlowTok}[1]{\textcolor[rgb]{0.13,0.29,0.53}{\textbf{#1}}}
\newcommand{\DataTypeTok}[1]{\textcolor[rgb]{0.13,0.29,0.53}{#1}}
\newcommand{\DecValTok}[1]{\textcolor[rgb]{0.00,0.00,0.81}{#1}}
\newcommand{\DocumentationTok}[1]{\textcolor[rgb]{0.56,0.35,0.01}{\textbf{\textit{#1}}}}
\newcommand{\ErrorTok}[1]{\textcolor[rgb]{0.64,0.00,0.00}{\textbf{#1}}}
\newcommand{\ExtensionTok}[1]{#1}
\newcommand{\FloatTok}[1]{\textcolor[rgb]{0.00,0.00,0.81}{#1}}
\newcommand{\FunctionTok}[1]{\textcolor[rgb]{0.13,0.29,0.53}{\textbf{#1}}}
\newcommand{\ImportTok}[1]{#1}
\newcommand{\InformationTok}[1]{\textcolor[rgb]{0.56,0.35,0.01}{\textbf{\textit{#1}}}}
\newcommand{\KeywordTok}[1]{\textcolor[rgb]{0.13,0.29,0.53}{\textbf{#1}}}
\newcommand{\NormalTok}[1]{#1}
\newcommand{\OperatorTok}[1]{\textcolor[rgb]{0.81,0.36,0.00}{\textbf{#1}}}
\newcommand{\OtherTok}[1]{\textcolor[rgb]{0.56,0.35,0.01}{#1}}
\newcommand{\PreprocessorTok}[1]{\textcolor[rgb]{0.56,0.35,0.01}{\textit{#1}}}
\newcommand{\RegionMarkerTok}[1]{#1}
\newcommand{\SpecialCharTok}[1]{\textcolor[rgb]{0.81,0.36,0.00}{\textbf{#1}}}
\newcommand{\SpecialStringTok}[1]{\textcolor[rgb]{0.31,0.60,0.02}{#1}}
\newcommand{\StringTok}[1]{\textcolor[rgb]{0.31,0.60,0.02}{#1}}
\newcommand{\VariableTok}[1]{\textcolor[rgb]{0.00,0.00,0.00}{#1}}
\newcommand{\VerbatimStringTok}[1]{\textcolor[rgb]{0.31,0.60,0.02}{#1}}
\newcommand{\WarningTok}[1]{\textcolor[rgb]{0.56,0.35,0.01}{\textbf{\textit{#1}}}}
\usepackage{graphicx}
\makeatletter
\def\maxwidth{\ifdim\Gin@nat@width>\linewidth\linewidth\else\Gin@nat@width\fi}
\def\maxheight{\ifdim\Gin@nat@height>\textheight\textheight\else\Gin@nat@height\fi}
\makeatother
% Scale images if necessary, so that they will not overflow the page
% margins by default, and it is still possible to overwrite the defaults
% using explicit options in \includegraphics[width, height, ...]{}
\setkeys{Gin}{width=\maxwidth,height=\maxheight,keepaspectratio}
% Set default figure placement to htbp
\makeatletter
\def\fps@figure{htbp}
\makeatother
\setlength{\emergencystretch}{3em} % prevent overfull lines
\providecommand{\tightlist}{%
  \setlength{\itemsep}{0pt}\setlength{\parskip}{0pt}}
\setcounter{secnumdepth}{-\maxdimen} % remove section numbering
\ifLuaTeX
  \usepackage{selnolig}  % disable illegal ligatures
\fi
\IfFileExists{bookmark.sty}{\usepackage{bookmark}}{\usepackage{hyperref}}
\IfFileExists{xurl.sty}{\usepackage{xurl}}{} % add URL line breaks if available
\urlstyle{same}
\hypersetup{
  pdftitle={Cheese Succession (Brooke paper)},
  pdfauthor={Collin Edwards},
  hidelinks,
  pdfcreator={LaTeX via pandoc}}

\title{Cheese Succession (Brooke paper)}
\author{Collin Edwards}
\date{2023-04-04}

\begin{document}
\maketitle

\hypertarget{setup}{%
\section{Setup}\label{setup}}

\hypertarget{libraries}{%
\subsection{libraries}\label{libraries}}

\begin{Shaded}
\begin{Highlighting}[]
\FunctionTok{library}\NormalTok{(here)}
\end{Highlighting}
\end{Shaded}

\begin{verbatim}
## here() starts at G:/repos/Anderson_et_al_Succession
\end{verbatim}

\begin{Shaded}
\begin{Highlighting}[]
\FunctionTok{library}\NormalTok{(tidyverse)}
\end{Highlighting}
\end{Shaded}

\begin{verbatim}
## -- Attaching packages --------------------------------------- tidyverse 1.3.2
## --
\end{verbatim}

\begin{verbatim}
## v ggplot2 3.4.1      v purrr   0.3.5 
## v tibble  3.1.8      v dplyr   1.0.10
## v tidyr   1.2.1      v stringr 1.4.1 
## v readr   2.1.3      v forcats 0.5.2 
## -- Conflicts ------------------------------------------ tidyverse_conflicts() --
## x dplyr::filter() masks stats::filter()
## x dplyr::lag()    masks stats::lag()
\end{verbatim}

\begin{Shaded}
\begin{Highlighting}[]
\FunctionTok{library}\NormalTok{(deSolve)}
\FunctionTok{library}\NormalTok{(ggplot2)}
\FunctionTok{library}\NormalTok{(gridExtra)}
\end{Highlighting}
\end{Shaded}

\begin{verbatim}
## 
## Attaching package: 'gridExtra'
## 
## The following object is masked from 'package:dplyr':
## 
##     combine
\end{verbatim}

\begin{Shaded}
\begin{Highlighting}[]
\FunctionTok{library}\NormalTok{(cowplot)}
\FunctionTok{library}\NormalTok{(viridis)}
\end{Highlighting}
\end{Shaded}

\begin{verbatim}
## Loading required package: viridisLite
\end{verbatim}

\begin{Shaded}
\begin{Highlighting}[]
\FunctionTok{library}\NormalTok{(pander)}
\FunctionTok{library}\NormalTok{(car)}
\end{Highlighting}
\end{Shaded}

\begin{verbatim}
## Loading required package: carData
## 
## Attaching package: 'car'
## 
## The following object is masked from 'package:dplyr':
## 
##     recode
## 
## The following object is masked from 'package:purrr':
## 
##     some
\end{verbatim}

\hypertarget{functions}{%
\subsection{Functions}\label{functions}}

Sometimes it's helpful to define a function for code we want to re-use a
lot. The key function like that for me is something I call
\texttt{gfig\_saver}, which makes saving ggplot figures systematic.

\begin{Shaded}
\begin{Highlighting}[]
\CommentTok{\# function for saving ggplot figures and metadata. As fig\_starter (which is for base graphics), except that saving ggfigures is inherently cleaner, as you are not feeding commands to an open graphics device}

\NormalTok{gfig\_saver}\OtherTok{=}\ControlFlowTok{function}\NormalTok{(gfig, }\CommentTok{\#object to be saved}
\NormalTok{                    filename, }\CommentTok{\#name of figure file to save as WITHOUT SUFFIX}
\NormalTok{                    description, }\CommentTok{\#vector of strings, each will be put in its own line of meta file}
                    \DocumentationTok{\#\#  Note: generating file is defined in the function, date and time is automatically added.}
                    \DocumentationTok{\#\#default figure info:}
                    \AttributeTok{width=}\DecValTok{12}\NormalTok{,}
                    \AttributeTok{height=}\DecValTok{8}\NormalTok{,}
                    \AttributeTok{res=}\DecValTok{300}\NormalTok{,}
                    \AttributeTok{units=}\StringTok{"in"}\NormalTok{,}
                    \AttributeTok{figfold=}\StringTok{"5\_figs"} \CommentTok{\#folder to save figures in}
\NormalTok{)\{}
  \DocumentationTok{\#\# save meta file}
  \FunctionTok{cat}\NormalTok{(}\FunctionTok{c}\NormalTok{(description,}
        \StringTok{""}\NormalTok{,}\DocumentationTok{\#\#easy way to add an extra line to separate description for basic data.}
        
        \FunctionTok{paste}\NormalTok{(}\StringTok{"from"}\NormalTok{, knitr}\SpecialCharTok{::}\FunctionTok{current\_input}\NormalTok{()),}
        \StringTok{"RDS file with same name contains ggplot object used to generate this figure."}\NormalTok{,}
        \FunctionTok{as.character}\NormalTok{(}\FunctionTok{Sys.time}\NormalTok{())),}
      \AttributeTok{sep=}\StringTok{"}\SpecialCharTok{\textbackslash{}n}\StringTok{"}\NormalTok{,}
      \AttributeTok{file=}\FunctionTok{here}\NormalTok{(figfold, }\FunctionTok{paste0}\NormalTok{(filename,}\StringTok{"\_meta.txt"}\NormalTok{, }\AttributeTok{sep=}\StringTok{""}\NormalTok{))}
\NormalTok{  )}
  \CommentTok{\#save figure as jpg (change code here for other figure types)}
  \FunctionTok{ggsave}\NormalTok{(}\AttributeTok{filename=}\FunctionTok{here}\NormalTok{(figfold, }\FunctionTok{paste0}\NormalTok{(filename,}\StringTok{".jpg"}\NormalTok{)),}
         \AttributeTok{plot=}\NormalTok{gfig,}
         \AttributeTok{device=}\StringTok{"jpeg"}\NormalTok{,}
         \AttributeTok{dpi=}\NormalTok{res,}
         \AttributeTok{width=}\NormalTok{width, }\AttributeTok{height=}\NormalTok{height, }\AttributeTok{units=}\NormalTok{units}
\NormalTok{  )}
  \FunctionTok{ggsave}\NormalTok{(}\AttributeTok{filename=}\FunctionTok{here}\NormalTok{(figfold, }\FunctionTok{paste0}\NormalTok{(filename,}\StringTok{".pdf"}\NormalTok{)),}
         \AttributeTok{plot=}\NormalTok{gfig,}
         \AttributeTok{device=}\StringTok{"pdf"}\NormalTok{,}
         \AttributeTok{dpi=}\NormalTok{res,}
         \AttributeTok{width=}\NormalTok{width, }\AttributeTok{height=}\NormalTok{height, }\AttributeTok{units=}\NormalTok{units}
\NormalTok{  )}
  \CommentTok{\#save ggplot object as RDS file, for easy manipulation later}
  \FunctionTok{saveRDS}\NormalTok{(}\AttributeTok{object =}\NormalTok{ gfig, }\AttributeTok{file=}\FunctionTok{here}\NormalTok{(figfold, }\FunctionTok{paste0}\NormalTok{(filename,}\StringTok{".RDS"}\NormalTok{)))}
\NormalTok{\}}
\end{Highlighting}
\end{Shaded}

\begin{Shaded}
\begin{Highlighting}[]
\NormalTok{theme.mine}\OtherTok{=}\FunctionTok{theme}\NormalTok{(}\AttributeTok{plot.title =} \FunctionTok{element\_text}\NormalTok{(}\AttributeTok{face=}\StringTok{"bold"}\NormalTok{, }\AttributeTok{size=}\DecValTok{24}\NormalTok{),}
                 \AttributeTok{text=}\FunctionTok{element\_text}\NormalTok{(}\AttributeTok{size=}\FunctionTok{rel}\NormalTok{(}\FloatTok{4.5}\NormalTok{)),}
                 \AttributeTok{legend.text=}\FunctionTok{element\_text}\NormalTok{(}\AttributeTok{size=}\FunctionTok{rel}\NormalTok{(}\DecValTok{2}\NormalTok{))}
\NormalTok{)}
\end{Highlighting}
\end{Shaded}

\hypertarget{purpose}{%
\subsection{Purpose}\label{purpose}}

The goal is to determine whether data from single-species and pairwise
experiments (e.g.~at most pairwise interactions) in Brooke's cheese data
can explain the community dynamics from treatments inoculated with a
full 7-species community. The manuscript for this project is
\href{https://docs.google.com/document/u/1/d/1KIhnJMmHf3wOXv_0DQ9UgcjIVkz1v0FGFiyUVdktreE/edit?ts=608b2425}{here}.

\hypertarget{plan}{%
\subsection{Plan}\label{plan}}

\textbf{The goal:}

Determine if the 7-species community can be predicted/described by
pairwise interactions.

\textbf{The problem:}

``Community'' is hard to define. With only 4 time points, classical
population dynamics approaches are liable to run into issues. Estimating
confidence intervals from the standard trajectory-matching approach for
fitting population dynamics models is unreasonable, as bootstrapping
seems problematic with such little data.

\textbf{The solution:}

If we imagine that the system is actually following Lotka-Volterra
competition (the simplest competition model), then at least for the
2-species experiments, the populations will eventually reach an
equilibrium (which can have one of the populations at size zero). The
equilibrium for Lotka Volterra is clearly defined in terms of just
carrying capacity and interaction terms (generally denoted with
\(\alpha\)s) that capture either competition or stimulation. Using that
knowledge, and assuming that the populations are at equilibrium on day
21, we can estimate the alphas and the carrying capacity from the paired
experiments, and see if those predict the final population sizes of the
7-species community.

\textbf{A little math background:}

The Lotka-Volterra competition model is defined by a system of Ordinary
Differential Equations (ODEs). This is fancy math-speak for some
equations that define how the size of the populations changes at any
given point in time, based on the current size of the populations. For
LV, it's this set of equations:

\[\frac{dx_1}{dt} = r_1x_1\big(1-\big(\frac{x_1 \alpha_1_2x_2}{K_2}\big)\big)\]

Where x\_i is the population size of species i, the r parameters
correspond to intrinsic growth rates (growth when not competing) the K
parameters correspond to the carrying capacity in the absence of other
species alpha\_ij is the competition coefficient for species j on
species i. I like to think of this as a conversion rate - an alpha\_ij
of 2 means that each individual of species j has the same competition
effect on species i as would 2 individuals of species i. So big values
mean the other species has a strong suppression effect, small values
mean a minimal suppression effect. I'm not used to LV with positive
effects, but negative alphas mean that one species is helping the other.
There is a lot of really cool math that has gone into things we can
discover/prove with ODEs, but we actually don't need most of it right
now. If our populations are at equilibrium in our day 21 data, that
means the population sizes aren't changing. So there's no rate of
change, so dx/dt is 0. If we set the left hand side of each equation to
zero, and do a little re-arranging of terms with algebra, we get:

\[x_1 = K_1 - \alpha_2_1 * x_2\]

(and the other version, where each number is swapped)

And that's kind of dry looking, but it's just a different version of

y = mx + b.

where y and x are the sizes of our two populations (e.g.~the things we
have data for)

Taking this approach, we can find carrying capacity and interaction
terms with linear regression using the single-species and pairwise
interaction data (!!). We still face the caveats of our various
assumptions (below), but this is suddenly incredibly tractable. And the
question of ``What would the 7-species community look like if it were
mostly the result of pairwise interactions'' could, I think, be
addressed pretty easily at that point. The 7-species version of lotka
volterra has more equations, but the equilibrium has the same structure,
I believe, except it's

K\_1 - (sum of all six appropriate alpha * x terms).

(I'll double check that at some point here). So we can use our fitted
versions of all our K and alpha values from the pairwise and
single-species experiments, plug them into the equation for equilibrium
of the 7-species model, and predict what we would expect the 7-species
equilibrium would be (assuming only pairwise interactions mattered). To
look at if reality is significantly different from what we predict, we
can use parametric bootstrapping: basically we generate ``random'' K and
alpha values based on our estimated distributions from fitting those
parameters from the single-species and pairwise experiments, and for
each set of random K and alpha values, we make a prediction for the
final community. We can do that a bunch of times, and that creates a
distribution of expectations assuming pairwise interactions - if the
real data is outside of 95\% of that, then it's significantly different.
Since in this case we have 7 abundances that are not independent, we may
need to use something like PERMANOVA to compare the bootstrap-simulated
final communities with the actual 7-species community data, but that
shouldn't actually be very hard.

Some caveats: First, we're assuming that lotka volterra dynamics do a
good job of capturing the key dynamics in the system. Second, we're
assuming that populations are at equilibrium on day 21 -- from the data,
it looks like they are in some cases, and not in others. Additionally,
communities of more than 2 species with lotka volterra competition
aren't guaranteed to have an equilibrium at all - they can exhibit
non-equilibrium dynamics like periodic cycles through time, or
mathematical chaos.

Looking at the results (e.g.~Fig 1A), it certainly looks like we're
wrong in making these assumptions, but that we're probably close enough.
(Any modeling work will be making some incorrect assumptions - the goal
is to make sure they don't matter). My take on all of this is that (a)
it doesn't look like there's anything too crazy going on here (no
sine-wave type behavior we can see, for example), (b) there's a lot
that's unknown in modeling microbial communities, and it seems like the
most elaborate things being used right now are Lotka Volterra, so I
think we're in fine company, and (c) we have to work with the data we
have, and while it's astounding in a lot of ways, we don't have enough
time points to effectively do more elaborate modeling of population
dynamics.

So I would go forward with this as described above, so long as we're
clear about what we're doing in the text (happy to help make the writing
clear on that). I think once I've at least prototyped this, I might also
run it by a few more senior theoretical ecologists I work with, in case
I'm missing some subtle issue.

One thing to bear in mind is that LV is also often unsuited to capturing
mutualisms -- there are some parameter values where both species explode
to infinite size even though there's a finite carrying capacity. I don't
think that's an issue with our data, but it's something to be aware of.
I haven't dug into it a ton, but there has been recent work on how to
model community dynamics and coexistence when working with mutualisms. I
might need to take a look at that at some point (although again, I think
we have a good plan outlined above).

Some anti-caveats

One of the nice things about only working with the stable equilibrium
(aside from the math being tractable) is that we're not actually making
any assumptions about growth rate or dynamics up to that point. We're
just assuming that each species has an equilibrium size that is a linear
function of the other species at equilibrium. So if we wanted to, we
could actually NOT call this lotka volterra, since our approach
encompasses ANY model with that equilibrium conditions, not just lotka
volterra. (But I think we still couch it in terms of LV, because folks
are familiar with that).

A few pieces I may need to explore:

We're using some of the data to estimate different parameters in
independent analyses, and it's not quite clear to me how we account for
that when representing error (e.g.~our estimate of alpha\_21 and
alpha\_12 come from fitting the same data of pairwise species
competition, but we're fitting that data in two different equations). I
may need to think some more and/or consult Elizabeth. Additionally,
there's a generalized Lotka Volterra model that I'm not super familiar
with, but appears to be more common for working with many-species
communities. I believe that it is fundamentally the same thing as I've
outlined above, but the formulation is different (e.g.~different
parameter names, somewhat different version of the equation), and it
might be good to rework my math to match that nomenclature. I might dig
into it.

The plan: I try this out with the data - fit the carrying capacities and
the alphas with linear regression, see how well it predicts the data
that I'm using to fit it, see how it predicts the 7-species community. I
expect this will take a day or less of time, and I should be able to fit
that in in the next two weeks. This won't focus on significance testing,
but just ``can we estimate parameters'' and ``what do our estimated
parameters suggest''. I write up a brief description of the methodology
with code and example figures, and run it by Elizabeth Crone
(statistical ecologist who can make suggestions on the non-independence
of the different regressions), and then Steve Ellner, Sebastian
Schrieber, and/or Giles Hooker, who have expertise in working with
dynamical models (and in Giles' case, literally wrote the book for
fitting dynamical models to data) and can act as an additional sounding
board for the ``let's assume we're at equilibrium'' bit. ??? Profit.

\hypertarget{read-in-data}{%
\subsection{Read in data}\label{read-in-data}}

\begin{Shaded}
\begin{Highlighting}[]
\NormalTok{dat.orig }\OtherTok{=}\NormalTok{ raw }\OtherTok{=} \FunctionTok{read.csv}\NormalTok{(}\FunctionTok{here}\NormalTok{(}\StringTok{"1\_raw\_data"}\NormalTok{,}
                               \StringTok{"2018\_08\_21\_Bayley\_allPW.csv"}\NormalTok{))}
\end{Highlighting}
\end{Shaded}

\hypertarget{approach-use-linear-models}{%
\section{Approach: Use linear models}\label{approach-use-linear-models}}

If we assume the communities are at equilibrium on day 21, and if their
dynamics are defined by lotka volterra dynamics, their abundance is a
linear function of the densities of competitors. For the two-species LV
model, the equilibrium densities of species 1 and species 2 (denoted
\(N_1^*\) and \(N_2^*\)) are defined by

\[N_1^* = K_1 - \alpha_{12}N_2^*\]

and

\[N_2^* = K_2 - \alpha_{21}N_1^*\]

Since we're assuming the populations are at equilibrium, our
observations give us \(N_1^*\) and \(N_2^*\), and we can fit a
regression model to estimate \(K\) and the \(\alpha\) terms. (and we can
use the regression framework to estimate our error)

Using our pairwise experiments, we can estimate all parameters for
species 1 simultaneously, with

\[N_1^* = K_1 - \alpha_{12}N_2^* - \alpha_{13}N_3^*-\dots\]

Note that when fitting the above model, for any given entry, at most one
of the Ns on the right hand side will be non-zero (that is, either the
observation comes from a species alone, in which case all other Ns are
0, or it comes from a two-species experiment, in which case ONE other N
will be some postitive number, and the rest will be zeros).

Using this framework, we can fit the data and see how well it predicts
the observed 7-species community.

Note two caveats:

\begin{itemize}
\tightlist
\item
  if it doesn't fit the 7-species data well, that suggests either
  complex interactions OR that one or more of the data sets isn't at
  equilibrium.
\item
  The error structure is probably pretty complicated, since we're
  ``re-using'' the same data to estimate the parameters for multiple
  species. Elizabeth and I both think this isn't worth worrying about
\end{itemize}

Update (8/26/21): fitting the data in two step process: first estimate
k, then calculate everything else. So estimate k using original data,
then for interaction terms fit a no-intercept model after subtracting
the carrying capacity.

\hypertarget{step-1-restructure-data}{%
\subsection{Step 1: restructure data}\label{step-1-restructure-data}}

We need a data frame with a row for each experiment, and a column for
the density of each species.

\begin{Shaded}
\begin{Highlighting}[]
\CommentTok{\#make empty data frame}
\CommentTok{\#We want to store cfus for each species, }
\NormalTok{df.template }\OtherTok{=} 
\NormalTok{  dat.mat }\OtherTok{=} \FunctionTok{setNames}\NormalTok{(}\FunctionTok{data.frame}\NormalTok{(}\FunctionTok{matrix}\NormalTok{(}\AttributeTok{ncol =} \DecValTok{2}\SpecialCharTok{*}\FunctionTok{length}\NormalTok{(}\FunctionTok{unique}\NormalTok{(raw}\SpecialCharTok{$}\NormalTok{spec))}\SpecialCharTok{+}\DecValTok{2}\NormalTok{, }\AttributeTok{nrow =} \DecValTok{0}\NormalTok{)),}
                     \FunctionTok{c}\NormalTok{(}\FunctionTok{as.character}\NormalTok{(}\FunctionTok{unique}\NormalTok{(raw}\SpecialCharTok{$}\NormalTok{spec)), }\StringTok{"cond"}\NormalTok{, }\StringTok{"rep"}\NormalTok{,}\FunctionTok{paste0}\NormalTok{(}\StringTok{"pres."}\NormalTok{,}\FunctionTok{c}\NormalTok{(}\FunctionTok{as.character}\NormalTok{(}\FunctionTok{unique}\NormalTok{(raw}\SpecialCharTok{$}\NormalTok{spec)))))}
\NormalTok{  )}
\NormalTok{raw.use }\OtherTok{=}\NormalTok{ raw }\SpecialCharTok{\%\textgreater{}\%} 
  \FunctionTok{filter}\NormalTok{(day }\SpecialCharTok{==} \DecValTok{21}\NormalTok{) }\SpecialCharTok{\%\textgreater{}\%} 
  \FunctionTok{filter}\NormalTok{(pH }\SpecialCharTok{==} \DecValTok{5}\NormalTok{)}
\CommentTok{\#handle the alones}
\ControlFlowTok{for}\NormalTok{(cur.spec }\ControlFlowTok{in} \FunctionTok{unique}\NormalTok{(raw.use}\SpecialCharTok{$}\NormalTok{spec))\{}
  \FunctionTok{print}\NormalTok{(cur.spec)}
\NormalTok{  dat.cur }\OtherTok{=}\NormalTok{ raw.use }\SpecialCharTok{\%\textgreater{}\%} 
    \FunctionTok{filter}\NormalTok{(spec }\SpecialCharTok{==}\NormalTok{ cur.spec) }\SpecialCharTok{\%\textgreater{}\%} 
    \FunctionTok{filter}\NormalTok{(cond }\SpecialCharTok{==} \StringTok{"alone"}\NormalTok{)}
\NormalTok{  df.fill}\OtherTok{=}\FunctionTok{as.data.frame}\NormalTok{(}\FunctionTok{matrix}\NormalTok{(}\DecValTok{0}\NormalTok{, }
                               \AttributeTok{nrow=}\FunctionTok{nrow}\NormalTok{(dat.cur),}
                               \AttributeTok{ncol=}\FunctionTok{ncol}\NormalTok{(dat.mat)))}
  \FunctionTok{names}\NormalTok{(df.fill) }\OtherTok{=} \FunctionTok{names}\NormalTok{(dat.mat)}
\NormalTok{  df.fill[,cur.spec]}\OtherTok{=}\NormalTok{dat.cur}\SpecialCharTok{$}\NormalTok{cfus}
\NormalTok{  df.fill[,}\StringTok{"rep"}\NormalTok{]}\OtherTok{=}\NormalTok{dat.cur}\SpecialCharTok{$}\NormalTok{rep}
\NormalTok{  df.fill[,}\StringTok{"cond"}\NormalTok{]}\OtherTok{=}\StringTok{"alone"}
\NormalTok{  df.fill[,}\FunctionTok{paste0}\NormalTok{(}\StringTok{"pres."}\NormalTok{,cur.spec)] }\OtherTok{=}\NormalTok{ T}
\NormalTok{  dat.mat }\OtherTok{=} \FunctionTok{rbind}\NormalTok{(dat.mat, df.fill)}
\NormalTok{\}}
\end{Highlighting}
\end{Shaded}

\begin{verbatim}
## [1] "BC9"
## [1] "BC10"
## [1] "JB5"
## [1] "JB7"
## [1] "135E"
## [1] "JBC"
## [1] "JB370"
\end{verbatim}

\begin{Shaded}
\begin{Highlighting}[]
\DocumentationTok{\#\# pairs}
\NormalTok{spec.vec }\OtherTok{=} \FunctionTok{as.character}\NormalTok{(}\FunctionTok{unique}\NormalTok{(raw.use}\SpecialCharTok{$}\NormalTok{spec))}

\ControlFlowTok{for}\NormalTok{(i.spec }\ControlFlowTok{in} \DecValTok{1}\SpecialCharTok{:}\NormalTok{(}\FunctionTok{length}\NormalTok{(spec.vec)}\SpecialCharTok{{-}}\DecValTok{1}\NormalTok{))\{}
\NormalTok{  cur.spec }\OtherTok{=}\NormalTok{ spec.vec[i.spec]}
  \ControlFlowTok{for}\NormalTok{(j.spec }\ControlFlowTok{in}\NormalTok{ (i.spec}\SpecialCharTok{+}\DecValTok{1}\NormalTok{)}\SpecialCharTok{:}\FunctionTok{length}\NormalTok{(spec.vec))\{}
\NormalTok{    other.spec }\OtherTok{=}\NormalTok{ spec.vec[j.spec]}
    \CommentTok{\#get cfus of current species}
\NormalTok{    dat.cur }\OtherTok{=}\NormalTok{ raw.use }\SpecialCharTok{\%\textgreater{}\%} 
      \FunctionTok{filter}\NormalTok{(spec }\SpecialCharTok{==}\NormalTok{ cur.spec) }\SpecialCharTok{\%\textgreater{}\%} 
      \FunctionTok{filter}\NormalTok{(cond }\SpecialCharTok{==}\NormalTok{ other.spec)}
    \CommentTok{\#get cfus of other species}
\NormalTok{    dat.other }\OtherTok{=}\NormalTok{ raw.use }\SpecialCharTok{\%\textgreater{}\%} 
      \FunctionTok{filter}\NormalTok{(spec }\SpecialCharTok{==}\NormalTok{ other.spec) }\SpecialCharTok{\%\textgreater{}\%} 
      \FunctionTok{filter}\NormalTok{(cond }\SpecialCharTok{==}\NormalTok{ cur.spec)}
    \CommentTok{\#handle replicate mismatching}
\NormalTok{    dat.cur }\OtherTok{=}\NormalTok{ dat.cur[dat.cur}\SpecialCharTok{$}\NormalTok{rep }\SpecialCharTok{\%in\%}\NormalTok{ dat.other}\SpecialCharTok{$}\NormalTok{rep,]}
\NormalTok{    dat.other }\OtherTok{=}\NormalTok{ dat.other[dat.other}\SpecialCharTok{$}\NormalTok{rep }\SpecialCharTok{\%in\%}\NormalTok{ dat.cur}\SpecialCharTok{$}\NormalTok{rep,]}
\NormalTok{    dat.cur}\OtherTok{=}\NormalTok{dat.cur[}\FunctionTok{order}\NormalTok{(dat.cur}\SpecialCharTok{$}\NormalTok{rep),]}
\NormalTok{    dat.other}\OtherTok{=}\NormalTok{dat.other[}\FunctionTok{order}\NormalTok{(dat.other}\SpecialCharTok{$}\NormalTok{rep),]}
    
    \DocumentationTok{\#\# The following code is a bunch of sanity checking.}
    \ControlFlowTok{if}\NormalTok{(}\FunctionTok{nrow}\NormalTok{(dat.cur) }\SpecialCharTok{!=} \FunctionTok{nrow}\NormalTok{(dat.other))\{}
      \FunctionTok{stop}\NormalTok{(}\StringTok{"dimensions not matching up. investigate"}\NormalTok{)}
      \CommentTok{\# print(paste(cur.spec, other.spec))}
      \CommentTok{\# print("dimensions not matching up. investigate")}
\NormalTok{    \}}
    \ControlFlowTok{if}\NormalTok{(}\FunctionTok{any}\NormalTok{(dat.cur}\SpecialCharTok{$}\NormalTok{rep }\SpecialCharTok{!=}\NormalTok{ dat.other}\SpecialCharTok{$}\NormalTok{rep))\{}
      \FunctionTok{stop}\NormalTok{(}\StringTok{"replicates not matching up. investigate"}\NormalTok{)}
      \CommentTok{\# print(paste(cur.spec, other.spec))}
      \CommentTok{\# print("replicates not matching up. investigate")}
\NormalTok{    \}}
\NormalTok{    df.fill}\OtherTok{=}\FunctionTok{as.data.frame}\NormalTok{(}\FunctionTok{matrix}\NormalTok{(}\DecValTok{0}\NormalTok{,}
                                 \AttributeTok{nrow=}\FunctionTok{nrow}\NormalTok{(dat.cur),}
                                 \AttributeTok{ncol=}\FunctionTok{ncol}\NormalTok{(dat.mat)))}
    \FunctionTok{names}\NormalTok{(df.fill) }\OtherTok{=} \FunctionTok{names}\NormalTok{(dat.mat)}
\NormalTok{    df.fill[ , cur.spec] }\OtherTok{=}\NormalTok{ dat.cur}\SpecialCharTok{$}\NormalTok{cfus}
\NormalTok{    df.fill[ , other.spec] }\OtherTok{=}\NormalTok{ dat.other}\SpecialCharTok{$}\NormalTok{cfus}
\NormalTok{    df.fill[,}\StringTok{"rep"}\NormalTok{]}\OtherTok{=}\NormalTok{dat.cur}\SpecialCharTok{$}\NormalTok{rep}
\NormalTok{    df.fill[,}\StringTok{"cond"}\NormalTok{]}\OtherTok{=}\StringTok{"paired{-}comp"}
\NormalTok{    df.fill[,}\FunctionTok{paste0}\NormalTok{(}\StringTok{"pres."}\NormalTok{,cur.spec)] }\OtherTok{=}\NormalTok{ T}
\NormalTok{    df.fill[,}\FunctionTok{paste0}\NormalTok{(}\StringTok{"pres."}\NormalTok{,other.spec)] }\OtherTok{=}\NormalTok{ T}
\NormalTok{    dat.mat }\OtherTok{=} \FunctionTok{rbind}\NormalTok{(dat.mat, df.fill)}
\NormalTok{  \}}
\NormalTok{\}}
\NormalTok{dat.mat}\SpecialCharTok{$}\NormalTok{day }\OtherTok{=} \DecValTok{21}
\NormalTok{dat.mat}\SpecialCharTok{$}\NormalTok{pH }\OtherTok{=} \DecValTok{5}
\FunctionTok{names}\NormalTok{(dat.mat)[}\FunctionTok{names}\NormalTok{(dat.mat)}\SpecialCharTok{==}\StringTok{"pres.135E"}\NormalTok{]}\OtherTok{=}\StringTok{"pres.X135E"}

\FunctionTok{write.csv}\NormalTok{(dat.mat, }
          \AttributeTok{file =} \FunctionTok{here}\NormalTok{(}\StringTok{"2\_data\_wrangling"}\NormalTok{,}\StringTok{"matrix{-}form.csv"}\NormalTok{),}
          \AttributeTok{row.names =} \ConstantTok{FALSE}
\NormalTok{)}
\end{Highlighting}
\end{Shaded}

Make a data frame to store ALL community prediction data for MSE
calculations at the end

\begin{Shaded}
\begin{Highlighting}[]
\NormalTok{pred.all }\OtherTok{=} \ConstantTok{NULL}
\end{Highlighting}
\end{Shaded}

\hypertarget{fitting}{%
\subsection{Fitting}\label{fitting}}

Note that we use carrying capacities calculated from the monocultures.

\hypertarget{testing-fit-one-species}{%
\subsubsection{Testing: fit one species}\label{testing-fit-one-species}}

\begin{Shaded}
\begin{Highlighting}[]
\FunctionTok{library}\NormalTok{(lme4)}
\end{Highlighting}
\end{Shaded}

\begin{verbatim}
## Loading required package: Matrix
\end{verbatim}

\begin{verbatim}
## 
## Attaching package: 'Matrix'
\end{verbatim}

\begin{verbatim}
## The following objects are masked from 'package:tidyr':
## 
##     expand, pack, unpack
\end{verbatim}

\begin{Shaded}
\begin{Highlighting}[]
\NormalTok{dat.fit }\OtherTok{=} \FunctionTok{read.csv}\NormalTok{(}\AttributeTok{file =} \FunctionTok{here}\NormalTok{(}\StringTok{"2\_data\_wrangling"}\NormalTok{,}\StringTok{"matrix{-}form.csv"}\NormalTok{))}
\CommentTok{\#NOTE: R adds a leading X to column names that start with numbers}

\NormalTok{dat.k }\OtherTok{=}\NormalTok{ dat.orig }\SpecialCharTok{\%\textgreater{}\%} 
  \FunctionTok{filter}\NormalTok{(day}\SpecialCharTok{==}\DecValTok{21}\NormalTok{) }\SpecialCharTok{\%\textgreater{}\%} 
  \FunctionTok{filter}\NormalTok{(spec}\SpecialCharTok{==}\StringTok{"BC9"}\NormalTok{) }\SpecialCharTok{\%\textgreater{}\%} 
  \FunctionTok{filter}\NormalTok{(cond}\SpecialCharTok{==}\StringTok{"alone"}\NormalTok{) }\SpecialCharTok{\%\textgreater{}\%} 
  \FunctionTok{filter}\NormalTok{(pH }\SpecialCharTok{==} \DecValTok{5}\NormalTok{)}
\NormalTok{out }\OtherTok{=} \FunctionTok{lm}\NormalTok{(cfus }\SpecialCharTok{\textasciitilde{}} \DecValTok{1}\NormalTok{, }\AttributeTok{data =}\NormalTok{ dat.k )}
\NormalTok{k.cur }\OtherTok{=}\NormalTok{ out}\SpecialCharTok{$}\NormalTok{coefficients[[}\DecValTok{1}\NormalTok{]]}
\NormalTok{dat.mod }\OtherTok{=}\NormalTok{ dat.fit[dat.fit}\SpecialCharTok{$}\NormalTok{pres.BC9}\SpecialCharTok{==}\DecValTok{1}\NormalTok{,]}
\NormalTok{dat.mod}\SpecialCharTok{$}\NormalTok{BC9}\OtherTok{=}\NormalTok{dat.mod}\SpecialCharTok{$}\NormalTok{BC9}\SpecialCharTok{{-}}\NormalTok{k.cur}
\NormalTok{out }\OtherTok{=} \FunctionTok{lm}\NormalTok{(BC9 }\SpecialCharTok{\textasciitilde{}} \SpecialCharTok{{-}}\DecValTok{1} \SpecialCharTok{+}\NormalTok{ BC10 }\SpecialCharTok{+}\NormalTok{ JB5 }\SpecialCharTok{+}\NormalTok{ JB7 }\SpecialCharTok{+}\NormalTok{ X135E }\SpecialCharTok{+}\NormalTok{ JBC }\SpecialCharTok{+}\NormalTok{ JB370,}
         \AttributeTok{data =}\NormalTok{ dat.mod)}
\FunctionTok{summary}\NormalTok{(out)}
\end{Highlighting}
\end{Shaded}

\begin{verbatim}
## 
## Call:
## lm(formula = BC9 ~ -1 + BC10 + JB5 + JB7 + X135E + JBC + JB370, 
##     data = dat.mod)
## 
## Residuals:
##        Min         1Q     Median         3Q        Max 
## -237586019  -14199353   -1060000   14414118  330351390 
## 
## Coefficients:
##         Estimate Std. Error t value Pr(>|t|)    
## BC10   2.209e-02  3.607e-01   0.061 0.951658    
## JB5    5.778e+03  4.211e+04   0.137 0.891965    
## JB7    4.759e+04  6.769e+05   0.070 0.944510    
## X135E  4.997e+00  1.107e+00   4.516 0.000131 ***
## JBC   -2.143e-02  6.839e-01  -0.031 0.975256    
## JB370  2.661e+01  3.578e+00   7.438 8.64e-08 ***
## ---
## Signif. codes:  0 '***' 0.001 '**' 0.01 '*' 0.05 '.' 0.1 ' ' 1
## 
## Residual standard error: 96680000 on 25 degrees of freedom
## Multiple R-squared:  0.7519, Adjusted R-squared:  0.6923 
## F-statistic: 12.62 on 6 and 25 DF,  p-value: 1.578e-06
\end{verbatim}

\begin{Shaded}
\begin{Highlighting}[]
\FunctionTok{hist}\NormalTok{(}\FunctionTok{resid}\NormalTok{(out), }\AttributeTok{breaks=}\DecValTok{20}\NormalTok{)}
\end{Highlighting}
\end{Shaded}

\includegraphics{succession-analysis-v1.2_files/figure-latex/unnamed-chunk-6-1.pdf}

\begin{Shaded}
\begin{Highlighting}[]
\CommentTok{\# coefficients(out)}

\CommentTok{\#UPDATED }\AlertTok{NOTE}\CommentTok{: upon reflection, the log scale enforces a different relationship between}
\CommentTok{\# individuals, and would NOT capture the L{-}V model. So we should not do this even from}
\CommentTok{\# a theoretical basis, even ignoring that practically the linear model fit better.}
\CommentTok{\# }
\CommentTok{\# dat.log=dat.fit}
\CommentTok{\# dat.log[,1:7] = log(dat.log[,1:7]+1)}
\CommentTok{\# \#let\textquotesingle{}s try a log transform, because that seems relevant}
\CommentTok{\# out.log = lm(BC9 \textasciitilde{} BC10 + JB5 + JB7 + X135E + JBC + JB370,}
\CommentTok{\#              data = dat.log[dat.log$pres.BC9==1,])}
\CommentTok{\# hist(resid(out.log), breaks=20)}
\CommentTok{\# \#The linear scale appears better, and is simpler to handle.}
\end{Highlighting}
\end{Shaded}

\hypertarget{loop-over-all-species}{%
\subsubsection{Loop over all species}\label{loop-over-all-species}}

\begin{Shaded}
\begin{Highlighting}[]
\NormalTok{res.est }\OtherTok{=} \ConstantTok{NULL}
\NormalTok{spec.vecfit }\OtherTok{=} \FunctionTok{unique}\NormalTok{(raw}\SpecialCharTok{$}\NormalTok{spec)}
\NormalTok{spec.vecfit[spec.vecfit}\SpecialCharTok{==}\StringTok{"135E"}\NormalTok{] }\OtherTok{=} \StringTok{"X135E"}
\ControlFlowTok{for}\NormalTok{(i.spec }\ControlFlowTok{in} \DecValTok{1}\SpecialCharTok{:}\FunctionTok{length}\NormalTok{(spec.vec))\{}
\NormalTok{  cur.spec }\OtherTok{=}\NormalTok{ spec.vecfit[i.spec]}
  \FunctionTok{print}\NormalTok{(cur.spec)}
  
  \DocumentationTok{\#\# calculate K}
\NormalTok{  dat.k }\OtherTok{=}\NormalTok{ dat.orig }
\NormalTok{  dat.k}\SpecialCharTok{$}\NormalTok{spec[dat.k}\SpecialCharTok{$}\NormalTok{spec}\SpecialCharTok{==}\StringTok{"135E"}\NormalTok{]}\OtherTok{=}\StringTok{"X135E"}
\NormalTok{  dat.k }\OtherTok{=}\NormalTok{ dat.k }\SpecialCharTok{\%\textgreater{}\%} 
    \FunctionTok{filter}\NormalTok{(day}\SpecialCharTok{==}\DecValTok{21}\NormalTok{) }\SpecialCharTok{\%\textgreater{}\%} 
    \FunctionTok{filter}\NormalTok{(spec}\SpecialCharTok{==}\NormalTok{cur.spec) }\SpecialCharTok{\%\textgreater{}\%} 
    \FunctionTok{filter}\NormalTok{(cond}\SpecialCharTok{==}\StringTok{"alone"}\NormalTok{) }\SpecialCharTok{\%\textgreater{}\%} 
    \FunctionTok{filter}\NormalTok{(pH }\SpecialCharTok{==} \DecValTok{5}\NormalTok{)}
\NormalTok{  out.k }\OtherTok{=} \FunctionTok{lm}\NormalTok{(cfus }\SpecialCharTok{\textasciitilde{}} \DecValTok{1}\NormalTok{, }\AttributeTok{data =}\NormalTok{ dat.k )}
\NormalTok{  k.cur }\OtherTok{=}\NormalTok{ out.k}\SpecialCharTok{$}\NormalTok{coefficients[[}\DecValTok{1}\NormalTok{]]}
  \DocumentationTok{\#\# modify the data to ensure we\textquotesingle{}re accounting for k}
\NormalTok{  ind.use }\OtherTok{=}\NormalTok{ dat.fit[,}\FunctionTok{paste0}\NormalTok{(}\StringTok{"pres."}\NormalTok{,cur.spec)]}\SpecialCharTok{==}\DecValTok{1}
\NormalTok{  dat.cur }\OtherTok{=}\NormalTok{ dat.fit[ind.use,]}
\NormalTok{  dat.cur }\OtherTok{=}\NormalTok{ dat.cur }\SpecialCharTok{\%\textgreater{}\%} 
    \FunctionTok{filter}\NormalTok{(cond }\SpecialCharTok{!=}\StringTok{"alone"}\NormalTok{)}
\NormalTok{  dat.cur[,cur.spec]}\OtherTok{=}\NormalTok{dat.cur[,cur.spec]}\SpecialCharTok{{-}}\NormalTok{k.cur}
  \CommentTok{\#generate formula automatically}
  \CommentTok{\# Note that we are fitting a zero{-}intercept model, as we have already subtracted out K.}
  \CommentTok{\#}
\NormalTok{  form }\OtherTok{=} \FunctionTok{paste0}\NormalTok{(cur.spec,}\StringTok{" \textasciitilde{} {-}1 + "}\NormalTok{, }\FunctionTok{paste}\NormalTok{(spec.vecfit[}\SpecialCharTok{{-}}\NormalTok{i.spec], }\AttributeTok{collapse =} \StringTok{" + "}\NormalTok{))}
\NormalTok{  out.cur }\OtherTok{=} \FunctionTok{lm}\NormalTok{(}\FunctionTok{formula}\NormalTok{(form),}
               \AttributeTok{data =}\NormalTok{ dat.cur)}
\NormalTok{  cur.df }\OtherTok{=} \FunctionTok{data.frame}\NormalTok{(}\AttributeTok{est =} \FunctionTok{c}\NormalTok{(}\FunctionTok{coefficients}\NormalTok{(out.k),}\FunctionTok{coefficients}\NormalTok{(out.cur)),}
                      \AttributeTok{se =} \FunctionTok{c}\NormalTok{(}\FunctionTok{coef}\NormalTok{(}\FunctionTok{summary}\NormalTok{(out.k))[,}\DecValTok{2}\NormalTok{], }\FunctionTok{coef}\NormalTok{(}\FunctionTok{summary}\NormalTok{(out.cur))[,}\DecValTok{2}\NormalTok{]))}
  \CommentTok{\#switch to alphas (for everything but Ks, which are the first entry) by multiplying by {-}1}
\NormalTok{  cur.df}\SpecialCharTok{$}\NormalTok{est[}\SpecialCharTok{{-}}\DecValTok{1}\NormalTok{]}\OtherTok{=}\SpecialCharTok{{-}}\NormalTok{cur.df}\SpecialCharTok{$}\NormalTok{est[}\SpecialCharTok{{-}}\DecValTok{1}\NormalTok{]}
\NormalTok{  cur.df}\SpecialCharTok{$}\NormalTok{spec }\OtherTok{=}\NormalTok{ cur.spec}
\NormalTok{  cur.df}\SpecialCharTok{$}\NormalTok{name }\OtherTok{=} \FunctionTok{rownames}\NormalTok{(cur.df)}
\NormalTok{  cur.df}\SpecialCharTok{$}\NormalTok{pH }\OtherTok{=} \DecValTok{5}
  \FunctionTok{rownames}\NormalTok{(cur.df)}\OtherTok{=}\ConstantTok{NULL}
\NormalTok{  cur.df}\SpecialCharTok{$}\NormalTok{name[}\DecValTok{1}\NormalTok{] }\OtherTok{=} \StringTok{"K"}
  \CommentTok{\#reorder}
\NormalTok{  cur.df}\OtherTok{=}\NormalTok{cur.df[,}\FunctionTok{c}\NormalTok{(}\StringTok{"spec"}\NormalTok{,}\StringTok{"name"}\NormalTok{,}\StringTok{"est"}\NormalTok{,}\StringTok{"se"}\NormalTok{, }\StringTok{"pH"}\NormalTok{)]}
\NormalTok{  res.est }\OtherTok{=} \FunctionTok{rbind}\NormalTok{(res.est, cur.df)}
\NormalTok{\}}
\end{Highlighting}
\end{Shaded}

\begin{verbatim}
## [1] "BC9"
## [1] "BC10"
## [1] "JB5"
## [1] "JB7"
## [1] "X135E"
## [1] "JBC"
## [1] "JB370"
\end{verbatim}

\begin{Shaded}
\begin{Highlighting}[]
\DocumentationTok{\#\# and that\textquotesingle{}s our fits}
\DocumentationTok{\#\# }
\FunctionTok{write.csv}\NormalTok{(res.est, }
          \AttributeTok{file =} \FunctionTok{here}\NormalTok{(}\StringTok{"4\_res"}\NormalTok{,}
                      \StringTok{"coefficient{-}estimates{-}pH5.csv"}\NormalTok{),}
          \AttributeTok{row.names =} \ConstantTok{FALSE}
\NormalTok{)}
\DocumentationTok{\#\# write metadata}
\FunctionTok{cat}\NormalTok{(}\StringTok{"meta{-}data for coefficeint{-}estimates.csv"}\NormalTok{,}
    \AttributeTok{file =} \FunctionTok{here}\NormalTok{(}\StringTok{"4\_res"}\NormalTok{, }\StringTok{"coefficient{-}estimates{-}metadata.txt"}\NormalTok{),}
    \AttributeTok{sep =} \StringTok{"}\SpecialCharTok{\textbackslash{}n}\StringTok{"}\NormalTok{)}
\FunctionTok{cat}\NormalTok{(}\FunctionTok{c}\NormalTok{(}\StringTok{"Fitting Lotka{-}Volterra coefficients K and alpha through linear regression (assuming day 21 populations are at equilibrium"}\NormalTok{,}
      \StringTok{"We assume the error is normally distributed {-} this seems close enough to true."}\NormalTok{,}
      \StringTok{""}\NormalTok{,}
      \StringTok{"spec is the focal species"}\NormalTok{,}
      \StringTok{"name is the coefficient name. In the cases where name is a species name, it\textquotesingle{}s an alpha term, and the species named is the competing species"}\NormalTok{,}
      \StringTok{"est is the estimate FOR LOTKA VOLTERRA MODEL. So the K is in units of individuals, and the other terms are the ALPHAs, which are the coefficient estimates times negative 1."}\NormalTok{,}
      \StringTok{"se is the standard error from the lienar regression"}\NormalTok{),}
    \AttributeTok{file =} \FunctionTok{here}\NormalTok{(}\StringTok{"4\_res"}\NormalTok{, }\StringTok{"coefficient{-}estimates{-}pH5{-}metadata.txt"}\NormalTok{),}
    \AttributeTok{sep =} \StringTok{"}\SpecialCharTok{\textbackslash{}n}\StringTok{"}\NormalTok{,}
    \AttributeTok{append=}\NormalTok{T)}
\end{Highlighting}
\end{Shaded}

\hypertarget{visualizing-for-talk}{%
\paragraph{Visualizing for talk}\label{visualizing-for-talk}}

\begin{Shaded}
\begin{Highlighting}[]
\FunctionTok{library}\NormalTok{(scales)}
\end{Highlighting}
\end{Shaded}

\begin{verbatim}
## 
## Attaching package: 'scales'
\end{verbatim}

\begin{verbatim}
## The following object is masked from 'package:viridis':
## 
##     viridis_pal
\end{verbatim}

\begin{verbatim}
## The following object is masked from 'package:purrr':
## 
##     discard
\end{verbatim}

\begin{verbatim}
## The following object is masked from 'package:readr':
## 
##     col_factor
\end{verbatim}

\begin{Shaded}
\begin{Highlighting}[]
\CommentTok{\# res.heatmap = matrix({-}999, nrow = length(spec.vec),}
\CommentTok{\#                      ncol = length(spec.vec))}
\CommentTok{\# colnames(res.heatmap)=spec.vec}
\CommentTok{\# rownames(res.heatmap)=spec.vec}
\CommentTok{\# }
\CommentTok{\# for(cur.spec in unique(res.est$spec))\{}
\CommentTok{\#   res.est.use = res.est[res.est$spec == cur.spec \& res.est$name!="K",]}
\CommentTok{\#   res.heatmap[cur.spec,res.est.use$name] = res.est.use$est}
\CommentTok{\# \}}
\CommentTok{\# diag(res.heatmap) = NA}

\NormalTok{res.est}\SpecialCharTok{$}\NormalTok{sign }\OtherTok{=} \FunctionTok{sign}\NormalTok{(res.est}\SpecialCharTok{$}\NormalTok{est)}
\CommentTok{\# res.est$est.sc = sign(res.est$est)*log(abs(res.est$est))}

\FunctionTok{ggplot}\NormalTok{(res.est }\SpecialCharTok{\%\textgreater{}\%} \FunctionTok{filter}\NormalTok{(name }\SpecialCharTok{!=} \StringTok{"K"}\NormalTok{), }\FunctionTok{aes}\NormalTok{(}\AttributeTok{x =}\NormalTok{ name, }\AttributeTok{y =}\NormalTok{ spec)) }\SpecialCharTok{+}
  \FunctionTok{geom\_tile}\NormalTok{(}\FunctionTok{aes}\NormalTok{(}\AttributeTok{fill =}\NormalTok{ sign))}\SpecialCharTok{+}
  \FunctionTok{scale\_fill\_gradient2}\NormalTok{(}\AttributeTok{low =} \FunctionTok{muted}\NormalTok{(}\StringTok{"blue"}\NormalTok{),}
                       \AttributeTok{high =} \FunctionTok{muted}\NormalTok{(}\StringTok{"red"}\NormalTok{))}
\end{Highlighting}
\end{Shaded}

\includegraphics{succession-analysis-v1.2_files/figure-latex/unnamed-chunk-8-1.pdf}

\begin{Shaded}
\begin{Highlighting}[]
\NormalTok{temp }\OtherTok{=}\NormalTok{ res.est }\SpecialCharTok{\%\textgreater{}\%} \FunctionTok{filter}\NormalTok{(name }\SpecialCharTok{!=} \StringTok{"K"}\NormalTok{)}
\FunctionTok{range}\NormalTok{(temp}\SpecialCharTok{$}\NormalTok{est)}
\end{Highlighting}
\end{Shaded}

\begin{verbatim}
## [1] -47588.24  46310.34
\end{verbatim}

\begin{Shaded}
\begin{Highlighting}[]
\DocumentationTok{\#\# quick thought: scale by carrying capacity}
\NormalTok{res.sc }\OtherTok{=}\NormalTok{ res.est}
\ControlFlowTok{for}\NormalTok{(cur.name }\ControlFlowTok{in} \FunctionTok{unique}\NormalTok{(res.sc}\SpecialCharTok{$}\NormalTok{spec))\{}
\NormalTok{  res.sc[res.sc}\SpecialCharTok{$}\NormalTok{name}\SpecialCharTok{==}\NormalTok{cur.name, }\StringTok{"est"}\NormalTok{] }\OtherTok{=} 
\NormalTok{    res.sc[res.sc}\SpecialCharTok{$}\NormalTok{name}\SpecialCharTok{==}\NormalTok{cur.name, }\StringTok{"est"}\NormalTok{] }\SpecialCharTok{*} 
\NormalTok{    res.sc}\SpecialCharTok{$}\NormalTok{est[res.sc}\SpecialCharTok{$}\NormalTok{spec }\SpecialCharTok{==}\NormalTok{ cur.name }\SpecialCharTok{\&}\NormalTok{ res.sc}\SpecialCharTok{$}\NormalTok{name }\SpecialCharTok{==} \StringTok{"K"}\NormalTok{]}
\NormalTok{\}}

\NormalTok{res.sc}\SpecialCharTok{$}\NormalTok{est.sc }\OtherTok{=} \FunctionTok{sign}\NormalTok{(res.sc}\SpecialCharTok{$}\NormalTok{est)}\SpecialCharTok{*}\FunctionTok{log}\NormalTok{(}\FunctionTok{abs}\NormalTok{(res.sc}\SpecialCharTok{$}\NormalTok{est))}

\FunctionTok{ggplot}\NormalTok{(res.sc }\SpecialCharTok{\%\textgreater{}\%} \FunctionTok{filter}\NormalTok{(name }\SpecialCharTok{!=} \StringTok{"K"}\NormalTok{), }\FunctionTok{aes}\NormalTok{(}\AttributeTok{x =}\NormalTok{ name, }\AttributeTok{y =}\NormalTok{ spec)) }\SpecialCharTok{+}
  \FunctionTok{geom\_tile}\NormalTok{(}\FunctionTok{aes}\NormalTok{(}\AttributeTok{fill =}\NormalTok{ est.sc))}\SpecialCharTok{+}
  \FunctionTok{scale\_fill\_gradient2}\NormalTok{(}\AttributeTok{low =} \FunctionTok{muted}\NormalTok{(}\StringTok{"blue"}\NormalTok{),}
                       \AttributeTok{high =} \FunctionTok{muted}\NormalTok{(}\StringTok{"red"}\NormalTok{))}
\end{Highlighting}
\end{Shaded}

\includegraphics{succession-analysis-v1.2_files/figure-latex/unnamed-chunk-9-1.pdf}

\hypertarget{comparing-to-community-data}{%
\subsection{Comparing to community
data}\label{comparing-to-community-data}}

\hypertarget{restructure-data}{%
\subsubsection{restructure data}\label{restructure-data}}

\begin{Shaded}
\begin{Highlighting}[]
\NormalTok{raw.com }\OtherTok{=}\NormalTok{ raw }\SpecialCharTok{\%\textgreater{}\%} 
  \FunctionTok{filter}\NormalTok{(cond }\SpecialCharTok{==} \StringTok{"community"}\NormalTok{) }\SpecialCharTok{\%\textgreater{}\%} 
  \FunctionTok{filter}\NormalTok{(day }\SpecialCharTok{==} \DecValTok{21}\NormalTok{) }\SpecialCharTok{\%\textgreater{}\%} 
  \FunctionTok{filter}\NormalTok{(pH }\SpecialCharTok{==} \DecValTok{5}\NormalTok{)}

\NormalTok{nrep }\OtherTok{=} \FunctionTok{max}\NormalTok{(raw.com}\SpecialCharTok{$}\NormalTok{rep)}
\NormalTok{dat.com }\OtherTok{=} \FunctionTok{data.frame}\NormalTok{(}\AttributeTok{BC9 =} \FunctionTok{rep}\NormalTok{(}\SpecialCharTok{{-}}\DecValTok{99}\NormalTok{, nrep),}
                     \AttributeTok{BC10 =} \FunctionTok{rep}\NormalTok{(}\SpecialCharTok{{-}}\DecValTok{99}\NormalTok{, nrep),}
                     \AttributeTok{JB5 =} \FunctionTok{rep}\NormalTok{(}\SpecialCharTok{{-}}\DecValTok{99}\NormalTok{, nrep),}
                     \AttributeTok{JB7 =} \FunctionTok{rep}\NormalTok{(}\SpecialCharTok{{-}}\DecValTok{99}\NormalTok{, nrep),}
                     \AttributeTok{X135E =} \FunctionTok{rep}\NormalTok{(}\SpecialCharTok{{-}}\DecValTok{99}\NormalTok{, nrep),}
                     \AttributeTok{JBC =} \FunctionTok{rep}\NormalTok{(}\SpecialCharTok{{-}}\DecValTok{99}\NormalTok{, nrep),}
                     \AttributeTok{JB370 =} \FunctionTok{rep}\NormalTok{(}\SpecialCharTok{{-}}\DecValTok{99}\NormalTok{, nrep),}
                     \AttributeTok{rep =} \DecValTok{1}\SpecialCharTok{:}\NormalTok{nrep}
\NormalTok{)}


\DocumentationTok{\#\# pairs}
\NormalTok{raw.com}\SpecialCharTok{$}\NormalTok{spec[raw.com}\SpecialCharTok{$}\NormalTok{spec}\SpecialCharTok{==}\StringTok{"135E"}\NormalTok{] }\OtherTok{=} \StringTok{"X135E"}
\NormalTok{spec.vec }\OtherTok{=} \FunctionTok{as.character}\NormalTok{(}\FunctionTok{unique}\NormalTok{(raw.com}\SpecialCharTok{$}\NormalTok{spec))}

\ControlFlowTok{for}\NormalTok{(cur.spec }\ControlFlowTok{in}\NormalTok{ spec.vec)\{}
  \CommentTok{\#get cfus of current species}
\NormalTok{  dat.cur }\OtherTok{=}\NormalTok{ raw.com }\SpecialCharTok{\%\textgreater{}\%} 
    \FunctionTok{filter}\NormalTok{(spec }\SpecialCharTok{==}\NormalTok{ cur.spec) }
\NormalTok{  dat.com[dat.cur}\SpecialCharTok{$}\NormalTok{rep,cur.spec] }\OtherTok{=}\NormalTok{ dat.cur}\SpecialCharTok{$}\NormalTok{cfus}
\NormalTok{\}}
\DocumentationTok{\#\# add NAs}
\NormalTok{dat.com[dat.com}\SpecialCharTok{=={-}}\DecValTok{99}\NormalTok{]}\OtherTok{=}\ConstantTok{NA}


\FunctionTok{write.csv}\NormalTok{(dat.com, }
          \AttributeTok{file =} \FunctionTok{here}\NormalTok{(}\StringTok{"2\_data\_wrangling"}\NormalTok{,}\StringTok{"matrix{-}form{-}comdat.csv"}\NormalTok{),}
          \AttributeTok{row.names =} \ConstantTok{FALSE}
\NormalTok{)}
\end{Highlighting}
\end{Shaded}

\hypertarget{predicting}{%
\subsubsection{predicting}\label{predicting}}

\begin{Shaded}
\begin{Highlighting}[]
\CommentTok{\# read in our community data (e.g. what we saved last chunk)}
\NormalTok{dat.com}\OtherTok{=}\FunctionTok{read.csv}\NormalTok{(}\FunctionTok{here}\NormalTok{(}\StringTok{"2\_data\_wrangling"}\NormalTok{,}\StringTok{"matrix{-}form{-}comdat.csv"}\NormalTok{))}
\CommentTok{\# read in our fitted coeffients}
\NormalTok{dat.coefs }\OtherTok{=} \FunctionTok{read.csv}\NormalTok{(}\FunctionTok{here}\NormalTok{(}\StringTok{"4\_res"}\NormalTok{,}
                          \StringTok{"coefficient{-}estimates{-}pH5.csv"}\NormalTok{))}
\NormalTok{dat.com }\OtherTok{=} \FunctionTok{na.omit}\NormalTok{(dat.com)}
\NormalTok{dat.pred }\OtherTok{=}\NormalTok{ dat.com}
\NormalTok{dat.pred[,}\SpecialCharTok{{-}}\FunctionTok{which}\NormalTok{(}\FunctionTok{names}\NormalTok{(dat.com)}\SpecialCharTok{==}\StringTok{"rep"}\NormalTok{)]}\OtherTok{=}\DecValTok{0}
\ControlFlowTok{for}\NormalTok{(i.spec }\ControlFlowTok{in} \DecValTok{1}\SpecialCharTok{:}\NormalTok{(}\FunctionTok{length}\NormalTok{(spec.vecfit)))\{}
\NormalTok{  cur.spec }\OtherTok{=}\NormalTok{ spec.vecfit[i.spec]}
  \FunctionTok{print}\NormalTok{(cur.spec)}
  \CommentTok{\#some cool manipulating to make it easy to access the right coefficients}
  \CommentTok{\# ending in coefs.vec: a vector of coefficients labeled by name}
\NormalTok{  coefs.cur }\OtherTok{=}\NormalTok{ dat.coefs }\SpecialCharTok{\%\textgreater{}\%} 
    \FunctionTok{filter}\NormalTok{(spec }\SpecialCharTok{==}\NormalTok{ cur.spec)}
\NormalTok{  coefs.vec }\OtherTok{=}\NormalTok{ coefs.cur}\SpecialCharTok{$}\NormalTok{est}
  \FunctionTok{names}\NormalTok{(coefs.vec) }\OtherTok{=}\NormalTok{ coefs.cur}\SpecialCharTok{$}\NormalTok{name}
\NormalTok{  x.cur }\OtherTok{=}\NormalTok{ dat.com[,}\SpecialCharTok{{-}}\FunctionTok{which}\NormalTok{(}\FunctionTok{colnames}\NormalTok{(dat.com) }\SpecialCharTok{\%in\%} \FunctionTok{c}\NormalTok{(cur.spec, }\StringTok{"rep"}\NormalTok{))]}
  \CommentTok{\#using element{-}wise multiplication to simplify the coding}
  \CommentTok{\#This may look opaque, but you can confirm that it\textquotesingle{}s doing the right thing with }
  \CommentTok{\# a test case:}
  \CommentTok{\# test.mat = matrix(1:4, 2,2)}
  \CommentTok{\# t(t(test.mat) * c(1,0))}
\NormalTok{  pred.cur }\OtherTok{=}\NormalTok{ coefs.vec[}\StringTok{"K"}\NormalTok{] }\SpecialCharTok{{-}} 
    \FunctionTok{rowSums}\NormalTok{(}\FunctionTok{t}\NormalTok{(}\FunctionTok{t}\NormalTok{(x.cur[,}\FunctionTok{names}\NormalTok{(coefs.vec)[}\SpecialCharTok{{-}}\DecValTok{1}\NormalTok{]]) }\SpecialCharTok{*}\NormalTok{ (coefs.vec[}\SpecialCharTok{{-}}\DecValTok{1}\NormalTok{])))}
\NormalTok{  pred.cur }\OtherTok{=} \FunctionTok{pmax}\NormalTok{(}\DecValTok{0}\NormalTok{,pred.cur)}
\NormalTok{  dat.pred[,cur.spec] }\OtherTok{=}\NormalTok{ pred.cur}
\NormalTok{\}}
\end{Highlighting}
\end{Shaded}

\begin{verbatim}
## [1] "BC9"
## [1] "BC10"
## [1] "JB5"
## [1] "JB7"
## [1] "X135E"
## [1] "JBC"
## [1] "JB370"
\end{verbatim}

\begin{Shaded}
\begin{Highlighting}[]
\NormalTok{dat.pred}\SpecialCharTok{$}\NormalTok{ph }\OtherTok{=} \DecValTok{5}

\FunctionTok{write.csv}\NormalTok{(dat.pred,}
          \FunctionTok{here}\NormalTok{(}\StringTok{"4\_res"}\NormalTok{,}\StringTok{"predictions{-}from{-}pH5.csv"}\NormalTok{),}
          \AttributeTok{row.names=}\NormalTok{F}
\NormalTok{)}
\NormalTok{dat.pred}\SpecialCharTok{$}\NormalTok{scenario }\OtherTok{=} \StringTok{"ph 5 predictions"}
\NormalTok{pred.all }\OtherTok{=} \FunctionTok{rbind}\NormalTok{(pred.all, dat.pred)}
\end{Highlighting}
\end{Shaded}

So our predicted community not only doesn't match our actual community
very well, it also doesn't really match possible realities very well. We
have a lot of negative numbers.

\hypertarget{visualizing}{%
\subsubsection{Visualizing}\label{visualizing}}

\begin{Shaded}
\begin{Highlighting}[]
\NormalTok{dat.plot }\OtherTok{=} \FunctionTok{data.frame}\NormalTok{(}\AttributeTok{est =} \FunctionTok{apply}\NormalTok{(dat.pred,}\DecValTok{2}\NormalTok{, median))}
\end{Highlighting}
\end{Shaded}

\begin{verbatim}
## Warning in mean.default(sort(x, partial = half + 0L:1L)[half + 0L:1L]):
## argument is not numeric or logical: returning NA

## Warning in mean.default(sort(x, partial = half + 0L:1L)[half + 0L:1L]):
## argument is not numeric or logical: returning NA

## Warning in mean.default(sort(x, partial = half + 0L:1L)[half + 0L:1L]):
## argument is not numeric or logical: returning NA

## Warning in mean.default(sort(x, partial = half + 0L:1L)[half + 0L:1L]):
## argument is not numeric or logical: returning NA

## Warning in mean.default(sort(x, partial = half + 0L:1L)[half + 0L:1L]):
## argument is not numeric or logical: returning NA

## Warning in mean.default(sort(x, partial = half + 0L:1L)[half + 0L:1L]):
## argument is not numeric or logical: returning NA

## Warning in mean.default(sort(x, partial = half + 0L:1L)[half + 0L:1L]):
## argument is not numeric or logical: returning NA

## Warning in mean.default(sort(x, partial = half + 0L:1L)[half + 0L:1L]):
## argument is not numeric or logical: returning NA

## Warning in mean.default(sort(x, partial = half + 0L:1L)[half + 0L:1L]):
## argument is not numeric or logical: returning NA

## Warning in mean.default(sort(x, partial = half + 0L:1L)[half + 0L:1L]):
## argument is not numeric or logical: returning NA
\end{verbatim}

\begin{Shaded}
\begin{Highlighting}[]
\NormalTok{dat.plot}\SpecialCharTok{$}\NormalTok{spec }\OtherTok{=} \FunctionTok{colnames}\NormalTok{(dat.pred)}
\NormalTok{dat.plot }\OtherTok{=}\NormalTok{ dat.plot[}\SpecialCharTok{{-}}\FunctionTok{which}\NormalTok{(dat.plot}\SpecialCharTok{$}\NormalTok{spec }\SpecialCharTok{\%in\%} \FunctionTok{c}\NormalTok{(}\StringTok{"rep"}\NormalTok{,}\StringTok{"ph"}\NormalTok{)),]}
\NormalTok{dat.plot}\SpecialCharTok{$}\NormalTok{est }\OtherTok{=} \FunctionTok{pmax}\NormalTok{(dat.plot}\SpecialCharTok{$}\NormalTok{est, }\DecValTok{0}\NormalTok{)}
\NormalTok{dat.plot}\SpecialCharTok{$}\NormalTok{rel }\OtherTok{=}\NormalTok{ dat.plot}\SpecialCharTok{$}\NormalTok{est}\SpecialCharTok{/}\FunctionTok{sum}\NormalTok{(dat.plot}\SpecialCharTok{$}\NormalTok{est)}
\NormalTok{dat.plot}\SpecialCharTok{$}\NormalTok{scenario }\OtherTok{=} \StringTok{"ph 5 predictions"}
\NormalTok{dat.gg }\OtherTok{=}\NormalTok{ dat.plot}

\NormalTok{dat.plot }\OtherTok{=} \FunctionTok{data.frame}\NormalTok{(}\AttributeTok{est =} \FunctionTok{apply}\NormalTok{(dat.com,}\DecValTok{2}\NormalTok{, median))}
\NormalTok{dat.plot}\SpecialCharTok{$}\NormalTok{spec }\OtherTok{=} \FunctionTok{colnames}\NormalTok{(dat.com)}
\NormalTok{dat.plot }\OtherTok{=}\NormalTok{ dat.plot[}\SpecialCharTok{{-}}\FunctionTok{which}\NormalTok{(dat.plot}\SpecialCharTok{$}\NormalTok{spec }\SpecialCharTok{\%in\%} \FunctionTok{c}\NormalTok{(}\StringTok{"rep"}\NormalTok{,}\StringTok{"ph"}\NormalTok{)),]}
\NormalTok{dat.plot}\SpecialCharTok{$}\NormalTok{est }\OtherTok{=} \FunctionTok{pmax}\NormalTok{(dat.plot}\SpecialCharTok{$}\NormalTok{est, }\DecValTok{0}\NormalTok{)}
\NormalTok{dat.plot}\SpecialCharTok{$}\NormalTok{rel }\OtherTok{=}\NormalTok{ dat.plot}\SpecialCharTok{$}\NormalTok{est}\SpecialCharTok{/}\FunctionTok{sum}\NormalTok{(dat.plot}\SpecialCharTok{$}\NormalTok{est)}
\NormalTok{dat.plot}\SpecialCharTok{$}\NormalTok{scenario }\OtherTok{=} \StringTok{"Actual community"}
\NormalTok{dat.gg }\OtherTok{=} \FunctionTok{rbind}\NormalTok{(dat.gg, dat.plot)}

\FunctionTok{ggplot}\NormalTok{(}\AttributeTok{data =}\NormalTok{ dat.gg, }\FunctionTok{aes}\NormalTok{(}\AttributeTok{fill =}\NormalTok{ spec, }\AttributeTok{x =}\NormalTok{ scenario, }\AttributeTok{y =}\NormalTok{ rel))}\SpecialCharTok{+}
  \FunctionTok{geom\_col}\NormalTok{()}\SpecialCharTok{+}
  \FunctionTok{ylab}\NormalTok{(}\StringTok{"relative abundance"}\NormalTok{)}\SpecialCharTok{+}
\NormalTok{  theme.mine}
\end{Highlighting}
\end{Shaded}

\begin{verbatim}
## Warning: Removed 8 rows containing missing values (`position_stack()`).
\end{verbatim}

\includegraphics{succession-analysis-v1.2_files/figure-latex/unnamed-chunk-12-1.pdf}

\hypertarget{ph-7}{%
\section{pH 7}\label{ph-7}}

Now that we can fit Lotka Volterra with regression models, let's try
again, but with pH 7.

\hypertarget{quick-sanity-check-for-ks}{%
\subsection{quick sanity check for ks}\label{quick-sanity-check-for-ks}}

\begin{Shaded}
\begin{Highlighting}[]
\NormalTok{dat.orig }\SpecialCharTok{\%\textgreater{}\%}  
  \FunctionTok{filter}\NormalTok{(cond }\SpecialCharTok{==} \StringTok{"alone"}\NormalTok{) }\SpecialCharTok{\%\textgreater{}\%}  
  \FunctionTok{filter}\NormalTok{(day }\SpecialCharTok{==} \DecValTok{21}\NormalTok{) }\SpecialCharTok{\%\textgreater{}\%} 
  \FunctionTok{filter}\NormalTok{(pH }\SpecialCharTok{==}  \DecValTok{7}\NormalTok{) }\SpecialCharTok{\%\textgreater{}\%}
  \CommentTok{\# View() \%\textgreater{}\% }
  \FunctionTok{group\_by}\NormalTok{(spec) }\SpecialCharTok{\%\textgreater{}\%} 
  \FunctionTok{summarize}\NormalTok{(}\AttributeTok{cfus =} \FunctionTok{median}\NormalTok{(cfus))}
\end{Highlighting}
\end{Shaded}

\begin{verbatim}
## # A tibble: 7 x 2
##   spec        cfus
##   <chr>      <dbl>
## 1 135E     5300000
## 2 BC10   201000000
## 3 BC9    260000000
## 4 JB370    1100000
## 5 JB5   4000000000
## 6 JB7   1780000000
## 7 JBC     67000000
\end{verbatim}

\hypertarget{restructuring-the-data}{%
\subsection{Restructuring the data}\label{restructuring-the-data}}

\begin{Shaded}
\begin{Highlighting}[]
\CommentTok{\#make empty data frame}
\CommentTok{\#We want to store cfus for each species, }
\NormalTok{dat.mat7 }\OtherTok{=} \FunctionTok{setNames}\NormalTok{(}\FunctionTok{data.frame}\NormalTok{(}\FunctionTok{matrix}\NormalTok{(}\AttributeTok{ncol =} \DecValTok{2}\SpecialCharTok{*}\FunctionTok{length}\NormalTok{(}\FunctionTok{unique}\NormalTok{(raw}\SpecialCharTok{$}\NormalTok{spec))}\SpecialCharTok{+}\DecValTok{2}\NormalTok{, }\AttributeTok{nrow =} \DecValTok{0}\NormalTok{)),}
                    \FunctionTok{c}\NormalTok{(}\FunctionTok{as.character}\NormalTok{(}\FunctionTok{unique}\NormalTok{(raw}\SpecialCharTok{$}\NormalTok{spec)), }\StringTok{"cond"}\NormalTok{, }\StringTok{"rep"}\NormalTok{,}\FunctionTok{paste0}\NormalTok{(}\StringTok{"pres."}\NormalTok{,}\FunctionTok{c}\NormalTok{(}\FunctionTok{as.character}\NormalTok{(}\FunctionTok{unique}\NormalTok{(raw}\SpecialCharTok{$}\NormalTok{spec)))))}
\NormalTok{)}
\NormalTok{raw.use }\OtherTok{=}\NormalTok{ raw }\SpecialCharTok{\%\textgreater{}\%} 
  \FunctionTok{filter}\NormalTok{(day }\SpecialCharTok{==} \DecValTok{21}\NormalTok{) }\SpecialCharTok{\%\textgreater{}\%} 
  \FunctionTok{filter}\NormalTok{(pH }\SpecialCharTok{==} \DecValTok{7}\NormalTok{)}
\CommentTok{\#handle the alones}
\ControlFlowTok{for}\NormalTok{(cur.spec }\ControlFlowTok{in} \FunctionTok{unique}\NormalTok{(raw.use}\SpecialCharTok{$}\NormalTok{spec))\{}
  \FunctionTok{print}\NormalTok{(cur.spec)}
\NormalTok{  dat.cur }\OtherTok{=}\NormalTok{ raw.use }\SpecialCharTok{\%\textgreater{}\%} 
    \FunctionTok{filter}\NormalTok{(spec }\SpecialCharTok{==}\NormalTok{ cur.spec) }\SpecialCharTok{\%\textgreater{}\%} 
    \FunctionTok{filter}\NormalTok{(cond }\SpecialCharTok{==} \StringTok{"alone"}\NormalTok{)}
\NormalTok{  df.fill}\OtherTok{=}\FunctionTok{as.data.frame}\NormalTok{(}\FunctionTok{matrix}\NormalTok{(}\DecValTok{0}\NormalTok{, }
                               \AttributeTok{nrow=}\FunctionTok{nrow}\NormalTok{(dat.cur),}
                               \AttributeTok{ncol=}\FunctionTok{ncol}\NormalTok{(dat.mat7)))}
  \FunctionTok{names}\NormalTok{(df.fill) }\OtherTok{=} \FunctionTok{names}\NormalTok{(dat.mat7)}
\NormalTok{  df.fill[,cur.spec]}\OtherTok{=}\NormalTok{dat.cur}\SpecialCharTok{$}\NormalTok{cfus}
\NormalTok{  df.fill[,}\StringTok{"rep"}\NormalTok{]}\OtherTok{=}\NormalTok{dat.cur}\SpecialCharTok{$}\NormalTok{rep}
\NormalTok{  df.fill[,}\StringTok{"cond"}\NormalTok{]}\OtherTok{=}\StringTok{"alone"}
\NormalTok{  df.fill[,}\FunctionTok{paste0}\NormalTok{(}\StringTok{"pres."}\NormalTok{,cur.spec)] }\OtherTok{=}\NormalTok{ T}
\NormalTok{  dat.mat7 }\OtherTok{=} \FunctionTok{rbind}\NormalTok{(dat.mat7, df.fill)}
\NormalTok{\}}
\end{Highlighting}
\end{Shaded}

\begin{verbatim}
## [1] "BC9"
## [1] "BC10"
## [1] "JB5"
## [1] "JB7"
## [1] "135E"
## [1] "JBC"
## [1] "JB370"
\end{verbatim}

\begin{Shaded}
\begin{Highlighting}[]
\DocumentationTok{\#\# pairs}
\NormalTok{spec.vec }\OtherTok{=} \FunctionTok{as.character}\NormalTok{(}\FunctionTok{unique}\NormalTok{(raw.use}\SpecialCharTok{$}\NormalTok{spec))}

\CommentTok{\# Process all pairwise interaction experiments}
\ControlFlowTok{for}\NormalTok{(i.spec }\ControlFlowTok{in} \DecValTok{1}\SpecialCharTok{:}\NormalTok{(}\FunctionTok{length}\NormalTok{(spec.vec)}\SpecialCharTok{{-}}\DecValTok{1}\NormalTok{))\{ }\CommentTok{\#focal species}
\NormalTok{  cur.spec }\OtherTok{=}\NormalTok{ spec.vec[i.spec]}
  \ControlFlowTok{for}\NormalTok{(j.spec }\ControlFlowTok{in}\NormalTok{ (i.spec}\SpecialCharTok{+}\DecValTok{1}\NormalTok{)}\SpecialCharTok{:}\FunctionTok{length}\NormalTok{(spec.vec))\{}
\NormalTok{    other.spec }\OtherTok{=}\NormalTok{ spec.vec[j.spec]}
    \CommentTok{\#get cfus of current species}
\NormalTok{    dat.cur }\OtherTok{=}\NormalTok{ raw.use }\SpecialCharTok{\%\textgreater{}\%} 
      \FunctionTok{filter}\NormalTok{(spec }\SpecialCharTok{==}\NormalTok{ cur.spec) }\SpecialCharTok{\%\textgreater{}\%} 
      \FunctionTok{filter}\NormalTok{(cond }\SpecialCharTok{==}\NormalTok{ other.spec)}
    \CommentTok{\#get cfus of other species}
\NormalTok{    dat.other }\OtherTok{=}\NormalTok{ raw.use }\SpecialCharTok{\%\textgreater{}\%} 
      \FunctionTok{filter}\NormalTok{(spec }\SpecialCharTok{==}\NormalTok{ other.spec) }\SpecialCharTok{\%\textgreater{}\%} 
      \FunctionTok{filter}\NormalTok{(cond }\SpecialCharTok{==}\NormalTok{ cur.spec)}
    \CommentTok{\#The replicates might not be lined up. Make it so.}
\NormalTok{    dat.cur }\OtherTok{=}\NormalTok{ dat.cur[dat.cur}\SpecialCharTok{$}\NormalTok{rep }\SpecialCharTok{\%in\%}\NormalTok{ dat.other}\SpecialCharTok{$}\NormalTok{rep,]}
\NormalTok{    dat.other }\OtherTok{=}\NormalTok{ dat.other[dat.other}\SpecialCharTok{$}\NormalTok{rep }\SpecialCharTok{\%in\%}\NormalTok{ dat.cur}\SpecialCharTok{$}\NormalTok{rep,]}
\NormalTok{    dat.cur}\OtherTok{=}\NormalTok{dat.cur[}\FunctionTok{order}\NormalTok{(dat.cur}\SpecialCharTok{$}\NormalTok{rep),]}
\NormalTok{    dat.other}\OtherTok{=}\NormalTok{dat.other[}\FunctionTok{order}\NormalTok{(dat.other}\SpecialCharTok{$}\NormalTok{rep),]}
    
    \DocumentationTok{\#\# Sanity checking to make sure everything is lining up.}
    \ControlFlowTok{if}\NormalTok{(}\FunctionTok{nrow}\NormalTok{(dat.cur) }\SpecialCharTok{!=} \FunctionTok{nrow}\NormalTok{(dat.other))\{}
      \FunctionTok{stop}\NormalTok{(}\StringTok{"dimensions not matching up. investigate"}\NormalTok{)}
\NormalTok{    \}}
    \ControlFlowTok{if}\NormalTok{(}\FunctionTok{any}\NormalTok{(dat.cur}\SpecialCharTok{$}\NormalTok{rep }\SpecialCharTok{!=}\NormalTok{ dat.other}\SpecialCharTok{$}\NormalTok{rep))\{}
      \FunctionTok{stop}\NormalTok{(}\StringTok{"replicates not matching up. investigate"}\NormalTok{)}
\NormalTok{    \}}
    \DocumentationTok{\#\# if all is working, combine the pieces into a data frame for }
    \DocumentationTok{\#\#   the current species pair, then stitch to the overall data frame.}
\NormalTok{    df.fill}\OtherTok{=}\FunctionTok{as.data.frame}\NormalTok{(}\FunctionTok{matrix}\NormalTok{(}\DecValTok{0}\NormalTok{,}
                                 \AttributeTok{nrow=}\FunctionTok{nrow}\NormalTok{(dat.cur),}
                                 \AttributeTok{ncol=}\FunctionTok{ncol}\NormalTok{(dat.mat7)))}
    \FunctionTok{names}\NormalTok{(df.fill) }\OtherTok{=} \FunctionTok{names}\NormalTok{(dat.mat7)}
\NormalTok{    df.fill[ , cur.spec] }\OtherTok{=}\NormalTok{ dat.cur}\SpecialCharTok{$}\NormalTok{cfus}
\NormalTok{    df.fill[ , other.spec] }\OtherTok{=}\NormalTok{ dat.other}\SpecialCharTok{$}\NormalTok{cfus}
\NormalTok{    df.fill[,}\StringTok{"rep"}\NormalTok{]}\OtherTok{=}\NormalTok{dat.cur}\SpecialCharTok{$}\NormalTok{rep}
\NormalTok{    df.fill[,}\StringTok{"cond"}\NormalTok{]}\OtherTok{=}\StringTok{"paired{-}comp"}
\NormalTok{    df.fill[,}\FunctionTok{paste0}\NormalTok{(}\StringTok{"pres."}\NormalTok{,cur.spec)] }\OtherTok{=}\NormalTok{ T}
\NormalTok{    df.fill[,}\FunctionTok{paste0}\NormalTok{(}\StringTok{"pres."}\NormalTok{,other.spec)] }\OtherTok{=}\NormalTok{ T}
\NormalTok{    dat.mat7 }\OtherTok{=} \FunctionTok{rbind}\NormalTok{(dat.mat7, df.fill)}
\NormalTok{  \}}
\NormalTok{\}}
\NormalTok{dat.mat7}\SpecialCharTok{$}\NormalTok{day }\OtherTok{=} \DecValTok{21}
\NormalTok{dat.mat7}\SpecialCharTok{$}\NormalTok{pH }\OtherTok{=} \DecValTok{7}
\FunctionTok{names}\NormalTok{(dat.mat7)[}\FunctionTok{names}\NormalTok{(dat.mat7)}\SpecialCharTok{==}\StringTok{"pres.135E"}\NormalTok{]}\OtherTok{=}\StringTok{"pres.X135E"}

\FunctionTok{write.csv}\NormalTok{(dat.mat7, }
          \AttributeTok{file =} \FunctionTok{here}\NormalTok{(}\StringTok{"2\_data\_wrangling"}\NormalTok{,}\StringTok{"matrix{-}form{-}pH7.csv"}\NormalTok{),}
          \AttributeTok{row.names =} \ConstantTok{FALSE}
\NormalTok{)}
\end{Highlighting}
\end{Shaded}

\hypertarget{fit-the-model}{%
\subsection{Fit the model}\label{fit-the-model}}

\begin{Shaded}
\begin{Highlighting}[]
\NormalTok{dat.fit }\OtherTok{=} \FunctionTok{read.csv}\NormalTok{(}\AttributeTok{file =} \FunctionTok{here}\NormalTok{(}\StringTok{"2\_data\_wrangling"}\NormalTok{,}\StringTok{"matrix{-}form{-}pH7.csv"}\NormalTok{))}
\NormalTok{res.est }\OtherTok{=} \ConstantTok{NULL}
\NormalTok{spec.vecfit }\OtherTok{=} \FunctionTok{unique}\NormalTok{(raw}\SpecialCharTok{$}\NormalTok{spec)}
\NormalTok{spec.vecfit[spec.vecfit}\SpecialCharTok{==}\StringTok{"135E"}\NormalTok{] }\OtherTok{=} \StringTok{"X135E"}
\ControlFlowTok{for}\NormalTok{(i.spec }\ControlFlowTok{in} \DecValTok{1}\SpecialCharTok{:}\FunctionTok{length}\NormalTok{(spec.vec))\{}
\NormalTok{  cur.spec }\OtherTok{=}\NormalTok{ spec.vecfit[i.spec]}
  \FunctionTok{print}\NormalTok{(cur.spec)}
  
  \DocumentationTok{\#\# calculate K}
\NormalTok{  dat.k }\OtherTok{=}\NormalTok{ dat.orig }
\NormalTok{  dat.k}\SpecialCharTok{$}\NormalTok{spec[dat.k}\SpecialCharTok{$}\NormalTok{spec}\SpecialCharTok{==}\StringTok{"135E"}\NormalTok{]}\OtherTok{=}\StringTok{"X135E"}
\NormalTok{  dat.k }\OtherTok{=}\NormalTok{ dat.k }\SpecialCharTok{\%\textgreater{}\%} 
    \FunctionTok{filter}\NormalTok{(day}\SpecialCharTok{==}\DecValTok{21}\NormalTok{) }\SpecialCharTok{\%\textgreater{}\%} 
    \FunctionTok{filter}\NormalTok{(spec}\SpecialCharTok{==}\NormalTok{cur.spec) }\SpecialCharTok{\%\textgreater{}\%} 
    \FunctionTok{filter}\NormalTok{(cond}\SpecialCharTok{==}\StringTok{"alone"}\NormalTok{) }\SpecialCharTok{\%\textgreater{}\%} 
    \FunctionTok{filter}\NormalTok{(pH }\SpecialCharTok{==} \DecValTok{7}\NormalTok{)}
\NormalTok{  out.k }\OtherTok{=} \FunctionTok{lm}\NormalTok{(cfus }\SpecialCharTok{\textasciitilde{}} \DecValTok{1}\NormalTok{, }\AttributeTok{data =}\NormalTok{ dat.k )}
\NormalTok{  k.cur }\OtherTok{=}\NormalTok{ out.k}\SpecialCharTok{$}\NormalTok{coefficients[[}\DecValTok{1}\NormalTok{]]}
  
  \DocumentationTok{\#\# modify the data to ensure we\textquotesingle{}re accounting for K}
\NormalTok{  ind.use }\OtherTok{=}\NormalTok{ dat.fit[,}\FunctionTok{paste0}\NormalTok{(}\StringTok{"pres."}\NormalTok{,cur.spec)]}\SpecialCharTok{==}\DecValTok{1}
\NormalTok{  dat.cur }\OtherTok{=}\NormalTok{ dat.fit[ind.use,]}
\NormalTok{  dat.cur[,cur.spec]}\OtherTok{=}\NormalTok{dat.cur[,cur.spec]}\SpecialCharTok{{-}}\NormalTok{k.cur}
  \CommentTok{\#generate formula automatically}
  \CommentTok{\# Note that we are fit a zero{-}intercept model, as we have already subtracted out K.}
  \CommentTok{\#}
\NormalTok{  form }\OtherTok{=} \FunctionTok{paste0}\NormalTok{(cur.spec,}\StringTok{" \textasciitilde{} {-}1 + "}\NormalTok{, }\FunctionTok{paste}\NormalTok{(spec.vecfit[}\SpecialCharTok{{-}}\NormalTok{i.spec], }\AttributeTok{collapse =} \StringTok{" + "}\NormalTok{))}
\NormalTok{  out.cur }\OtherTok{=} \FunctionTok{lm}\NormalTok{(}\FunctionTok{formula}\NormalTok{(form),}
               \AttributeTok{data =}\NormalTok{ dat.cur)}
\NormalTok{  est }\OtherTok{=} \FunctionTok{coefficients}\NormalTok{(out.cur)}
  \CommentTok{\#sometimes we\textquotesingle{}re getting NAs for se}
  \CommentTok{\#and this doesn\textquotesingle{}t even show up in coef(summary())}
  \CommentTok{\# so we have to do something a little fiddly.}
\NormalTok{  se }\OtherTok{=}\NormalTok{ est}\SpecialCharTok{*}\ConstantTok{NA}
\NormalTok{  coef.temp }\OtherTok{=} \FunctionTok{coef}\NormalTok{(}\FunctionTok{summary}\NormalTok{(out.cur))[,}\DecValTok{2}\NormalTok{]}
\NormalTok{  se[}\FunctionTok{names}\NormalTok{(coef.temp)] }\OtherTok{=}\NormalTok{ coef.temp}
\NormalTok{  cur.df }\OtherTok{=} \FunctionTok{data.frame}\NormalTok{(}\AttributeTok{est =} \FunctionTok{c}\NormalTok{(}\FunctionTok{coefficients}\NormalTok{(out.k), est),}
                      \AttributeTok{se =} \FunctionTok{c}\NormalTok{(}\FunctionTok{coef}\NormalTok{(}\FunctionTok{summary}\NormalTok{(out.k))[,}\DecValTok{2}\NormalTok{],se))}
  \CommentTok{\#switch to alphas (for everything but Ks, which are the first entry) by multiplying by {-}1}
\NormalTok{  cur.df}\SpecialCharTok{$}\NormalTok{est[}\SpecialCharTok{{-}}\DecValTok{1}\NormalTok{]}\OtherTok{=}\SpecialCharTok{{-}}\NormalTok{cur.df}\SpecialCharTok{$}\NormalTok{est[}\SpecialCharTok{{-}}\DecValTok{1}\NormalTok{]}
\NormalTok{  cur.df}\SpecialCharTok{$}\NormalTok{spec }\OtherTok{=}\NormalTok{ cur.spec}
\NormalTok{  cur.df}\SpecialCharTok{$}\NormalTok{name }\OtherTok{=} \FunctionTok{rownames}\NormalTok{(cur.df)}
\NormalTok{  cur.df}\SpecialCharTok{$}\NormalTok{pH }\OtherTok{=} \DecValTok{7}
  \FunctionTok{rownames}\NormalTok{(cur.df)}\OtherTok{=}\ConstantTok{NULL}
\NormalTok{  cur.df}\SpecialCharTok{$}\NormalTok{name[}\DecValTok{1}\NormalTok{] }\OtherTok{=} \StringTok{"K"}
  \CommentTok{\#reorder}
\NormalTok{  cur.df}\OtherTok{=}\NormalTok{cur.df[,}\FunctionTok{c}\NormalTok{(}\StringTok{"spec"}\NormalTok{,}\StringTok{"name"}\NormalTok{,}\StringTok{"est"}\NormalTok{,}\StringTok{"se"}\NormalTok{, }\StringTok{"pH"}\NormalTok{)]}
\NormalTok{  res.est }\OtherTok{=} \FunctionTok{rbind}\NormalTok{(res.est, cur.df)}
\NormalTok{\}}
\end{Highlighting}
\end{Shaded}

\begin{verbatim}
## [1] "BC9"
## [1] "BC10"
## [1] "JB5"
## [1] "JB7"
## [1] "X135E"
## [1] "JBC"
## [1] "JB370"
\end{verbatim}

\begin{Shaded}
\begin{Highlighting}[]
\FunctionTok{write.csv}\NormalTok{(res.est, }
          \AttributeTok{file =} \FunctionTok{here}\NormalTok{(}\StringTok{"4\_res"}\NormalTok{,}
                      \StringTok{"coefficient{-}estimates{-}pH7.csv"}\NormalTok{),}
          \AttributeTok{row.names =} \ConstantTok{FALSE}
\NormalTok{)}
\DocumentationTok{\#\# write metadata}
\FunctionTok{cat}\NormalTok{(}\StringTok{"meta{-}data for coefficeint{-}estimates.csv"}\NormalTok{,}
    \AttributeTok{file =} \FunctionTok{here}\NormalTok{(}\StringTok{"4\_res"}\NormalTok{, }\StringTok{"coefficient{-}estimates{-}pH7{-}metadata.txt"}\NormalTok{),}
    \AttributeTok{sep =} \StringTok{"}\SpecialCharTok{\textbackslash{}n}\StringTok{"}\NormalTok{)}
\FunctionTok{cat}\NormalTok{(}\FunctionTok{c}\NormalTok{(}\StringTok{"Fitting Lotka{-}Volterra coefficients K and alpha through linear regression (assuming day 21 populations are at equilibrium"}\NormalTok{,}
      \StringTok{"As before, but now we are doing pH7 data"}\NormalTok{,}
      \StringTok{"We assume the error is normally distributed {-} this seems close enough to true."}\NormalTok{,}
      \StringTok{""}\NormalTok{,}
      \StringTok{"spec is the focal species"}\NormalTok{,}
      \StringTok{"name is the coefficient name. In the cases where name is a species name, it\textquotesingle{}s an alpha term, and the species named is the competing species"}\NormalTok{,}
      \StringTok{"est is the estimate FOR LOTKA VOLTERRA MODEL. So the K is in units of individuals, and the other terms are the ALPHAs, which are the coefficient estimates times negative 1."}\NormalTok{,}
      \StringTok{"se is the standard error from the lienar regression"}\NormalTok{),}
    \AttributeTok{file =} \FunctionTok{here}\NormalTok{(}\StringTok{"4\_res"}\NormalTok{, }\StringTok{"coefficient{-}estimates{-}pH7{-}metadata.txt"}\NormalTok{),}
    \AttributeTok{sep =} \StringTok{"}\SpecialCharTok{\textbackslash{}n}\StringTok{"}\NormalTok{,}
    \AttributeTok{append=}\NormalTok{T)}
\end{Highlighting}
\end{Shaded}

\hypertarget{comparing-to-community-data-1}{%
\subsection{Comparing to community
data}\label{comparing-to-community-data-1}}

\hypertarget{restructure-data-1}{%
\subsubsection{restructure data}\label{restructure-data-1}}

We already did this for the pH5 predictions, and we're using the same
community data. We can just read in \texttt{matrix-form-comdat.csv}.

\begin{Shaded}
\begin{Highlighting}[]
\NormalTok{dat.com }\OtherTok{=} \FunctionTok{read.csv}\NormalTok{(}\AttributeTok{file =} \FunctionTok{here}\NormalTok{(}\StringTok{"2\_data\_wrangling"}\NormalTok{,}\StringTok{"matrix{-}form{-}comdat.csv"}\NormalTok{)}
\NormalTok{)}
\end{Highlighting}
\end{Shaded}

\hypertarget{predicting-1}{%
\subsubsection{predicting}\label{predicting-1}}

Note that we have no estimate for the effect of 135E on JBC since it
never survives, BUT we don't have it in the final community. I'm
converting all estimates of interactions coming from X135E to 0 in the
calculations to avoid NA propogation.

\begin{Shaded}
\begin{Highlighting}[]
\NormalTok{dat.com}\OtherTok{=}\FunctionTok{read.csv}\NormalTok{(}\FunctionTok{here}\NormalTok{(}\StringTok{"2\_data\_wrangling"}\NormalTok{,}\StringTok{"matrix{-}form{-}comdat.csv"}\NormalTok{))}
\NormalTok{dat.com}\OtherTok{=}\FunctionTok{na.omit}\NormalTok{(dat.com)}
\CommentTok{\# read in our fitted coeffients}
\NormalTok{dat.coefs }\OtherTok{=} \FunctionTok{read.csv}\NormalTok{(}\FunctionTok{here}\NormalTok{(}\StringTok{"4\_res"}\NormalTok{,}
                          \StringTok{"coefficient{-}estimates{-}pH7.csv"}\NormalTok{))}
\CommentTok{\#make a data fram to store predictions {-} easiest way is to grab existing community data frame and empty it.}
\NormalTok{dat.pred }\OtherTok{=}\NormalTok{ dat.com}
\NormalTok{dat.pred[,}\SpecialCharTok{{-}}\FunctionTok{which}\NormalTok{(}\FunctionTok{names}\NormalTok{(dat.com)}\SpecialCharTok{==}\StringTok{"rep"}\NormalTok{)]}\OtherTok{=}\DecValTok{0}
\ControlFlowTok{for}\NormalTok{(i.spec }\ControlFlowTok{in} \DecValTok{1}\SpecialCharTok{:}\FunctionTok{length}\NormalTok{(spec.vecfit))\{}
\NormalTok{  cur.spec }\OtherTok{=}\NormalTok{ spec.vecfit[i.spec]}
  \FunctionTok{print}\NormalTok{(cur.spec)}
  \CommentTok{\#some cool manipulating to make it easy to access the right coefficients}
  \CommentTok{\# ending in coefs.vec: a vector of coefficients labeled by name}
\NormalTok{  coefs.cur }\OtherTok{=}\NormalTok{ dat.coefs }\SpecialCharTok{\%\textgreater{}\%} 
    \FunctionTok{filter}\NormalTok{(spec }\SpecialCharTok{==}\NormalTok{ cur.spec)}
\NormalTok{  coefs.vec }\OtherTok{=}\NormalTok{ coefs.cur}\SpecialCharTok{$}\NormalTok{est}
  \FunctionTok{names}\NormalTok{(coefs.vec) }\OtherTok{=}\NormalTok{ coefs.cur}\SpecialCharTok{$}\NormalTok{name}
  \ControlFlowTok{if}\NormalTok{(cur.spec }\SpecialCharTok{==}\StringTok{"JBC"}\NormalTok{)\{}
\NormalTok{    coefs.vec[}\StringTok{"X135E"}\NormalTok{]}\OtherTok{=}\DecValTok{0}
\NormalTok{  \}}
\NormalTok{  x.cur }\OtherTok{=}\NormalTok{ dat.com[,}\SpecialCharTok{{-}}\FunctionTok{which}\NormalTok{(}\FunctionTok{colnames}\NormalTok{(dat.com) }\SpecialCharTok{\%in\%} \FunctionTok{c}\NormalTok{(cur.spec, }\StringTok{"rep"}\NormalTok{))]}
  \CommentTok{\#using element{-}wise multiplication to simplify the coding}
  \CommentTok{\# (using some transformations (the \textasciigrave{}t()\textasciigrave{} function) to make the right pieces line}
  \CommentTok{\# up)}
  \CommentTok{\#This may look opaque, but you can confirm that it\textquotesingle{}s doing the right thing with }
  \CommentTok{\# a test case:}
  \CommentTok{\# test.mat = matrix(1:4, 2,2)}
  \CommentTok{\# t(t(test.mat) * c(1,0))}
\NormalTok{  pred.cur }\OtherTok{=}\NormalTok{ coefs.vec[}\StringTok{"K"}\NormalTok{] }\SpecialCharTok{{-}} 
    \FunctionTok{rowSums}\NormalTok{(}\FunctionTok{t}\NormalTok{(}\FunctionTok{t}\NormalTok{(x.cur[,}\FunctionTok{names}\NormalTok{(coefs.vec)[}\SpecialCharTok{{-}}\DecValTok{1}\NormalTok{]]) }\SpecialCharTok{*}\NormalTok{ (coefs.vec[}\SpecialCharTok{{-}}\DecValTok{1}\NormalTok{])))}
\NormalTok{  pred.cur }\OtherTok{=} \FunctionTok{pmax}\NormalTok{(}\DecValTok{0}\NormalTok{,pred.cur)}
\NormalTok{  dat.pred[,cur.spec] }\OtherTok{=}\NormalTok{ pred.cur}
\NormalTok{\}}
\end{Highlighting}
\end{Shaded}

\begin{verbatim}
## [1] "BC9"
## [1] "BC10"
## [1] "JB5"
## [1] "JB7"
## [1] "X135E"
## [1] "JBC"
## [1] "JB370"
\end{verbatim}

\begin{Shaded}
\begin{Highlighting}[]
\NormalTok{dat.pred}\SpecialCharTok{$}\NormalTok{ph }\OtherTok{=} \DecValTok{7}
\FunctionTok{write.csv}\NormalTok{(dat.pred,}
          \FunctionTok{here}\NormalTok{(}\StringTok{"4\_res"}\NormalTok{,}\StringTok{"predictions{-}from{-}pH7.csv"}\NormalTok{),}
          \AttributeTok{row.names=}\NormalTok{F}
\NormalTok{)}
\NormalTok{dat.pred}\SpecialCharTok{$}\NormalTok{scenario }\OtherTok{=} \StringTok{"ph 7 predictions"}
\NormalTok{pred.all }\OtherTok{=} \FunctionTok{rbind}\NormalTok{(pred.all, dat.pred)}
\end{Highlighting}
\end{Shaded}

\hypertarget{visualizing-1}{%
\subsubsection{Visualizing}\label{visualizing-1}}

Adding to our existing set of predictions

\begin{Shaded}
\begin{Highlighting}[]
\NormalTok{dat.plot }\OtherTok{=} \FunctionTok{data.frame}\NormalTok{(}\AttributeTok{est =} \FunctionTok{apply}\NormalTok{(dat.pred,}\DecValTok{2}\NormalTok{, median))}
\end{Highlighting}
\end{Shaded}

\begin{verbatim}
## Warning in mean.default(sort(x, partial = half + 0L:1L)[half + 0L:1L]):
## argument is not numeric or logical: returning NA

## Warning in mean.default(sort(x, partial = half + 0L:1L)[half + 0L:1L]):
## argument is not numeric or logical: returning NA

## Warning in mean.default(sort(x, partial = half + 0L:1L)[half + 0L:1L]):
## argument is not numeric or logical: returning NA

## Warning in mean.default(sort(x, partial = half + 0L:1L)[half + 0L:1L]):
## argument is not numeric or logical: returning NA

## Warning in mean.default(sort(x, partial = half + 0L:1L)[half + 0L:1L]):
## argument is not numeric or logical: returning NA

## Warning in mean.default(sort(x, partial = half + 0L:1L)[half + 0L:1L]):
## argument is not numeric or logical: returning NA

## Warning in mean.default(sort(x, partial = half + 0L:1L)[half + 0L:1L]):
## argument is not numeric or logical: returning NA

## Warning in mean.default(sort(x, partial = half + 0L:1L)[half + 0L:1L]):
## argument is not numeric or logical: returning NA

## Warning in mean.default(sort(x, partial = half + 0L:1L)[half + 0L:1L]):
## argument is not numeric or logical: returning NA

## Warning in mean.default(sort(x, partial = half + 0L:1L)[half + 0L:1L]):
## argument is not numeric or logical: returning NA
\end{verbatim}

\begin{Shaded}
\begin{Highlighting}[]
\NormalTok{dat.plot}\SpecialCharTok{$}\NormalTok{spec }\OtherTok{=} \FunctionTok{colnames}\NormalTok{(dat.pred)}
\NormalTok{dat.plot }\OtherTok{=}\NormalTok{ dat.plot[}\SpecialCharTok{{-}}\FunctionTok{which}\NormalTok{(dat.plot}\SpecialCharTok{$}\NormalTok{spec }\SpecialCharTok{\%in\%} \FunctionTok{c}\NormalTok{(}\StringTok{"rep"}\NormalTok{,}\StringTok{"ph"}\NormalTok{)),]}
\NormalTok{dat.plot}\SpecialCharTok{$}\NormalTok{est }\OtherTok{=} \FunctionTok{pmax}\NormalTok{(dat.plot}\SpecialCharTok{$}\NormalTok{est, }\DecValTok{0}\NormalTok{)}
\NormalTok{dat.plot}\SpecialCharTok{$}\NormalTok{rel }\OtherTok{=}\NormalTok{ dat.plot}\SpecialCharTok{$}\NormalTok{est}\SpecialCharTok{/}\FunctionTok{sum}\NormalTok{(dat.plot}\SpecialCharTok{$}\NormalTok{est)}
\NormalTok{dat.plot}\SpecialCharTok{$}\NormalTok{scenario }\OtherTok{=} \StringTok{"ph 7 predictions"}
\NormalTok{dat.gg }\OtherTok{=} \FunctionTok{rbind}\NormalTok{(dat.gg, dat.plot)}

\FunctionTok{ggplot}\NormalTok{(}\AttributeTok{data =}\NormalTok{ dat.gg, }\FunctionTok{aes}\NormalTok{(}\AttributeTok{fill =}\NormalTok{ spec, }\AttributeTok{x =}\NormalTok{ scenario, }\AttributeTok{y =}\NormalTok{ rel))}\SpecialCharTok{+}
  \FunctionTok{geom\_col}\NormalTok{()}\SpecialCharTok{+}
  \FunctionTok{ylab}\NormalTok{(}\StringTok{"relative abundance"}\NormalTok{)}\SpecialCharTok{+}
\NormalTok{  theme.mine}
\end{Highlighting}
\end{Shaded}

\begin{verbatim}
## Warning: Removed 16 rows containing missing values (`position_stack()`).
\end{verbatim}

\includegraphics{succession-analysis-v1.2_files/figure-latex/unnamed-chunk-18-1.pdf}

\hypertarget{other-predictions}{%
\section{Other predictions}\label{other-predictions}}

pH 7 has better correspondence to our actual community, but it's far
from correct. Brooke has some hypotheses on this.

\hypertarget{jbc-coefficient-ph-7}{%
\subsection{JBC coefficient @ pH 7}\label{jbc-coefficient-ph-7}}

What if only JBC had any effects on other species?

\begin{Shaded}
\begin{Highlighting}[]
\NormalTok{dat.com}\OtherTok{=}\FunctionTok{read.csv}\NormalTok{(}\FunctionTok{here}\NormalTok{(}\StringTok{"2\_data\_wrangling"}\NormalTok{,}\StringTok{"matrix{-}form{-}comdat.csv"}\NormalTok{))}
\NormalTok{dat.com}\OtherTok{=}\FunctionTok{na.omit}\NormalTok{(dat.com)}
\CommentTok{\# read in our fitted coeffients}
\NormalTok{dat.coefs }\OtherTok{=} \FunctionTok{read.csv}\NormalTok{(}\FunctionTok{here}\NormalTok{(}\StringTok{"4\_res"}\NormalTok{,}
                          \StringTok{"coefficient{-}estimates{-}pH7.csv"}\NormalTok{))}
\DocumentationTok{\#\# turn anything that\textquotesingle{}s not a K or from JBC to 0}
\NormalTok{dat.coefs[}\SpecialCharTok{!}\NormalTok{dat.coefs}\SpecialCharTok{$}\NormalTok{name }\SpecialCharTok{\%in\%} \FunctionTok{c}\NormalTok{(}\StringTok{"K"}\NormalTok{,}\StringTok{"JBC"}\NormalTok{), }\StringTok{"est"}\NormalTok{]}\OtherTok{=}\DecValTok{0}
\NormalTok{dat.pred }\OtherTok{=}\NormalTok{ dat.com}
\NormalTok{dat.pred[,}\SpecialCharTok{{-}}\FunctionTok{which}\NormalTok{(}\FunctionTok{names}\NormalTok{(dat.com)}\SpecialCharTok{==}\StringTok{"rep"}\NormalTok{)]}\OtherTok{=}\DecValTok{0}
\ControlFlowTok{for}\NormalTok{(i.spec }\ControlFlowTok{in} \DecValTok{1}\SpecialCharTok{:}\FunctionTok{length}\NormalTok{(spec.vecfit))\{}
\NormalTok{  cur.spec }\OtherTok{=}\NormalTok{ spec.vecfit[i.spec]}
  \FunctionTok{print}\NormalTok{(cur.spec)}
  \CommentTok{\#some cool manipulating to make it easy to access the right coefficients}
  \CommentTok{\# ending in coefs.vec: a vector of coefficients labeled by name}
\NormalTok{  coefs.cur }\OtherTok{=}\NormalTok{ dat.coefs }\SpecialCharTok{\%\textgreater{}\%} 
    \FunctionTok{filter}\NormalTok{(spec }\SpecialCharTok{==}\NormalTok{ cur.spec)}
\NormalTok{  coefs.vec }\OtherTok{=}\NormalTok{ coefs.cur}\SpecialCharTok{$}\NormalTok{est}
  \FunctionTok{names}\NormalTok{(coefs.vec) }\OtherTok{=}\NormalTok{ coefs.cur}\SpecialCharTok{$}\NormalTok{name}
  \ControlFlowTok{if}\NormalTok{(cur.spec }\SpecialCharTok{==}\StringTok{"JBC"}\NormalTok{)\{}
\NormalTok{    coefs.vec[}\StringTok{"X135E"}\NormalTok{]}\OtherTok{=}\DecValTok{0}
\NormalTok{  \}}
\NormalTok{  x.cur }\OtherTok{=}\NormalTok{ dat.com[,}\SpecialCharTok{{-}}\FunctionTok{which}\NormalTok{(}\FunctionTok{colnames}\NormalTok{(dat.com) }\SpecialCharTok{\%in\%} \FunctionTok{c}\NormalTok{(cur.spec, }\StringTok{"rep"}\NormalTok{))]}
  \CommentTok{\#using element{-}wise multiplication to simplify the coding}
  \CommentTok{\#This may look opaque, but you can confirm that it\textquotesingle{}s doing the right thing with }
  \CommentTok{\# a test case:}
  \CommentTok{\# test.mat = matrix(1:4, 2,2)}
  \CommentTok{\# t(t(test.mat) * c(1,0))}
\NormalTok{  pred.cur }\OtherTok{=}\NormalTok{ coefs.vec[}\StringTok{"K"}\NormalTok{] }\SpecialCharTok{{-}} 
    \FunctionTok{rowSums}\NormalTok{(}\FunctionTok{t}\NormalTok{(}\FunctionTok{t}\NormalTok{(x.cur[,}\FunctionTok{names}\NormalTok{(coefs.vec)[}\SpecialCharTok{{-}}\DecValTok{1}\NormalTok{]]) }\SpecialCharTok{*}\NormalTok{ (coefs.vec[}\SpecialCharTok{{-}}\DecValTok{1}\NormalTok{])))}
\NormalTok{  pred.cur }\OtherTok{=} \FunctionTok{pmax}\NormalTok{(}\DecValTok{0}\NormalTok{,pred.cur)}
\NormalTok{  dat.pred[,cur.spec] }\OtherTok{=}\NormalTok{ pred.cur}
\NormalTok{\}}
\end{Highlighting}
\end{Shaded}

\begin{verbatim}
## [1] "BC9"
## [1] "BC10"
## [1] "JB5"
## [1] "JB7"
## [1] "X135E"
## [1] "JBC"
## [1] "JB370"
\end{verbatim}

\begin{Shaded}
\begin{Highlighting}[]
\NormalTok{dat.pred}\SpecialCharTok{$}\NormalTok{ph }\OtherTok{=} \DecValTok{7}

\FunctionTok{write.csv}\NormalTok{(dat.pred,}
          \FunctionTok{here}\NormalTok{(}\StringTok{"4\_res"}\NormalTok{,}\StringTok{"predictions{-}onlyJBC{-}pH7.csv"}\NormalTok{),}
          \AttributeTok{row.names=}\NormalTok{F}
\NormalTok{)}
\NormalTok{dat.pred}\SpecialCharTok{$}\NormalTok{scenario }\OtherTok{=} \StringTok{"ph 7 only{-}JBC predictions"}
\NormalTok{pred.all }\OtherTok{=} \FunctionTok{rbind}\NormalTok{(pred.all, dat.pred)}
\end{Highlighting}
\end{Shaded}

Add to our quick visualization dataframe

\begin{Shaded}
\begin{Highlighting}[]
\NormalTok{dat.plot }\OtherTok{=} \FunctionTok{data.frame}\NormalTok{(}\AttributeTok{est =} \FunctionTok{apply}\NormalTok{(dat.pred,}\DecValTok{2}\NormalTok{, median))}
\end{Highlighting}
\end{Shaded}

\begin{verbatim}
## Warning in mean.default(sort(x, partial = half + 0L:1L)[half + 0L:1L]):
## argument is not numeric or logical: returning NA

## Warning in mean.default(sort(x, partial = half + 0L:1L)[half + 0L:1L]):
## argument is not numeric or logical: returning NA

## Warning in mean.default(sort(x, partial = half + 0L:1L)[half + 0L:1L]):
## argument is not numeric or logical: returning NA

## Warning in mean.default(sort(x, partial = half + 0L:1L)[half + 0L:1L]):
## argument is not numeric or logical: returning NA

## Warning in mean.default(sort(x, partial = half + 0L:1L)[half + 0L:1L]):
## argument is not numeric or logical: returning NA

## Warning in mean.default(sort(x, partial = half + 0L:1L)[half + 0L:1L]):
## argument is not numeric or logical: returning NA

## Warning in mean.default(sort(x, partial = half + 0L:1L)[half + 0L:1L]):
## argument is not numeric or logical: returning NA

## Warning in mean.default(sort(x, partial = half + 0L:1L)[half + 0L:1L]):
## argument is not numeric or logical: returning NA

## Warning in mean.default(sort(x, partial = half + 0L:1L)[half + 0L:1L]):
## argument is not numeric or logical: returning NA

## Warning in mean.default(sort(x, partial = half + 0L:1L)[half + 0L:1L]):
## argument is not numeric or logical: returning NA
\end{verbatim}

\begin{Shaded}
\begin{Highlighting}[]
\NormalTok{dat.plot}\SpecialCharTok{$}\NormalTok{spec }\OtherTok{=} \FunctionTok{colnames}\NormalTok{(dat.pred)}
\NormalTok{dat.plot }\OtherTok{=}\NormalTok{ dat.plot[}\SpecialCharTok{{-}}\FunctionTok{which}\NormalTok{(dat.plot}\SpecialCharTok{$}\NormalTok{spec }\SpecialCharTok{\%in\%} \FunctionTok{c}\NormalTok{(}\StringTok{"rep"}\NormalTok{,}\StringTok{"ph"}\NormalTok{)),]}
\NormalTok{dat.plot}\SpecialCharTok{$}\NormalTok{est }\OtherTok{=} \FunctionTok{pmax}\NormalTok{(dat.plot}\SpecialCharTok{$}\NormalTok{est, }\DecValTok{0}\NormalTok{)}
\NormalTok{dat.plot}\SpecialCharTok{$}\NormalTok{rel }\OtherTok{=}\NormalTok{ dat.plot}\SpecialCharTok{$}\NormalTok{est}\SpecialCharTok{/}\FunctionTok{sum}\NormalTok{(dat.plot}\SpecialCharTok{$}\NormalTok{est)}
\NormalTok{dat.plot}\SpecialCharTok{$}\NormalTok{scenario }\OtherTok{=} \StringTok{"ph 7 only{-}JBC predictions"}
\NormalTok{dat.gg }\OtherTok{=} \FunctionTok{rbind}\NormalTok{(dat.gg, dat.plot)}
\end{Highlighting}
\end{Shaded}

\hypertarget{jbc-135e-ph-5}{%
\subsection{JBC + 135E @ pH 5}\label{jbc-135e-ph-5}}

\begin{Shaded}
\begin{Highlighting}[]
\NormalTok{dat.com}\OtherTok{=}\FunctionTok{read.csv}\NormalTok{(}\FunctionTok{here}\NormalTok{(}\StringTok{"2\_data\_wrangling"}\NormalTok{,}\StringTok{"matrix{-}form{-}comdat.csv"}\NormalTok{))}
\NormalTok{dat.com}\OtherTok{=}\FunctionTok{na.omit}\NormalTok{(dat.com)}
\CommentTok{\# read in our fitted coeffients}
\NormalTok{dat.coefs }\OtherTok{=} \FunctionTok{read.csv}\NormalTok{(}\FunctionTok{here}\NormalTok{(}\StringTok{"4\_res"}\NormalTok{,}
                          \StringTok{"coefficient{-}estimates{-}pH5.csv"}\NormalTok{))}
\DocumentationTok{\#\# turn anything that\textquotesingle{}s not a K or from JBC to 0}
\NormalTok{dat.coefs[}\SpecialCharTok{!}\NormalTok{dat.coefs}\SpecialCharTok{$}\NormalTok{name }\SpecialCharTok{\%in\%} \FunctionTok{c}\NormalTok{(}\StringTok{"K"}\NormalTok{,}\StringTok{"JBC"}\NormalTok{, }\StringTok{"X135E"}\NormalTok{), }\StringTok{"est"}\NormalTok{]}\OtherTok{=}\DecValTok{0}
\NormalTok{dat.pred }\OtherTok{=}\NormalTok{ dat.com}
\NormalTok{dat.pred[,}\SpecialCharTok{{-}}\FunctionTok{which}\NormalTok{(}\FunctionTok{names}\NormalTok{(dat.com)}\SpecialCharTok{==}\StringTok{"rep"}\NormalTok{)]}\OtherTok{=}\DecValTok{0}
\ControlFlowTok{for}\NormalTok{(i.spec }\ControlFlowTok{in} \DecValTok{1}\SpecialCharTok{:}\FunctionTok{length}\NormalTok{(spec.vecfit))\{}
\NormalTok{  cur.spec }\OtherTok{=}\NormalTok{ spec.vecfit[i.spec]}
  \FunctionTok{print}\NormalTok{(cur.spec)}
  \CommentTok{\#some cool manipulating to make it easy to access the right coefficients}
  \CommentTok{\# ending in coefs.vec: a vector of coefficients labeled by name}
\NormalTok{  coefs.cur }\OtherTok{=}\NormalTok{ dat.coefs }\SpecialCharTok{\%\textgreater{}\%} 
    \FunctionTok{filter}\NormalTok{(spec }\SpecialCharTok{==}\NormalTok{ cur.spec)}
\NormalTok{  coefs.vec }\OtherTok{=}\NormalTok{ coefs.cur}\SpecialCharTok{$}\NormalTok{est}
  \FunctionTok{names}\NormalTok{(coefs.vec) }\OtherTok{=}\NormalTok{ coefs.cur}\SpecialCharTok{$}\NormalTok{name}
  \ControlFlowTok{if}\NormalTok{(cur.spec }\SpecialCharTok{==}\StringTok{"JBC"}\NormalTok{)\{}
\NormalTok{    coefs.vec[}\StringTok{"X135E"}\NormalTok{]}\OtherTok{=}\DecValTok{0}
\NormalTok{  \}}
\NormalTok{  x.cur }\OtherTok{=}\NormalTok{ dat.com[,}\SpecialCharTok{{-}}\FunctionTok{which}\NormalTok{(}\FunctionTok{colnames}\NormalTok{(dat.com) }\SpecialCharTok{\%in\%} \FunctionTok{c}\NormalTok{(cur.spec, }\StringTok{"rep"}\NormalTok{))]}
  \CommentTok{\#using element{-}wise multiplication to simplify the coding}
  \CommentTok{\#This may look opaque, but you can confirm that it\textquotesingle{}s doing the right thing with }
  \CommentTok{\# a test case:}
  \CommentTok{\# test.mat = matrix(1:4, 2,2)}
  \CommentTok{\# t(t(test.mat) * c(1,0))}
\NormalTok{  pred.cur }\OtherTok{=}\NormalTok{ coefs.vec[}\StringTok{"K"}\NormalTok{] }\SpecialCharTok{{-}} 
    \FunctionTok{rowSums}\NormalTok{(}\FunctionTok{t}\NormalTok{(}\FunctionTok{t}\NormalTok{(x.cur[,}\FunctionTok{names}\NormalTok{(coefs.vec)[}\SpecialCharTok{{-}}\DecValTok{1}\NormalTok{]]) }\SpecialCharTok{*}\NormalTok{ (coefs.vec[}\SpecialCharTok{{-}}\DecValTok{1}\NormalTok{])))}
\NormalTok{  pred.cur }\OtherTok{=} \FunctionTok{pmax}\NormalTok{(}\DecValTok{0}\NormalTok{,pred.cur)}
\NormalTok{  dat.pred[,cur.spec] }\OtherTok{=}\NormalTok{ pred.cur}
\NormalTok{\}}
\end{Highlighting}
\end{Shaded}

\begin{verbatim}
## [1] "BC9"
## [1] "BC10"
## [1] "JB5"
## [1] "JB7"
## [1] "X135E"
## [1] "JBC"
## [1] "JB370"
\end{verbatim}

\begin{Shaded}
\begin{Highlighting}[]
\NormalTok{dat.pred}\SpecialCharTok{$}\NormalTok{ph }\OtherTok{=} \DecValTok{5}

\FunctionTok{write.csv}\NormalTok{(dat.pred,}
          \FunctionTok{here}\NormalTok{(}\StringTok{"4\_res"}\NormalTok{,}\StringTok{"predictions{-}onlyJBC{-}135E{-}pH5.csv"}\NormalTok{),}
          \AttributeTok{row.names=}\NormalTok{F}
\NormalTok{)}
\NormalTok{dat.pred}\SpecialCharTok{$}\NormalTok{scenario }\OtherTok{=} \StringTok{"ph 5 only{-}JBC+135E predictions"}
\NormalTok{pred.all }\OtherTok{=} \FunctionTok{rbind}\NormalTok{(pred.all, dat.pred)}
\end{Highlighting}
\end{Shaded}

Add to our quick visualization dataframe

\begin{Shaded}
\begin{Highlighting}[]
\NormalTok{dat.plot }\OtherTok{=} \FunctionTok{data.frame}\NormalTok{(}\AttributeTok{est =} \FunctionTok{apply}\NormalTok{(dat.pred,}\DecValTok{2}\NormalTok{, median))}
\end{Highlighting}
\end{Shaded}

\begin{verbatim}
## Warning in mean.default(sort(x, partial = half + 0L:1L)[half + 0L:1L]):
## argument is not numeric or logical: returning NA

## Warning in mean.default(sort(x, partial = half + 0L:1L)[half + 0L:1L]):
## argument is not numeric or logical: returning NA

## Warning in mean.default(sort(x, partial = half + 0L:1L)[half + 0L:1L]):
## argument is not numeric or logical: returning NA

## Warning in mean.default(sort(x, partial = half + 0L:1L)[half + 0L:1L]):
## argument is not numeric or logical: returning NA

## Warning in mean.default(sort(x, partial = half + 0L:1L)[half + 0L:1L]):
## argument is not numeric or logical: returning NA

## Warning in mean.default(sort(x, partial = half + 0L:1L)[half + 0L:1L]):
## argument is not numeric or logical: returning NA

## Warning in mean.default(sort(x, partial = half + 0L:1L)[half + 0L:1L]):
## argument is not numeric or logical: returning NA

## Warning in mean.default(sort(x, partial = half + 0L:1L)[half + 0L:1L]):
## argument is not numeric or logical: returning NA

## Warning in mean.default(sort(x, partial = half + 0L:1L)[half + 0L:1L]):
## argument is not numeric or logical: returning NA

## Warning in mean.default(sort(x, partial = half + 0L:1L)[half + 0L:1L]):
## argument is not numeric or logical: returning NA
\end{verbatim}

\begin{Shaded}
\begin{Highlighting}[]
\NormalTok{dat.plot}\SpecialCharTok{$}\NormalTok{spec }\OtherTok{=} \FunctionTok{colnames}\NormalTok{(dat.pred)}
\NormalTok{dat.plot }\OtherTok{=}\NormalTok{ dat.plot[}\SpecialCharTok{{-}}\FunctionTok{which}\NormalTok{(dat.plot}\SpecialCharTok{$}\NormalTok{spec }\SpecialCharTok{\%in\%} \FunctionTok{c}\NormalTok{(}\StringTok{"rep"}\NormalTok{,}\StringTok{"ph"}\NormalTok{)),]}
\NormalTok{dat.plot}\SpecialCharTok{$}\NormalTok{est }\OtherTok{=} \FunctionTok{pmax}\NormalTok{(dat.plot}\SpecialCharTok{$}\NormalTok{est, }\DecValTok{0}\NormalTok{)}
\NormalTok{dat.plot}\SpecialCharTok{$}\NormalTok{rel }\OtherTok{=}\NormalTok{ dat.plot}\SpecialCharTok{$}\NormalTok{est}\SpecialCharTok{/}\FunctionTok{sum}\NormalTok{(dat.plot}\SpecialCharTok{$}\NormalTok{est)}
\NormalTok{dat.plot}\SpecialCharTok{$}\NormalTok{scenario }\OtherTok{=} \StringTok{"ph 5 only{-}JBC+135E predictions"}
\NormalTok{dat.gg }\OtherTok{=} \FunctionTok{rbind}\NormalTok{(dat.gg, dat.plot)}
\end{Highlighting}
\end{Shaded}

\hypertarget{mse}{%
\section{MSE}\label{mse}}

\begin{Shaded}
\begin{Highlighting}[]
\NormalTok{real }\OtherTok{=}\NormalTok{ dat.gg }\SpecialCharTok{\%\textgreater{}\%} 
  \FunctionTok{filter}\NormalTok{(scenario }\SpecialCharTok{==} \StringTok{"Actual community"}\NormalTok{)}

\FunctionTok{print}\NormalTok{(}\StringTok{"ph 5 all MSE"}\NormalTok{)}
\end{Highlighting}
\end{Shaded}

\begin{verbatim}
## [1] "ph 5 all MSE"
\end{verbatim}

\begin{Shaded}
\begin{Highlighting}[]
\NormalTok{pred.cur }\OtherTok{=}\NormalTok{ dat.gg }\SpecialCharTok{\%\textgreater{}\%} 
  \FunctionTok{filter}\NormalTok{(scenario }\SpecialCharTok{==} \StringTok{"ph 5 predictions"}\NormalTok{)}
\FunctionTok{mean}\NormalTok{((real}\SpecialCharTok{$}\NormalTok{est}\SpecialCharTok{{-}}\NormalTok{pred.cur}\SpecialCharTok{$}\NormalTok{est)}\SpecialCharTok{\^{}}\DecValTok{2}\NormalTok{)}
\end{Highlighting}
\end{Shaded}

\begin{verbatim}
## Warning in real$est - pred.cur$est: longer object length is not a multiple of
## shorter object length
\end{verbatim}

\begin{verbatim}
## [1] NA
\end{verbatim}

\begin{Shaded}
\begin{Highlighting}[]
\FunctionTok{print}\NormalTok{(}\StringTok{"ph 7 all MSE"}\NormalTok{)}
\end{Highlighting}
\end{Shaded}

\begin{verbatim}
## [1] "ph 7 all MSE"
\end{verbatim}

\begin{Shaded}
\begin{Highlighting}[]
\NormalTok{pred.cur }\OtherTok{=}\NormalTok{ dat.gg }\SpecialCharTok{\%\textgreater{}\%} 
  \FunctionTok{filter}\NormalTok{(scenario }\SpecialCharTok{==} \StringTok{"ph 7 predictions"}\NormalTok{)}
\FunctionTok{mean}\NormalTok{((real}\SpecialCharTok{$}\NormalTok{est}\SpecialCharTok{{-}}\NormalTok{pred.cur}\SpecialCharTok{$}\NormalTok{est)}\SpecialCharTok{\^{}}\DecValTok{2}\NormalTok{)}
\end{Highlighting}
\end{Shaded}

\begin{verbatim}
## Warning in real$est - pred.cur$est: longer object length is not a multiple of
## shorter object length
\end{verbatim}

\begin{verbatim}
## [1] NA
\end{verbatim}

\begin{Shaded}
\begin{Highlighting}[]
\FunctionTok{print}\NormalTok{(}\StringTok{"ph 7 one effector"}\NormalTok{)}
\end{Highlighting}
\end{Shaded}

\begin{verbatim}
## [1] "ph 7 one effector"
\end{verbatim}

\begin{Shaded}
\begin{Highlighting}[]
\NormalTok{pred.cur }\OtherTok{=}\NormalTok{ dat.gg }\SpecialCharTok{\%\textgreater{}\%} 
  \FunctionTok{filter}\NormalTok{(scenario }\SpecialCharTok{==} \StringTok{"ph 7 only{-}JBC predictions"}\NormalTok{)}
\FunctionTok{mean}\NormalTok{((real}\SpecialCharTok{$}\NormalTok{est}\SpecialCharTok{{-}}\NormalTok{pred.cur}\SpecialCharTok{$}\NormalTok{est)}\SpecialCharTok{\^{}}\DecValTok{2}\NormalTok{)}
\end{Highlighting}
\end{Shaded}

\begin{verbatim}
## Warning in real$est - pred.cur$est: longer object length is not a multiple of
## shorter object length
\end{verbatim}

\begin{verbatim}
## [1] NA
\end{verbatim}

\begin{Shaded}
\begin{Highlighting}[]
\FunctionTok{print}\NormalTok{(}\StringTok{"ph 5 two effectors"}\NormalTok{)}
\end{Highlighting}
\end{Shaded}

\begin{verbatim}
## [1] "ph 5 two effectors"
\end{verbatim}

\begin{Shaded}
\begin{Highlighting}[]
\NormalTok{pred.cur }\OtherTok{=}\NormalTok{ dat.gg }\SpecialCharTok{\%\textgreater{}\%} 
  \FunctionTok{filter}\NormalTok{(scenario }\SpecialCharTok{==} \StringTok{"ph 5 only{-}JBC+135E predictions"}\NormalTok{)}
\FunctionTok{mean}\NormalTok{((real}\SpecialCharTok{$}\NormalTok{est}\SpecialCharTok{{-}}\NormalTok{pred.cur}\SpecialCharTok{$}\NormalTok{est)}\SpecialCharTok{\^{}}\DecValTok{2}\NormalTok{)}
\end{Highlighting}
\end{Shaded}

\begin{verbatim}
## Warning in real$est - pred.cur$est: longer object length is not a multiple of
## shorter object length
\end{verbatim}

\begin{verbatim}
## [1] NA
\end{verbatim}

\hypertarget{final-visualization}{%
\section{Final visualization}\label{final-visualization}}

\begin{Shaded}
\begin{Highlighting}[]
\FunctionTok{ggplot}\NormalTok{(}\AttributeTok{data =}\NormalTok{ dat.gg, }\FunctionTok{aes}\NormalTok{(}\AttributeTok{fill =}\NormalTok{ spec, }\AttributeTok{x =}\NormalTok{ scenario, }\AttributeTok{y =}\NormalTok{ rel))}\SpecialCharTok{+}
  \FunctionTok{geom\_col}\NormalTok{()}\SpecialCharTok{+}
  \FunctionTok{ylab}\NormalTok{(}\StringTok{"relative abundance"}\NormalTok{)}\SpecialCharTok{+}
\NormalTok{  theme.mine}
\end{Highlighting}
\end{Shaded}

\begin{verbatim}
## Warning: Removed 32 rows containing missing values (`position_stack()`).
\end{verbatim}

\includegraphics{succession-analysis-v1.2_files/figure-latex/unnamed-chunk-24-1.pdf}

\hypertarget{updated-checking}{%
\section{Updated checking}\label{updated-checking}}

From Brooke, 2/18/23: \emph{I poked around your modeling code to try
asking whether ANY taxon capable of deacidifying the media could pair
with JBC in the pH 5-based model to get the same results as the ph5
2-effector scenario (JBC, JB370, and BC10 are all deacidifiers, so I
tried JBC alone @ ph5, JBC+JB370, and JBC+BC10).}

TO check, I'll add in the 3 new scenarios:

\hypertarget{jbc-ph-5}{%
\subsection{JBC @ pH 5}\label{jbc-ph-5}}

\begin{Shaded}
\begin{Highlighting}[]
\NormalTok{dat.com}\OtherTok{=}\FunctionTok{read.csv}\NormalTok{(}\FunctionTok{here}\NormalTok{(}\StringTok{"2\_data\_wrangling"}\NormalTok{,}\StringTok{"matrix{-}form{-}comdat.csv"}\NormalTok{))}
\NormalTok{dat.com}\OtherTok{=}\FunctionTok{na.omit}\NormalTok{(dat.com)}
\CommentTok{\# read in our fitted coeffients}
\NormalTok{dat.coefs }\OtherTok{=} \FunctionTok{read.csv}\NormalTok{(}\FunctionTok{here}\NormalTok{(}\StringTok{"4\_res"}\NormalTok{,}
                          \StringTok{"coefficient{-}estimates{-}pH5.csv"}\NormalTok{))}
\DocumentationTok{\#\# turn anything that\textquotesingle{}s not a K or from JBC to 0}
\NormalTok{dat.coefs[}\SpecialCharTok{!}\NormalTok{dat.coefs}\SpecialCharTok{$}\NormalTok{name }\SpecialCharTok{\%in\%} \FunctionTok{c}\NormalTok{(}\StringTok{"K"}\NormalTok{,}\StringTok{"JBC"}\NormalTok{), }\StringTok{"est"}\NormalTok{]}\OtherTok{=}\DecValTok{0}
\NormalTok{dat.pred }\OtherTok{=}\NormalTok{ dat.com}
\NormalTok{dat.pred[,}\SpecialCharTok{{-}}\FunctionTok{which}\NormalTok{(}\FunctionTok{names}\NormalTok{(dat.com)}\SpecialCharTok{==}\StringTok{"rep"}\NormalTok{)]}\OtherTok{=}\DecValTok{0}
\ControlFlowTok{for}\NormalTok{(i.spec }\ControlFlowTok{in} \DecValTok{1}\SpecialCharTok{:}\FunctionTok{length}\NormalTok{(spec.vecfit))\{}
\NormalTok{  cur.spec }\OtherTok{=}\NormalTok{ spec.vecfit[i.spec]}
  \FunctionTok{print}\NormalTok{(cur.spec)}
  \CommentTok{\#some cool manipulating to make it easy to access the right coefficients}
  \CommentTok{\# ending in coefs.vec: a vector of coefficients labeled by name}
\NormalTok{  coefs.cur }\OtherTok{=}\NormalTok{ dat.coefs }\SpecialCharTok{\%\textgreater{}\%} 
    \FunctionTok{filter}\NormalTok{(spec }\SpecialCharTok{==}\NormalTok{ cur.spec)}
\NormalTok{  coefs.vec }\OtherTok{=}\NormalTok{ coefs.cur}\SpecialCharTok{$}\NormalTok{est}
  \FunctionTok{names}\NormalTok{(coefs.vec) }\OtherTok{=}\NormalTok{ coefs.cur}\SpecialCharTok{$}\NormalTok{name}
  \ControlFlowTok{if}\NormalTok{(cur.spec }\SpecialCharTok{==}\StringTok{"JBC"}\NormalTok{)\{}
\NormalTok{    coefs.vec[}\StringTok{"X135E"}\NormalTok{]}\OtherTok{=}\DecValTok{0}
\NormalTok{  \}}
\NormalTok{  x.cur }\OtherTok{=}\NormalTok{ dat.com[,}\SpecialCharTok{{-}}\FunctionTok{which}\NormalTok{(}\FunctionTok{colnames}\NormalTok{(dat.com) }\SpecialCharTok{\%in\%} \FunctionTok{c}\NormalTok{(cur.spec, }\StringTok{"rep"}\NormalTok{))]}
  \CommentTok{\#using element{-}wise multiplication to simplify the coding}
  \CommentTok{\#This may look opaque, but you can confirm that it\textquotesingle{}s doing the right thing with }
  \CommentTok{\# a test case:}
  \CommentTok{\# test.mat = matrix(1:4, 2,2)}
  \CommentTok{\# t(t(test.mat) * c(1,0))}
\NormalTok{  pred.cur }\OtherTok{=}\NormalTok{ coefs.vec[}\StringTok{"K"}\NormalTok{] }\SpecialCharTok{{-}} 
    \FunctionTok{rowSums}\NormalTok{(}\FunctionTok{t}\NormalTok{(}\FunctionTok{t}\NormalTok{(x.cur[,}\FunctionTok{names}\NormalTok{(coefs.vec)[}\SpecialCharTok{{-}}\DecValTok{1}\NormalTok{]]) }\SpecialCharTok{*}\NormalTok{ (coefs.vec[}\SpecialCharTok{{-}}\DecValTok{1}\NormalTok{])))}
\NormalTok{  pred.cur }\OtherTok{=} \FunctionTok{pmax}\NormalTok{(}\DecValTok{0}\NormalTok{,pred.cur)}
\NormalTok{  dat.pred[,cur.spec] }\OtherTok{=}\NormalTok{ pred.cur}
\NormalTok{\}}
\end{Highlighting}
\end{Shaded}

\begin{verbatim}
## [1] "BC9"
## [1] "BC10"
## [1] "JB5"
## [1] "JB7"
## [1] "X135E"
## [1] "JBC"
## [1] "JB370"
\end{verbatim}

\begin{Shaded}
\begin{Highlighting}[]
\NormalTok{dat.pred}\SpecialCharTok{$}\NormalTok{ph }\OtherTok{=} \DecValTok{5}

\FunctionTok{write.csv}\NormalTok{(dat.pred,}
          \FunctionTok{here}\NormalTok{(}\StringTok{"4\_res"}\NormalTok{,}\StringTok{"predictions{-}onlyJBC{-}pH5.csv"}\NormalTok{),}
          \AttributeTok{row.names=}\NormalTok{F}
\NormalTok{)}
\NormalTok{dat.pred}\SpecialCharTok{$}\NormalTok{scenario }\OtherTok{=} \StringTok{"ph 5 only{-}JBC predictions"}
\NormalTok{pred.all }\OtherTok{=} \FunctionTok{rbind}\NormalTok{(pred.all, dat.pred)}
\end{Highlighting}
\end{Shaded}

Add to our quick visualization dataframe

\begin{Shaded}
\begin{Highlighting}[]
\NormalTok{dat.plot }\OtherTok{=} \FunctionTok{data.frame}\NormalTok{(}\AttributeTok{est =} \FunctionTok{apply}\NormalTok{(dat.pred,}\DecValTok{2}\NormalTok{, median))}
\end{Highlighting}
\end{Shaded}

\begin{verbatim}
## Warning in mean.default(sort(x, partial = half + 0L:1L)[half + 0L:1L]):
## argument is not numeric or logical: returning NA

## Warning in mean.default(sort(x, partial = half + 0L:1L)[half + 0L:1L]):
## argument is not numeric or logical: returning NA

## Warning in mean.default(sort(x, partial = half + 0L:1L)[half + 0L:1L]):
## argument is not numeric or logical: returning NA

## Warning in mean.default(sort(x, partial = half + 0L:1L)[half + 0L:1L]):
## argument is not numeric or logical: returning NA

## Warning in mean.default(sort(x, partial = half + 0L:1L)[half + 0L:1L]):
## argument is not numeric or logical: returning NA

## Warning in mean.default(sort(x, partial = half + 0L:1L)[half + 0L:1L]):
## argument is not numeric or logical: returning NA

## Warning in mean.default(sort(x, partial = half + 0L:1L)[half + 0L:1L]):
## argument is not numeric or logical: returning NA

## Warning in mean.default(sort(x, partial = half + 0L:1L)[half + 0L:1L]):
## argument is not numeric or logical: returning NA

## Warning in mean.default(sort(x, partial = half + 0L:1L)[half + 0L:1L]):
## argument is not numeric or logical: returning NA

## Warning in mean.default(sort(x, partial = half + 0L:1L)[half + 0L:1L]):
## argument is not numeric or logical: returning NA
\end{verbatim}

\begin{Shaded}
\begin{Highlighting}[]
\NormalTok{dat.plot}\SpecialCharTok{$}\NormalTok{spec }\OtherTok{=} \FunctionTok{colnames}\NormalTok{(dat.pred)}
\NormalTok{dat.plot }\OtherTok{=}\NormalTok{ dat.plot[}\SpecialCharTok{{-}}\FunctionTok{which}\NormalTok{(dat.plot}\SpecialCharTok{$}\NormalTok{spec }\SpecialCharTok{\%in\%} \FunctionTok{c}\NormalTok{(}\StringTok{"rep"}\NormalTok{,}\StringTok{"ph"}\NormalTok{)),]}
\NormalTok{dat.plot}\SpecialCharTok{$}\NormalTok{est }\OtherTok{=} \FunctionTok{pmax}\NormalTok{(dat.plot}\SpecialCharTok{$}\NormalTok{est, }\DecValTok{0}\NormalTok{)}
\NormalTok{dat.plot}\SpecialCharTok{$}\NormalTok{rel }\OtherTok{=}\NormalTok{ dat.plot}\SpecialCharTok{$}\NormalTok{est}\SpecialCharTok{/}\FunctionTok{sum}\NormalTok{(dat.plot}\SpecialCharTok{$}\NormalTok{est)}
\NormalTok{dat.plot}\SpecialCharTok{$}\NormalTok{scenario }\OtherTok{=} \StringTok{"ph 5 only{-}JBC predictions"}
\NormalTok{dat.gg }\OtherTok{=} \FunctionTok{rbind}\NormalTok{(dat.gg, dat.plot)}
\end{Highlighting}
\end{Shaded}

\hypertarget{jbc-jb370-ph-5}{%
\subsection{JBC + JB370 @ pH 5}\label{jbc-jb370-ph-5}}

\begin{Shaded}
\begin{Highlighting}[]
\NormalTok{dat.com}\OtherTok{=}\FunctionTok{read.csv}\NormalTok{(}\FunctionTok{here}\NormalTok{(}\StringTok{"2\_data\_wrangling"}\NormalTok{,}\StringTok{"matrix{-}form{-}comdat.csv"}\NormalTok{))}
\NormalTok{dat.com}\OtherTok{=}\FunctionTok{na.omit}\NormalTok{(dat.com)}
\CommentTok{\# read in our fitted coeffients}
\NormalTok{dat.coefs }\OtherTok{=} \FunctionTok{read.csv}\NormalTok{(}\FunctionTok{here}\NormalTok{(}\StringTok{"4\_res"}\NormalTok{,}
                          \StringTok{"coefficient{-}estimates{-}pH5.csv"}\NormalTok{))}
\DocumentationTok{\#\# turn anything that\textquotesingle{}s not a K or from JBC to 0}
\NormalTok{dat.coefs[}\SpecialCharTok{!}\NormalTok{dat.coefs}\SpecialCharTok{$}\NormalTok{name }\SpecialCharTok{\%in\%} \FunctionTok{c}\NormalTok{(}\StringTok{"K"}\NormalTok{,}\StringTok{"JBC"}\NormalTok{, }\StringTok{"JB370"}\NormalTok{), }\StringTok{"est"}\NormalTok{]}\OtherTok{=}\DecValTok{0}
\NormalTok{dat.pred }\OtherTok{=}\NormalTok{ dat.com}
\NormalTok{dat.pred[,}\SpecialCharTok{{-}}\FunctionTok{which}\NormalTok{(}\FunctionTok{names}\NormalTok{(dat.com)}\SpecialCharTok{==}\StringTok{"rep"}\NormalTok{)]}\OtherTok{=}\DecValTok{0}
\ControlFlowTok{for}\NormalTok{(i.spec }\ControlFlowTok{in} \DecValTok{1}\SpecialCharTok{:}\FunctionTok{length}\NormalTok{(spec.vecfit))\{}
\NormalTok{  cur.spec }\OtherTok{=}\NormalTok{ spec.vecfit[i.spec]}
  \FunctionTok{print}\NormalTok{(cur.spec)}
  \CommentTok{\#some cool manipulating to make it easy to access the right coefficients}
  \CommentTok{\# ending in coefs.vec: a vector of coefficients labeled by name}
\NormalTok{  coefs.cur }\OtherTok{=}\NormalTok{ dat.coefs }\SpecialCharTok{\%\textgreater{}\%} 
    \FunctionTok{filter}\NormalTok{(spec }\SpecialCharTok{==}\NormalTok{ cur.spec)}
\NormalTok{  coefs.vec }\OtherTok{=}\NormalTok{ coefs.cur}\SpecialCharTok{$}\NormalTok{est}
  \FunctionTok{names}\NormalTok{(coefs.vec) }\OtherTok{=}\NormalTok{ coefs.cur}\SpecialCharTok{$}\NormalTok{name}
  \ControlFlowTok{if}\NormalTok{(cur.spec }\SpecialCharTok{==}\StringTok{"JBC"}\NormalTok{)\{}
\NormalTok{    coefs.vec[}\StringTok{"X135E"}\NormalTok{]}\OtherTok{=}\DecValTok{0}
\NormalTok{  \}}
\NormalTok{  x.cur }\OtherTok{=}\NormalTok{ dat.com[,}\SpecialCharTok{{-}}\FunctionTok{which}\NormalTok{(}\FunctionTok{colnames}\NormalTok{(dat.com) }\SpecialCharTok{\%in\%} \FunctionTok{c}\NormalTok{(cur.spec, }\StringTok{"rep"}\NormalTok{))]}
  \CommentTok{\#using element{-}wise multiplication to simplify the coding}
  \CommentTok{\#This may look opaque, but you can confirm that it\textquotesingle{}s doing the right thing with }
  \CommentTok{\# a test case:}
  \CommentTok{\# test.mat = matrix(1:4, 2,2)}
  \CommentTok{\# t(t(test.mat) * c(1,0))}
\NormalTok{  pred.cur }\OtherTok{=}\NormalTok{ coefs.vec[}\StringTok{"K"}\NormalTok{] }\SpecialCharTok{{-}} 
    \FunctionTok{rowSums}\NormalTok{(}\FunctionTok{t}\NormalTok{(}\FunctionTok{t}\NormalTok{(x.cur[,}\FunctionTok{names}\NormalTok{(coefs.vec)[}\SpecialCharTok{{-}}\DecValTok{1}\NormalTok{]]) }\SpecialCharTok{*}\NormalTok{ (coefs.vec[}\SpecialCharTok{{-}}\DecValTok{1}\NormalTok{])))}
\NormalTok{  pred.cur }\OtherTok{=} \FunctionTok{pmax}\NormalTok{(}\DecValTok{0}\NormalTok{,pred.cur)}
\NormalTok{  dat.pred[,cur.spec] }\OtherTok{=}\NormalTok{ pred.cur}
\NormalTok{\}}
\end{Highlighting}
\end{Shaded}

\begin{verbatim}
## [1] "BC9"
## [1] "BC10"
## [1] "JB5"
## [1] "JB7"
## [1] "X135E"
## [1] "JBC"
## [1] "JB370"
\end{verbatim}

\begin{Shaded}
\begin{Highlighting}[]
\NormalTok{dat.pred}\SpecialCharTok{$}\NormalTok{ph }\OtherTok{=} \DecValTok{5}

\FunctionTok{write.csv}\NormalTok{(dat.pred,}
          \FunctionTok{here}\NormalTok{(}\StringTok{"4\_res"}\NormalTok{,}\StringTok{"predictions{-}onlyJBC{-}JB370{-}pH5.csv"}\NormalTok{),}
          \AttributeTok{row.names=}\NormalTok{F}
\NormalTok{)}
\NormalTok{dat.pred}\SpecialCharTok{$}\NormalTok{scenario }\OtherTok{=} \StringTok{"ph 5 only{-}JBC+JB370 predictions"}
\NormalTok{pred.all }\OtherTok{=} \FunctionTok{rbind}\NormalTok{(pred.all, dat.pred)}
\end{Highlighting}
\end{Shaded}

Add to our quick visualization dataframe

\begin{Shaded}
\begin{Highlighting}[]
\NormalTok{dat.plot }\OtherTok{=} \FunctionTok{data.frame}\NormalTok{(}\AttributeTok{est =} \FunctionTok{apply}\NormalTok{(dat.pred,}\DecValTok{2}\NormalTok{, median))}
\end{Highlighting}
\end{Shaded}

\begin{verbatim}
## Warning in mean.default(sort(x, partial = half + 0L:1L)[half + 0L:1L]):
## argument is not numeric or logical: returning NA

## Warning in mean.default(sort(x, partial = half + 0L:1L)[half + 0L:1L]):
## argument is not numeric or logical: returning NA

## Warning in mean.default(sort(x, partial = half + 0L:1L)[half + 0L:1L]):
## argument is not numeric or logical: returning NA

## Warning in mean.default(sort(x, partial = half + 0L:1L)[half + 0L:1L]):
## argument is not numeric or logical: returning NA

## Warning in mean.default(sort(x, partial = half + 0L:1L)[half + 0L:1L]):
## argument is not numeric or logical: returning NA

## Warning in mean.default(sort(x, partial = half + 0L:1L)[half + 0L:1L]):
## argument is not numeric or logical: returning NA

## Warning in mean.default(sort(x, partial = half + 0L:1L)[half + 0L:1L]):
## argument is not numeric or logical: returning NA

## Warning in mean.default(sort(x, partial = half + 0L:1L)[half + 0L:1L]):
## argument is not numeric or logical: returning NA

## Warning in mean.default(sort(x, partial = half + 0L:1L)[half + 0L:1L]):
## argument is not numeric or logical: returning NA

## Warning in mean.default(sort(x, partial = half + 0L:1L)[half + 0L:1L]):
## argument is not numeric or logical: returning NA
\end{verbatim}

\begin{Shaded}
\begin{Highlighting}[]
\NormalTok{dat.plot}\SpecialCharTok{$}\NormalTok{spec }\OtherTok{=} \FunctionTok{colnames}\NormalTok{(dat.pred)}
\NormalTok{dat.plot }\OtherTok{=}\NormalTok{ dat.plot[}\SpecialCharTok{{-}}\FunctionTok{which}\NormalTok{(dat.plot}\SpecialCharTok{$}\NormalTok{spec }\SpecialCharTok{\%in\%} \FunctionTok{c}\NormalTok{(}\StringTok{"rep"}\NormalTok{,}\StringTok{"ph"}\NormalTok{)),]}
\NormalTok{dat.plot}\SpecialCharTok{$}\NormalTok{est }\OtherTok{=} \FunctionTok{pmax}\NormalTok{(dat.plot}\SpecialCharTok{$}\NormalTok{est, }\DecValTok{0}\NormalTok{)}
\NormalTok{dat.plot}\SpecialCharTok{$}\NormalTok{rel }\OtherTok{=}\NormalTok{ dat.plot}\SpecialCharTok{$}\NormalTok{est}\SpecialCharTok{/}\FunctionTok{sum}\NormalTok{(dat.plot}\SpecialCharTok{$}\NormalTok{est)}
\NormalTok{dat.plot}\SpecialCharTok{$}\NormalTok{scenario }\OtherTok{=} \StringTok{"ph 5 only{-}JBC+JB370 predictions"}
\NormalTok{dat.gg }\OtherTok{=} \FunctionTok{rbind}\NormalTok{(dat.gg, dat.plot)}
\end{Highlighting}
\end{Shaded}

\hypertarget{jbc-bc10-ph-5}{%
\subsection{JBC + BC10 @ pH 5}\label{jbc-bc10-ph-5}}

\begin{Shaded}
\begin{Highlighting}[]
\NormalTok{dat.com}\OtherTok{=}\FunctionTok{read.csv}\NormalTok{(}\FunctionTok{here}\NormalTok{(}\StringTok{"2\_data\_wrangling"}\NormalTok{,}\StringTok{"matrix{-}form{-}comdat.csv"}\NormalTok{))}
\NormalTok{dat.com}\OtherTok{=}\FunctionTok{na.omit}\NormalTok{(dat.com)}
\CommentTok{\# read in our fitted coeffients}
\NormalTok{dat.coefs }\OtherTok{=} \FunctionTok{read.csv}\NormalTok{(}\FunctionTok{here}\NormalTok{(}\StringTok{"4\_res"}\NormalTok{,}
                          \StringTok{"coefficient{-}estimates{-}pH5.csv"}\NormalTok{))}
\DocumentationTok{\#\# turn anything that\textquotesingle{}s not a K or from JBC to 0}
\NormalTok{dat.coefs[}\SpecialCharTok{!}\NormalTok{dat.coefs}\SpecialCharTok{$}\NormalTok{name }\SpecialCharTok{\%in\%} \FunctionTok{c}\NormalTok{(}\StringTok{"K"}\NormalTok{,}\StringTok{"JBC"}\NormalTok{, }\StringTok{"BC10"}\NormalTok{), }\StringTok{"est"}\NormalTok{]}\OtherTok{=}\DecValTok{0}
\NormalTok{dat.pred }\OtherTok{=}\NormalTok{ dat.com}
\NormalTok{dat.pred[,}\SpecialCharTok{{-}}\FunctionTok{which}\NormalTok{(}\FunctionTok{names}\NormalTok{(dat.com)}\SpecialCharTok{==}\StringTok{"rep"}\NormalTok{)]}\OtherTok{=}\DecValTok{0}
\ControlFlowTok{for}\NormalTok{(i.spec }\ControlFlowTok{in} \DecValTok{1}\SpecialCharTok{:}\FunctionTok{length}\NormalTok{(spec.vecfit))\{}
\NormalTok{  cur.spec }\OtherTok{=}\NormalTok{ spec.vecfit[i.spec]}
  \FunctionTok{print}\NormalTok{(cur.spec)}
  \CommentTok{\#some cool manipulating to make it easy to access the right coefficients}
  \CommentTok{\# ending in coefs.vec: a vector of coefficients labeled by name}
\NormalTok{  coefs.cur }\OtherTok{=}\NormalTok{ dat.coefs }\SpecialCharTok{\%\textgreater{}\%} 
    \FunctionTok{filter}\NormalTok{(spec }\SpecialCharTok{==}\NormalTok{ cur.spec)}
\NormalTok{  coefs.vec }\OtherTok{=}\NormalTok{ coefs.cur}\SpecialCharTok{$}\NormalTok{est}
  \FunctionTok{names}\NormalTok{(coefs.vec) }\OtherTok{=}\NormalTok{ coefs.cur}\SpecialCharTok{$}\NormalTok{name}
  \ControlFlowTok{if}\NormalTok{(cur.spec }\SpecialCharTok{==}\StringTok{"JBC"}\NormalTok{)\{}
\NormalTok{    coefs.vec[}\StringTok{"X135E"}\NormalTok{]}\OtherTok{=}\DecValTok{0}
\NormalTok{  \}}
\NormalTok{  x.cur }\OtherTok{=}\NormalTok{ dat.com[,}\SpecialCharTok{{-}}\FunctionTok{which}\NormalTok{(}\FunctionTok{colnames}\NormalTok{(dat.com) }\SpecialCharTok{\%in\%} \FunctionTok{c}\NormalTok{(cur.spec, }\StringTok{"rep"}\NormalTok{))]}
  \CommentTok{\#using element{-}wise multiplication to simplify the coding}
  \CommentTok{\#This may look opaque, but you can confirm that it\textquotesingle{}s doing the right thing with }
  \CommentTok{\# a test case:}
  \CommentTok{\# test.mat = matrix(1:4, 2,2)}
  \CommentTok{\# t(t(test.mat) * c(1,0))}
\NormalTok{  pred.cur }\OtherTok{=}\NormalTok{ coefs.vec[}\StringTok{"K"}\NormalTok{] }\SpecialCharTok{{-}} 
    \FunctionTok{rowSums}\NormalTok{(}\FunctionTok{t}\NormalTok{(}\FunctionTok{t}\NormalTok{(x.cur[,}\FunctionTok{names}\NormalTok{(coefs.vec)[}\SpecialCharTok{{-}}\DecValTok{1}\NormalTok{]]) }\SpecialCharTok{*}\NormalTok{ (coefs.vec[}\SpecialCharTok{{-}}\DecValTok{1}\NormalTok{])))}
\NormalTok{  pred.cur }\OtherTok{=} \FunctionTok{pmax}\NormalTok{(}\DecValTok{0}\NormalTok{,pred.cur)}
\NormalTok{  dat.pred[,cur.spec] }\OtherTok{=}\NormalTok{ pred.cur}
\NormalTok{\}}
\end{Highlighting}
\end{Shaded}

\begin{verbatim}
## [1] "BC9"
## [1] "BC10"
## [1] "JB5"
## [1] "JB7"
## [1] "X135E"
## [1] "JBC"
## [1] "JB370"
\end{verbatim}

\begin{Shaded}
\begin{Highlighting}[]
\NormalTok{dat.pred}\SpecialCharTok{$}\NormalTok{ph }\OtherTok{=} \DecValTok{5}

\FunctionTok{write.csv}\NormalTok{(dat.pred,}
          \FunctionTok{here}\NormalTok{(}\StringTok{"4\_res"}\NormalTok{,}\StringTok{"predictions{-}onlyJBC{-}BC10{-}pH5.csv"}\NormalTok{),}
          \AttributeTok{row.names=}\NormalTok{F}
\NormalTok{)}
\NormalTok{dat.pred}\SpecialCharTok{$}\NormalTok{scenario }\OtherTok{=} \StringTok{"ph 5 only{-}JBC+BC10 predictions"}
\NormalTok{pred.all }\OtherTok{=} \FunctionTok{rbind}\NormalTok{(pred.all, dat.pred)}
\end{Highlighting}
\end{Shaded}

Add to our quick visualization dataframe

\begin{Shaded}
\begin{Highlighting}[]
\NormalTok{dat.plot }\OtherTok{=} \FunctionTok{data.frame}\NormalTok{(}\AttributeTok{est =} \FunctionTok{apply}\NormalTok{(dat.pred,}\DecValTok{2}\NormalTok{, median))}
\end{Highlighting}
\end{Shaded}

\begin{verbatim}
## Warning in mean.default(sort(x, partial = half + 0L:1L)[half + 0L:1L]):
## argument is not numeric or logical: returning NA

## Warning in mean.default(sort(x, partial = half + 0L:1L)[half + 0L:1L]):
## argument is not numeric or logical: returning NA

## Warning in mean.default(sort(x, partial = half + 0L:1L)[half + 0L:1L]):
## argument is not numeric or logical: returning NA

## Warning in mean.default(sort(x, partial = half + 0L:1L)[half + 0L:1L]):
## argument is not numeric or logical: returning NA

## Warning in mean.default(sort(x, partial = half + 0L:1L)[half + 0L:1L]):
## argument is not numeric or logical: returning NA

## Warning in mean.default(sort(x, partial = half + 0L:1L)[half + 0L:1L]):
## argument is not numeric or logical: returning NA

## Warning in mean.default(sort(x, partial = half + 0L:1L)[half + 0L:1L]):
## argument is not numeric or logical: returning NA

## Warning in mean.default(sort(x, partial = half + 0L:1L)[half + 0L:1L]):
## argument is not numeric or logical: returning NA

## Warning in mean.default(sort(x, partial = half + 0L:1L)[half + 0L:1L]):
## argument is not numeric or logical: returning NA

## Warning in mean.default(sort(x, partial = half + 0L:1L)[half + 0L:1L]):
## argument is not numeric or logical: returning NA
\end{verbatim}

\begin{Shaded}
\begin{Highlighting}[]
\NormalTok{dat.plot}\SpecialCharTok{$}\NormalTok{spec }\OtherTok{=} \FunctionTok{colnames}\NormalTok{(dat.pred)}
\NormalTok{dat.plot }\OtherTok{=}\NormalTok{ dat.plot[}\SpecialCharTok{{-}}\FunctionTok{which}\NormalTok{(dat.plot}\SpecialCharTok{$}\NormalTok{spec }\SpecialCharTok{\%in\%} \FunctionTok{c}\NormalTok{(}\StringTok{"rep"}\NormalTok{,}\StringTok{"ph"}\NormalTok{)),]}
\NormalTok{dat.plot}\SpecialCharTok{$}\NormalTok{est }\OtherTok{=} \FunctionTok{pmax}\NormalTok{(dat.plot}\SpecialCharTok{$}\NormalTok{est, }\DecValTok{0}\NormalTok{)}
\NormalTok{dat.plot}\SpecialCharTok{$}\NormalTok{rel }\OtherTok{=}\NormalTok{ dat.plot}\SpecialCharTok{$}\NormalTok{est}\SpecialCharTok{/}\FunctionTok{sum}\NormalTok{(dat.plot}\SpecialCharTok{$}\NormalTok{est)}
\NormalTok{dat.plot}\SpecialCharTok{$}\NormalTok{scenario }\OtherTok{=} \StringTok{"ph 5 only{-}JBC+BC10 predictions"}
\NormalTok{dat.gg }\OtherTok{=} \FunctionTok{rbind}\NormalTok{(dat.gg, dat.plot)}
\end{Highlighting}
\end{Shaded}

\hypertarget{updated-viz}{%
\subsection{Updated Viz}\label{updated-viz}}

\begin{Shaded}
\begin{Highlighting}[]
\FunctionTok{ggplot}\NormalTok{(}\AttributeTok{data =}\NormalTok{ dat.gg, }\FunctionTok{aes}\NormalTok{(}\AttributeTok{fill =}\NormalTok{ spec, }\AttributeTok{x =}\NormalTok{ scenario, }\AttributeTok{y =}\NormalTok{ rel))}\SpecialCharTok{+}
  \FunctionTok{geom\_col}\NormalTok{()}\SpecialCharTok{+}
  \FunctionTok{ylab}\NormalTok{(}\StringTok{"relative abundance"}\NormalTok{)}\SpecialCharTok{+}
\NormalTok{  theme.mine }\SpecialCharTok{+}
  \FunctionTok{coord\_flip}\NormalTok{()}
\end{Highlighting}
\end{Shaded}

\begin{verbatim}
## Warning: Removed 56 rows containing missing values (`position_stack()`).
\end{verbatim}

\includegraphics{succession-analysis-v1.2_files/figure-latex/unnamed-chunk-31-1.pdf}

\hypertarget{updated-mse}{%
\subsection{Updated MSE}\label{updated-mse}}

Note: doing 2 fiddly bits here. Estimating for each replicate
separately, and also dividing by 1000, so that we're looking at MSE of
thousands of CFUS. This is identical in theory (lowest MSE is still
best), but keeps numbers smaller, improving the practicalities.

\begin{Shaded}
\begin{Highlighting}[]
\NormalTok{scaling }\OtherTok{=} \DecValTok{1000}
\NormalTok{mse.df}\OtherTok{=}\ConstantTok{NULL}

\NormalTok{dat.real }\OtherTok{=}\NormalTok{ dat.com }\SpecialCharTok{\%\textgreater{}\%}  \FunctionTok{select}\NormalTok{(}\SpecialCharTok{{-}}\NormalTok{rep)}\SpecialCharTok{/}\NormalTok{scaling}

\FunctionTok{print}\NormalTok{(}\StringTok{"ph 5 all MSE"}\NormalTok{)}
\end{Highlighting}
\end{Shaded}

\begin{verbatim}
## [1] "ph 5 all MSE"
\end{verbatim}

\begin{Shaded}
\begin{Highlighting}[]
\NormalTok{pred.cur }\OtherTok{=}\NormalTok{ pred.all }\SpecialCharTok{\%\textgreater{}\%} 
  \FunctionTok{filter}\NormalTok{(scenario }\SpecialCharTok{==} \StringTok{"ph 5 predictions"}\NormalTok{) }\SpecialCharTok{\%\textgreater{}\%} 
  \FunctionTok{select}\NormalTok{(}\SpecialCharTok{{-}}\NormalTok{rep, }\SpecialCharTok{{-}}\NormalTok{ph, }\SpecialCharTok{{-}}\NormalTok{scenario)}
\NormalTok{pred.cur }\OtherTok{=}\NormalTok{ pred.cur}\SpecialCharTok{/}\NormalTok{scaling}
\NormalTok{sq.er }\OtherTok{=}\NormalTok{(dat.real }\SpecialCharTok{{-}}\NormalTok{ pred.cur[,}\FunctionTok{names}\NormalTok{(dat.real)])}\SpecialCharTok{\^{}}\DecValTok{2}
\NormalTok{(}\AttributeTok{mse.res =} \FunctionTok{mean}\NormalTok{(}\FunctionTok{as.matrix}\NormalTok{(sq.er)))}
\end{Highlighting}
\end{Shaded}

\begin{verbatim}
## [1] 1.692738e+17
\end{verbatim}

\begin{Shaded}
\begin{Highlighting}[]
\NormalTok{mse.df }\OtherTok{=} \FunctionTok{rbind}\NormalTok{(mse.df, }\FunctionTok{data.frame}\NormalTok{(}\AttributeTok{scenario =} \StringTok{"ph 5 predictions"}\NormalTok{,}
                                  \AttributeTok{mse =}\NormalTok{ mse.res))}

\FunctionTok{print}\NormalTok{(}\StringTok{"ph 7 all MSE"}\NormalTok{)}
\end{Highlighting}
\end{Shaded}

\begin{verbatim}
## [1] "ph 7 all MSE"
\end{verbatim}

\begin{Shaded}
\begin{Highlighting}[]
\NormalTok{pred.cur }\OtherTok{=}\NormalTok{ pred.all }\SpecialCharTok{\%\textgreater{}\%} 
  \FunctionTok{filter}\NormalTok{(scenario }\SpecialCharTok{==} \StringTok{"ph 7 predictions"}\NormalTok{)  }\SpecialCharTok{\%\textgreater{}\%} 
  \FunctionTok{select}\NormalTok{(}\SpecialCharTok{{-}}\NormalTok{rep, }\SpecialCharTok{{-}}\NormalTok{ph, }\SpecialCharTok{{-}}\NormalTok{scenario)}
\NormalTok{pred.cur }\OtherTok{=}\NormalTok{ pred.cur}\SpecialCharTok{/}\NormalTok{scaling}
\NormalTok{sq.er }\OtherTok{=}\NormalTok{(dat.real }\SpecialCharTok{{-}}\NormalTok{ pred.cur[,}\FunctionTok{names}\NormalTok{(dat.real)])}\SpecialCharTok{\^{}}\DecValTok{2}
\NormalTok{(}\AttributeTok{mse.res =} \FunctionTok{mean}\NormalTok{(}\FunctionTok{as.matrix}\NormalTok{(sq.er)))}
\end{Highlighting}
\end{Shaded}

\begin{verbatim}
## [1] 384455411322
\end{verbatim}

\begin{Shaded}
\begin{Highlighting}[]
\NormalTok{mse.df }\OtherTok{=} \FunctionTok{rbind}\NormalTok{(mse.df, }\FunctionTok{data.frame}\NormalTok{(}\AttributeTok{scenario =} \StringTok{"ph 7 predictions"}\NormalTok{,}
                                  \AttributeTok{mse =}\NormalTok{ mse.res))}


\FunctionTok{print}\NormalTok{(}\StringTok{"ph 7 JBC"}\NormalTok{)}
\end{Highlighting}
\end{Shaded}

\begin{verbatim}
## [1] "ph 7 JBC"
\end{verbatim}

\begin{Shaded}
\begin{Highlighting}[]
\NormalTok{pred.cur }\OtherTok{=}\NormalTok{ pred.all }\SpecialCharTok{\%\textgreater{}\%} 
  \FunctionTok{filter}\NormalTok{(scenario }\SpecialCharTok{==} \StringTok{"ph 7 only{-}JBC predictions"}\NormalTok{) }\SpecialCharTok{\%\textgreater{}\%} 
  \FunctionTok{select}\NormalTok{(}\SpecialCharTok{{-}}\NormalTok{rep, }\SpecialCharTok{{-}}\NormalTok{ph, }\SpecialCharTok{{-}}\NormalTok{scenario)}
\NormalTok{pred.cur }\OtherTok{=}\NormalTok{ pred.cur}\SpecialCharTok{/}\NormalTok{scaling}
\NormalTok{sq.er }\OtherTok{=}\NormalTok{(dat.real }\SpecialCharTok{{-}}\NormalTok{ pred.cur[,}\FunctionTok{names}\NormalTok{(dat.real)])}\SpecialCharTok{\^{}}\DecValTok{2}
\NormalTok{(}\AttributeTok{mse.res =} \FunctionTok{mean}\NormalTok{(}\FunctionTok{as.matrix}\NormalTok{(sq.er)))}
\end{Highlighting}
\end{Shaded}

\begin{verbatim}
## [1] 372834889020
\end{verbatim}

\begin{Shaded}
\begin{Highlighting}[]
\NormalTok{mse.df }\OtherTok{=} \FunctionTok{rbind}\NormalTok{(mse.df, }\FunctionTok{data.frame}\NormalTok{(}\AttributeTok{scenario =} \StringTok{"ph 7 only{-}JBC predictions"}\NormalTok{,}
                                  \AttributeTok{mse =}\NormalTok{ mse.res))}

\FunctionTok{print}\NormalTok{(}\StringTok{"ph 5 JBC + 135E effectors"}\NormalTok{)}
\end{Highlighting}
\end{Shaded}

\begin{verbatim}
## [1] "ph 5 JBC + 135E effectors"
\end{verbatim}

\begin{Shaded}
\begin{Highlighting}[]
\NormalTok{pred.cur }\OtherTok{=}\NormalTok{ pred.all }\SpecialCharTok{\%\textgreater{}\%} 
  \FunctionTok{filter}\NormalTok{(scenario }\SpecialCharTok{==} \StringTok{"ph 5 only{-}JBC+135E predictions"}\NormalTok{) }\SpecialCharTok{\%\textgreater{}\%} 
  \FunctionTok{select}\NormalTok{(}\SpecialCharTok{{-}}\NormalTok{rep, }\SpecialCharTok{{-}}\NormalTok{ph, }\SpecialCharTok{{-}}\NormalTok{scenario)}
\NormalTok{pred.cur }\OtherTok{=}\NormalTok{ pred.cur}\SpecialCharTok{/}\NormalTok{scaling}
\NormalTok{sq.er }\OtherTok{=}\NormalTok{(dat.real }\SpecialCharTok{{-}}\NormalTok{ pred.cur[,}\FunctionTok{names}\NormalTok{(dat.real)])}\SpecialCharTok{\^{}}\DecValTok{2}
\NormalTok{(}\AttributeTok{mse.res =} \FunctionTok{mean}\NormalTok{(}\FunctionTok{as.matrix}\NormalTok{(sq.er)))}
\end{Highlighting}
\end{Shaded}

\begin{verbatim}
## [1] 205281133
\end{verbatim}

\begin{Shaded}
\begin{Highlighting}[]
\NormalTok{mse.df }\OtherTok{=} \FunctionTok{rbind}\NormalTok{(mse.df, }\FunctionTok{data.frame}\NormalTok{(}\AttributeTok{scenario =} \StringTok{"ph 5 only{-}JBC+135E predictions"}\NormalTok{,}
                                  \AttributeTok{mse =}\NormalTok{ mse.res))}

\FunctionTok{print}\NormalTok{(}\StringTok{"ph 5 JBC"}\NormalTok{)}
\end{Highlighting}
\end{Shaded}

\begin{verbatim}
## [1] "ph 5 JBC"
\end{verbatim}

\begin{Shaded}
\begin{Highlighting}[]
\NormalTok{pred.cur }\OtherTok{=}\NormalTok{ pred.all }\SpecialCharTok{\%\textgreater{}\%} 
  \FunctionTok{filter}\NormalTok{(scenario }\SpecialCharTok{==} \StringTok{"ph 5 only{-}JBC predictions"}\NormalTok{) }\SpecialCharTok{\%\textgreater{}\%} 
  \FunctionTok{select}\NormalTok{(}\SpecialCharTok{{-}}\NormalTok{rep, }\SpecialCharTok{{-}}\NormalTok{ph, }\SpecialCharTok{{-}}\NormalTok{scenario)}
\NormalTok{pred.cur }\OtherTok{=}\NormalTok{ pred.cur}\SpecialCharTok{/}\NormalTok{scaling}
\NormalTok{sq.er }\OtherTok{=}\NormalTok{(dat.real }\SpecialCharTok{{-}}\NormalTok{ pred.cur[,}\FunctionTok{names}\NormalTok{(dat.real)])}\SpecialCharTok{\^{}}\DecValTok{2}
\NormalTok{(}\AttributeTok{mse.res =} \FunctionTok{mean}\NormalTok{(}\FunctionTok{as.matrix}\NormalTok{(sq.er)))}
\end{Highlighting}
\end{Shaded}

\begin{verbatim}
## [1] 205281133
\end{verbatim}

\begin{Shaded}
\begin{Highlighting}[]
\NormalTok{mse.df }\OtherTok{=} \FunctionTok{rbind}\NormalTok{(mse.df, }\FunctionTok{data.frame}\NormalTok{(}\AttributeTok{scenario =} \StringTok{"ph 5 only{-}JBC predictions"}\NormalTok{,}
                                  \AttributeTok{mse =}\NormalTok{ mse.res))}

\FunctionTok{print}\NormalTok{(}\StringTok{"ph 5 JBC+JB370"}\NormalTok{)}
\end{Highlighting}
\end{Shaded}

\begin{verbatim}
## [1] "ph 5 JBC+JB370"
\end{verbatim}

\begin{Shaded}
\begin{Highlighting}[]
\NormalTok{pred.cur }\OtherTok{=}\NormalTok{ pred.all }\SpecialCharTok{\%\textgreater{}\%} 
  \FunctionTok{filter}\NormalTok{(scenario }\SpecialCharTok{==} \StringTok{"ph 5 only{-}JBC+JB370 predictions"}\NormalTok{) }\SpecialCharTok{\%\textgreater{}\%} 
  \FunctionTok{select}\NormalTok{(}\SpecialCharTok{{-}}\NormalTok{rep, }\SpecialCharTok{{-}}\NormalTok{ph, }\SpecialCharTok{{-}}\NormalTok{scenario)}
\NormalTok{pred.cur }\OtherTok{=}\NormalTok{ pred.cur}\SpecialCharTok{/}\NormalTok{scaling}
\NormalTok{sq.er }\OtherTok{=}\NormalTok{(dat.real }\SpecialCharTok{{-}}\NormalTok{ pred.cur[,}\FunctionTok{names}\NormalTok{(dat.real)])}\SpecialCharTok{\^{}}\DecValTok{2}
\NormalTok{(}\AttributeTok{mse.res =} \FunctionTok{mean}\NormalTok{(}\FunctionTok{as.matrix}\NormalTok{(sq.er)))}
\end{Highlighting}
\end{Shaded}

\begin{verbatim}
## [1] 205486373
\end{verbatim}

\begin{Shaded}
\begin{Highlighting}[]
\NormalTok{mse.df }\OtherTok{=} \FunctionTok{rbind}\NormalTok{(mse.df, }\FunctionTok{data.frame}\NormalTok{(}\AttributeTok{scenario =} \StringTok{"ph 5 only{-}JBC+JB370 predictions"}\NormalTok{,}
                                  \AttributeTok{mse =}\NormalTok{ mse.res))}

\FunctionTok{print}\NormalTok{(}\StringTok{"ph 5 JBC+BC10"}\NormalTok{)}
\end{Highlighting}
\end{Shaded}

\begin{verbatim}
## [1] "ph 5 JBC+BC10"
\end{verbatim}

\begin{Shaded}
\begin{Highlighting}[]
\NormalTok{pred.cur }\OtherTok{=}\NormalTok{ pred.all }\SpecialCharTok{\%\textgreater{}\%} 
  \FunctionTok{filter}\NormalTok{(scenario }\SpecialCharTok{==} \StringTok{"ph 5 only{-}JBC+BC10 predictions"}\NormalTok{) }\SpecialCharTok{\%\textgreater{}\%} 
  \FunctionTok{select}\NormalTok{(}\SpecialCharTok{{-}}\NormalTok{rep, }\SpecialCharTok{{-}}\NormalTok{ph, }\SpecialCharTok{{-}}\NormalTok{scenario)}
\NormalTok{pred.cur }\OtherTok{=}\NormalTok{ pred.cur}\SpecialCharTok{/}\NormalTok{scaling}
\NormalTok{sq.er }\OtherTok{=}\NormalTok{(dat.real }\SpecialCharTok{{-}}\NormalTok{ pred.cur[,}\FunctionTok{names}\NormalTok{(dat.real)])}\SpecialCharTok{\^{}}\DecValTok{2}
\NormalTok{(}\AttributeTok{mse.res =} \FunctionTok{mean}\NormalTok{(}\FunctionTok{as.matrix}\NormalTok{(sq.er)))}
\end{Highlighting}
\end{Shaded}

\begin{verbatim}
## [1] 19681024217
\end{verbatim}

\begin{Shaded}
\begin{Highlighting}[]
\NormalTok{mse.df }\OtherTok{=} \FunctionTok{rbind}\NormalTok{(mse.df, }\FunctionTok{data.frame}\NormalTok{(}\AttributeTok{scenario =} \StringTok{"ph 5 only{-}JBC+BC10 predictions"}\NormalTok{,}
                                  \AttributeTok{mse =}\NormalTok{ mse.res))}
\end{Highlighting}
\end{Shaded}

\hypertarget{mse-viz}{%
\subsection{MSE Viz}\label{mse-viz}}

\begin{Shaded}
\begin{Highlighting}[]
\FunctionTok{ggplot}\NormalTok{(mse.df, }\FunctionTok{aes}\NormalTok{(}\AttributeTok{x =} \FunctionTok{fct\_rev}\NormalTok{(}\FunctionTok{fct\_reorder}\NormalTok{(scenario, mse)), }\AttributeTok{y =}\NormalTok{ mse))}\SpecialCharTok{+}
  \FunctionTok{geom\_col}\NormalTok{()}\SpecialCharTok{+}
  \FunctionTok{scale\_y\_log10}\NormalTok{()}\SpecialCharTok{+}
  \FunctionTok{xlab}\NormalTok{(}\StringTok{"Scenario"}\NormalTok{)}\SpecialCharTok{+}
  \FunctionTok{ylab}\NormalTok{(}\StringTok{"Mean Squared Error: lowest = most realistic model"}\NormalTok{)}\SpecialCharTok{+}
  \FunctionTok{coord\_flip}\NormalTok{()}
\end{Highlighting}
\end{Shaded}

\includegraphics{succession-analysis-v1.2_files/figure-latex/unnamed-chunk-33-1.pdf}

\hypertarget{quick-visual-check-of-ph-7-jbc-only}{%
\section{Quick visual check of pH 7
JBC-only}\label{quick-visual-check-of-ph-7-jbc-only}}

We notice that the pH 7 predictions are almost the same if we only
include JBC. This would make sense only if the overwhelming majority of
interactions come from JBC (Penicillium). Remember that the total effect
of an interaction is the interaction coefficient times the population
size of the source of interaction. Here we plot this term (as an
absolute value, on a log scale).

\begin{Shaded}
\begin{Highlighting}[]
\NormalTok{dat.com}\OtherTok{=}\FunctionTok{read.csv}\NormalTok{(}\FunctionTok{here}\NormalTok{(}\StringTok{"2\_data\_wrangling"}\NormalTok{,}\StringTok{"matrix{-}form{-}comdat.csv"}\NormalTok{))}
\NormalTok{dat.com}\OtherTok{=}\FunctionTok{na.omit}\NormalTok{(dat.com)}
\CommentTok{\# read in our fitted coeffients}
\NormalTok{dat.coefs }\OtherTok{=} \FunctionTok{read.csv}\NormalTok{(}\FunctionTok{here}\NormalTok{(}\StringTok{"4\_res"}\NormalTok{,}
                          \StringTok{"coefficient{-}estimates{-}pH7.csv"}\NormalTok{))}

\NormalTok{dat.commed }\OtherTok{=} \FunctionTok{apply}\NormalTok{(dat.com, }\DecValTok{2}\NormalTok{, median)}
\NormalTok{dat.temp }\OtherTok{=}\NormalTok{ dat.coefs }\SpecialCharTok{\%\textgreater{}\%} 
  \FunctionTok{filter}\NormalTok{(name }\SpecialCharTok{!=} \StringTok{"K"}\NormalTok{)}
\NormalTok{dat.temp}\SpecialCharTok{$}\NormalTok{eff }\OtherTok{=}\NormalTok{ dat.temp}\SpecialCharTok{$}\NormalTok{est }\SpecialCharTok{*}\NormalTok{ dat.commed[dat.temp}\SpecialCharTok{$}\NormalTok{name]}
\NormalTok{dat.temp}\SpecialCharTok{$}\NormalTok{effab}\OtherTok{=}\FunctionTok{abs}\NormalTok{(dat.temp}\SpecialCharTok{$}\NormalTok{eff)}
\CommentTok{\# }
\CommentTok{\# ggplot(dat.temp, aes(x = name, y = est))+}
\CommentTok{\#   geom\_point()+}
\CommentTok{\#   scale\_y\_log10()}
\CommentTok{\# ggplot(dat.temp, aes(x = name, y = eff, col=spec))+}
\CommentTok{\#   geom\_line(aes(group=spec))+}
\CommentTok{\#   geom\_point()}

\FunctionTok{ggplot}\NormalTok{(dat.temp, }\FunctionTok{aes}\NormalTok{(}\AttributeTok{x =}\NormalTok{ name, }\AttributeTok{y =}\NormalTok{ effab, }\AttributeTok{col=}\NormalTok{spec))}\SpecialCharTok{+}
  \FunctionTok{geom\_line}\NormalTok{(}\FunctionTok{aes}\NormalTok{(}\AttributeTok{group=}\NormalTok{spec))}\SpecialCharTok{+}
  \FunctionTok{geom\_point}\NormalTok{()}\SpecialCharTok{+}
  \FunctionTok{scale\_y\_log10}\NormalTok{()}\SpecialCharTok{+}\NormalTok{theme.mine}
\end{Highlighting}
\end{Shaded}

\begin{verbatim}
## Warning: Transformation introduced infinite values in continuous y-axis
## Transformation introduced infinite values in continuous y-axis
\end{verbatim}

\begin{verbatim}
## Warning: Removed 1 row containing missing values (`geom_line()`).
\end{verbatim}

\begin{verbatim}
## Warning: Removed 1 rows containing missing values (`geom_point()`).
\end{verbatim}

\includegraphics{succession-analysis-v1.2_files/figure-latex/unnamed-chunk-34-1.pdf}

We see by far the strongest estimated interactions in the pH 7 data are
expected to come from JBC, so cutting out all other interactions should
not have any effect.

\hypertarget{which-interactions-change}{%
\section{Which interactions change}\label{which-interactions-change}}

This is attempt 2, which is excluding the interaction terms for species
which are at 0 density in pH 5

Note: In addition to identifying the significant changes, we identify
(a) the original sign of the ALPHA (negative of coefficient estimate)
(b) the direction of change of THE ALPHA, and the change in the
magnitude of the alpha.

\begin{Shaded}
\begin{Highlighting}[]
\NormalTok{dat}\FloatTok{.7} \OtherTok{=} \FunctionTok{read.csv}\NormalTok{(}\FunctionTok{here}\NormalTok{(}\StringTok{"2\_data\_wrangling"}\NormalTok{,}
                      \StringTok{"matrix{-}form{-}pH7.csv"}\NormalTok{))}
\NormalTok{dat}\FloatTok{.5} \OtherTok{=} \FunctionTok{read.csv}\NormalTok{(}\FunctionTok{here}\NormalTok{(}\StringTok{"2\_data\_wrangling"}\NormalTok{,}
                      \StringTok{"matrix{-}form.csv"}\NormalTok{))}
\NormalTok{dat.full }\OtherTok{=} \FunctionTok{rbind}\NormalTok{(dat}\FloatTok{.7}\NormalTok{, dat}\FloatTok{.5}\NormalTok{)}
\NormalTok{dat.full}\SpecialCharTok{$}\NormalTok{pH}\OtherTok{=}\FunctionTok{as.character}\NormalTok{(dat.full}\SpecialCharTok{$}\NormalTok{pH)}
\end{Highlighting}
\end{Shaded}

\begin{Shaded}
\begin{Highlighting}[]
\NormalTok{findings }\OtherTok{=} \FunctionTok{list}\NormalTok{()}
\NormalTok{sigfind }\OtherTok{=} \ConstantTok{NULL}
\end{Highlighting}
\end{Shaded}

\hypertarget{bc9}{%
\subsection{BC9}\label{bc9}}

\begin{Shaded}
\begin{Highlighting}[]
\NormalTok{dat.cur }\OtherTok{=}\NormalTok{ dat.full[dat.full}\SpecialCharTok{$}\NormalTok{pres.BC9}\SpecialCharTok{==}\DecValTok{1}\NormalTok{,]}
\NormalTok{out }\OtherTok{=} \FunctionTok{lm}\NormalTok{(BC9 }\SpecialCharTok{\textasciitilde{}}\NormalTok{ BC10}\SpecialCharTok{*}\NormalTok{pH }\SpecialCharTok{+}\NormalTok{ JB5}\SpecialCharTok{*}\NormalTok{pH }\SpecialCharTok{+}\NormalTok{ JB7 }\SpecialCharTok{+}\NormalTok{ X135E}\SpecialCharTok{*}\NormalTok{pH }\SpecialCharTok{+}\NormalTok{ JBC}\SpecialCharTok{*}\NormalTok{pH }\SpecialCharTok{+}\NormalTok{ JB370}\SpecialCharTok{*}\NormalTok{pH,}
         \AttributeTok{data =}\NormalTok{ dat.cur)}
\FunctionTok{summary}\NormalTok{(out)}
\end{Highlighting}
\end{Shaded}

\begin{verbatim}
## 
## Call:
## lm(formula = BC9 ~ BC10 * pH + JB5 * pH + JB7 + X135E * pH + 
##     JBC * pH + JB370 * pH, data = dat.cur)
## 
## Residuals:
##        Min         1Q     Median         3Q        Max 
## -304110573  -28923023  -10062481   46280718  315530032 
## 
## Coefficients:
##               Estimate Std. Error t value Pr(>|t|)    
## (Intercept)  4.249e+07  3.343e+07   1.271  0.20935    
## BC10        -1.793e-01  4.843e-01  -0.370  0.71262    
## pH7          2.341e+08  4.837e+07   4.840 1.16e-05 ***
## JB5         -1.472e+04  5.443e+04  -0.271  0.78782    
## JB7          2.835e-02  2.351e-02   1.206  0.23327    
## X135E        4.365e+00  1.497e+00   2.916  0.00518 ** 
## JBC         -4.125e-01  9.250e-01  -0.446  0.65749    
## JB370        2.507e+01  4.494e+00   5.579 8.39e-07 ***
## BC10:pH7    -7.617e-01  5.413e-01  -1.407  0.16521    
## pH7:JB5      1.472e+04  5.443e+04   0.271  0.78782    
## pH7:X135E   -3.698e+00  3.143e+00  -1.177  0.24461    
## pH7:JBC     -1.801e+00  1.663e+00  -1.083  0.28378    
## pH7:JB370   -1.358e+02  8.151e+01  -1.667  0.10147    
## ---
## Signif. codes:  0 '***' 0.001 '**' 0.01 '*' 0.05 '.' 0.1 ' ' 1
## 
## Residual standard error: 108100000 on 53 degrees of freedom
## Multiple R-squared:  0.6425, Adjusted R-squared:  0.5616 
## F-statistic: 7.939 on 12 and 53 DF,  p-value: 3.258e-08
\end{verbatim}

\begin{Shaded}
\begin{Highlighting}[]
\NormalTok{temp }\OtherTok{=} \FunctionTok{Anova}\NormalTok{(out)}
\NormalTok{temp }\OtherTok{=}\NormalTok{ temp[}\FunctionTok{grep}\NormalTok{(}\StringTok{\textquotesingle{}:\textquotesingle{}}\NormalTok{,}\FunctionTok{rownames}\NormalTok{(temp)),]}
\NormalTok{temp }\OtherTok{=}\NormalTok{ temp[temp}\SpecialCharTok{$}\StringTok{\textasciigrave{}}\AttributeTok{Pr(\textgreater{}F)}\StringTok{\textasciigrave{}}\SpecialCharTok{\textless{}}\NormalTok{.}\DecValTok{1}\NormalTok{,]}
\NormalTok{findings}\SpecialCharTok{$}\NormalTok{BC9 }\OtherTok{=} \FunctionTok{list}\NormalTok{(}\AttributeTok{all =} \FunctionTok{Anova}\NormalTok{(out),}
                    \AttributeTok{sigint =}\NormalTok{ temp,}
                    \AttributeTok{meta =} \StringTok{"Insufficient variation in JB7 at pH5. Excluding JB7:pH"}\NormalTok{)}
\CommentTok{\#no terms mattered}
\end{Highlighting}
\end{Shaded}

\hypertarget{bc10}{%
\subsection{BC10}\label{bc10}}

\begin{Shaded}
\begin{Highlighting}[]
\NormalTok{dat.cur }\OtherTok{=}\NormalTok{ dat.full[dat.full}\SpecialCharTok{$}\NormalTok{pres.BC10}\SpecialCharTok{==}\DecValTok{1}\NormalTok{,]}
\NormalTok{out }\OtherTok{=} \FunctionTok{lm}\NormalTok{(BC10 }\SpecialCharTok{\textasciitilde{}}\NormalTok{ BC9}\SpecialCharTok{*}\NormalTok{pH }\SpecialCharTok{+}\NormalTok{ JB5}\SpecialCharTok{*}\NormalTok{pH }\SpecialCharTok{+}\NormalTok{ JB7}\SpecialCharTok{*}\NormalTok{pH }\SpecialCharTok{+}\NormalTok{ X135E}\SpecialCharTok{*}\NormalTok{pH }\SpecialCharTok{+}\NormalTok{ JBC}\SpecialCharTok{*}\NormalTok{pH }\SpecialCharTok{+}\NormalTok{ JB370}\SpecialCharTok{*}\NormalTok{pH,}
         \AttributeTok{data =}\NormalTok{ dat.cur)}
\FunctionTok{summary}\NormalTok{(out)}
\end{Highlighting}
\end{Shaded}

\begin{verbatim}
## 
## Call:
## lm(formula = BC10 ~ BC9 * pH + JB5 * pH + JB7 * pH + X135E * 
##     pH + JBC * pH + JB370 * pH, data = dat.cur)
## 
## Residuals:
##        Min         1Q     Median         3Q        Max 
## -132388910  -44471680   -7298838   12969637  196274416 
## 
## Coefficients:
##               Estimate Std. Error t value Pr(>|t|)    
## (Intercept)  8.591e+07  2.865e+07   2.999  0.00424 ** 
## BC9          6.087e-01  1.498e+00   0.407  0.68614    
## pH7          1.691e+08  3.987e+07   4.241 9.84e-05 ***
## JB5         -4.004e-03  1.848e-02  -0.217  0.82941    
## JB7          2.403e-03  3.473e-02   0.069  0.94511    
## X135E        4.502e-01  1.326e+00   0.339  0.73570    
## JBC         -1.120e+00  8.822e-01  -1.269  0.21032    
## JB370        2.098e+01  7.064e+00   2.970  0.00460 ** 
## BC9:pH7     -1.088e+00  1.816e+00  -0.599  0.55207    
## pH7:JB5      6.773e-03  2.215e-02   0.306  0.76109    
## pH7:JB7      1.440e-02  3.882e-02   0.371  0.71218    
## pH7:X135E   -1.999e+00  3.100e+00  -0.645  0.52202    
## pH7:JBC     -1.093e+00  1.173e+00  -0.932  0.35600    
## pH7:JB370   -1.948e+02  7.980e+01  -2.442  0.01828 *  
## ---
## Signif. codes:  0 '***' 0.001 '**' 0.01 '*' 0.05 '.' 0.1 ' ' 1
## 
## Residual standard error: 83150000 on 49 degrees of freedom
## Multiple R-squared:  0.5794, Adjusted R-squared:  0.4678 
## F-statistic: 5.193 on 13 and 49 DF,  p-value: 1.064e-05
\end{verbatim}

\begin{Shaded}
\begin{Highlighting}[]
\NormalTok{temp }\OtherTok{=} \FunctionTok{Anova}\NormalTok{(out)}
\NormalTok{temp }\OtherTok{=}\NormalTok{ temp[}\FunctionTok{grep}\NormalTok{(}\StringTok{\textquotesingle{}:\textquotesingle{}}\NormalTok{,}\FunctionTok{rownames}\NormalTok{(temp)),]}
\NormalTok{temp }\OtherTok{=}\NormalTok{ temp[temp}\SpecialCharTok{$}\StringTok{\textasciigrave{}}\AttributeTok{Pr(\textgreater{}F)}\StringTok{\textasciigrave{}}\SpecialCharTok{\textless{}}\NormalTok{.}\DecValTok{1}\NormalTok{,]}

\NormalTok{findings}\SpecialCharTok{$}\NormalTok{BC10 }\OtherTok{=} \FunctionTok{list}\NormalTok{(}\AttributeTok{all =} \FunctionTok{Anova}\NormalTok{(out),}
                     \AttributeTok{sigint =}\NormalTok{ temp,}
                     \AttributeTok{meta =} \StringTok{"All terms included"}\NormalTok{)}
\CommentTok{\#making piece to add to to overall data frame}
\NormalTok{sigfind.cur }\OtherTok{=} \FunctionTok{as.data.frame}\NormalTok{(temp)}

\DocumentationTok{\#\# calculating the original interaction coefficient sign}
\NormalTok{names.temp }\OtherTok{=} \FunctionTok{gsub}\NormalTok{(}\StringTok{"pH"}\NormalTok{,}\StringTok{""}\NormalTok{,}\FunctionTok{rownames}\NormalTok{(temp))}
\NormalTok{names.temp }\OtherTok{=} \FunctionTok{gsub}\NormalTok{(}\StringTok{":"}\NormalTok{,}\StringTok{""}\NormalTok{, names.temp)}
\NormalTok{coef.orig }\OtherTok{=} \SpecialCharTok{{-}}\FunctionTok{coefficients}\NormalTok{(out)[names.temp]}
\DocumentationTok{\#\# kludgy way of extracting coefficients}
\DocumentationTok{\#\# Note that Anova gives rownames for coefficients, while out gives rownames for}
\DocumentationTok{\#\#   the coefficient LEVELS, so we need to turn pH into pH7}
\NormalTok{coef.temp }\OtherTok{=} \FunctionTok{coefficients}\NormalTok{(out)[}\FunctionTok{gsub}\NormalTok{(}\StringTok{"pH"}\NormalTok{,}\StringTok{"pH7"}\NormalTok{,}\FunctionTok{rownames}\NormalTok{(temp))]}
\NormalTok{coef.new }\OtherTok{=}\NormalTok{coef.orig }\SpecialCharTok{{-}}\NormalTok{ coef.temp}
\DocumentationTok{\#\# calculate change in the sign}
\NormalTok{sigfind.cur}\SpecialCharTok{$}\NormalTok{coef.orig }\OtherTok{=}\NormalTok{ coef.orig}
\NormalTok{sigfind.cur}\SpecialCharTok{$}\NormalTok{coef.new }\OtherTok{=}\NormalTok{ coef.new}
\NormalTok{sigfind.cur}\SpecialCharTok{$}\NormalTok{sign.change }\OtherTok{=} \SpecialCharTok{{-}}\FunctionTok{sign}\NormalTok{(coef.temp)}
\NormalTok{sigfind.cur}\SpecialCharTok{$}\NormalTok{magnitude.change }\OtherTok{=} \FunctionTok{abs}\NormalTok{(coef.new}\SpecialCharTok{/}\NormalTok{coef.orig)}
\NormalTok{sigfind.cur}\SpecialCharTok{$}\NormalTok{source.species }\OtherTok{=} \FunctionTok{gsub}\NormalTok{(}\StringTok{":"}\NormalTok{,}\StringTok{""}\NormalTok{,}
                                  \FunctionTok{gsub}\NormalTok{(}\StringTok{"pH"}\NormalTok{,}\StringTok{""}\NormalTok{,}\FunctionTok{rownames}\NormalTok{(sigfind.cur)))}
\NormalTok{sigfind.cur}\SpecialCharTok{$}\NormalTok{target.species }\OtherTok{=} \StringTok{"BC10"}
\NormalTok{sigfind }\OtherTok{=} \FunctionTok{rbind}\NormalTok{(sigfind, sigfind.cur)}
\end{Highlighting}
\end{Shaded}

\hypertarget{jb5}{%
\subsection{JB5}\label{jb5}}

\begin{Shaded}
\begin{Highlighting}[]
\NormalTok{dat.cur }\OtherTok{=}\NormalTok{ dat.full[dat.full}\SpecialCharTok{$}\NormalTok{pres.JB5}\SpecialCharTok{==}\DecValTok{1}\NormalTok{,]}
\NormalTok{out }\OtherTok{=} \FunctionTok{lm}\NormalTok{(JB5 }\SpecialCharTok{\textasciitilde{}}\NormalTok{ BC9}\SpecialCharTok{*}\NormalTok{pH }\SpecialCharTok{+}\NormalTok{ BC10}\SpecialCharTok{*}\NormalTok{pH }\SpecialCharTok{+}\NormalTok{ JB7}\SpecialCharTok{*}\NormalTok{pH }\SpecialCharTok{+}\NormalTok{ X135E}\SpecialCharTok{*}\NormalTok{pH }\SpecialCharTok{+}\NormalTok{ JBC}\SpecialCharTok{*}\NormalTok{pH }\SpecialCharTok{+}\NormalTok{ JB370}\SpecialCharTok{*}\NormalTok{pH,}
         \AttributeTok{data =}\NormalTok{ dat.cur)}
\FunctionTok{summary}\NormalTok{(out)}
\end{Highlighting}
\end{Shaded}

\begin{verbatim}
## 
## Call:
## lm(formula = JB5 ~ BC9 * pH + BC10 * pH + JB7 * pH + X135E * 
##     pH + JBC * pH + JB370 * pH, data = dat.cur)
## 
## Residuals:
##        Min         1Q     Median         3Q        Max 
## -1.791e+09 -3.570e+08 -3.262e+07  4.426e+08  2.520e+09 
## 
## Coefficients:
##               Estimate Std. Error t value Pr(>|t|)    
## (Intercept)  3.571e+08  3.424e+08   1.043 0.302058    
## BC9         -1.050e+01  1.939e+01  -0.542 0.590509    
## pH7          3.368e+09  5.259e+08   6.404 5.13e-08 ***
## BC10         1.774e+01  6.679e+00   2.656 0.010571 *  
## JB7         -2.892e+05  6.914e+05  -0.418 0.677518    
## X135E        9.282e+01  3.801e+01   2.442 0.018187 *  
## JBC         -5.391e+00  9.475e+00  -0.569 0.571948    
## JB370       -2.905e+01  6.625e+01  -0.438 0.662978    
## BC9:pH7      8.380e+00  1.948e+01   0.430 0.668874    
## pH7:BC10    -1.837e+01  7.073e+00  -2.597 0.012326 *  
## pH7:JB7      2.892e+05  6.914e+05   0.418 0.677518    
## pH7:X135E   -1.156e+02  6.516e+01  -1.774 0.082171 .  
## pH7:JBC     -6.359e+01  1.565e+01  -4.063 0.000171 ***
## pH7:JB370   -2.781e+03  5.883e+02  -4.728 1.89e-05 ***
## ---
## Signif. codes:  0 '***' 0.001 '**' 0.01 '*' 0.05 '.' 0.1 ' ' 1
## 
## Residual standard error: 1.036e+09 on 50 degrees of freedom
## Multiple R-squared:  0.7172, Adjusted R-squared:  0.6436 
## F-statistic: 9.753 on 13 and 50 DF,  p-value: 1.119e-09
\end{verbatim}

\begin{Shaded}
\begin{Highlighting}[]
\NormalTok{temp }\OtherTok{=} \FunctionTok{Anova}\NormalTok{(out)}
\NormalTok{temp }\OtherTok{=}\NormalTok{ temp[}\FunctionTok{grep}\NormalTok{(}\StringTok{\textquotesingle{}:\textquotesingle{}}\NormalTok{,}\FunctionTok{rownames}\NormalTok{(temp)),]}
\NormalTok{temp }\OtherTok{=}\NormalTok{ temp[temp}\SpecialCharTok{$}\StringTok{\textasciigrave{}}\AttributeTok{Pr(\textgreater{}F)}\StringTok{\textasciigrave{}}\SpecialCharTok{\textless{}}\NormalTok{.}\DecValTok{1}\NormalTok{,]}
\CommentTok{\#adding to overall data frame}
\NormalTok{sigfind.cur }\OtherTok{=}\FunctionTok{as.data.frame}\NormalTok{(temp) }
\DocumentationTok{\#\# calculating the original interaction coefficient sign}
\NormalTok{names.temp }\OtherTok{=} \FunctionTok{gsub}\NormalTok{(}\StringTok{"pH"}\NormalTok{,}\StringTok{""}\NormalTok{,}\FunctionTok{rownames}\NormalTok{(temp))}
\NormalTok{names.temp }\OtherTok{=} \FunctionTok{gsub}\NormalTok{(}\StringTok{":"}\NormalTok{,}\StringTok{""}\NormalTok{, names.temp)}
\NormalTok{coef.orig }\OtherTok{=} \SpecialCharTok{{-}}\FunctionTok{coefficients}\NormalTok{(out)[names.temp]}
\DocumentationTok{\#\# kludgy way of extracting coefficients}
\DocumentationTok{\#\# Note that Anova gives rownames for coefficients, while out gives rownames for}
\DocumentationTok{\#\#   the coefficient LEVELS, so we need to turn pH into pH7}
\NormalTok{coef.temp }\OtherTok{=} \FunctionTok{coefficients}\NormalTok{(out)[}\FunctionTok{gsub}\NormalTok{(}\StringTok{"pH"}\NormalTok{,}\StringTok{"pH7"}\NormalTok{,}\FunctionTok{rownames}\NormalTok{(temp))]}
\NormalTok{coef.new }\OtherTok{=}\NormalTok{coef.orig }\SpecialCharTok{{-}}\NormalTok{coef.temp}
\DocumentationTok{\#\# calculate change in the sign}
\NormalTok{sigfind.cur}\SpecialCharTok{$}\NormalTok{coef.orig }\OtherTok{=}\NormalTok{ coef.orig}
\NormalTok{sigfind.cur}\SpecialCharTok{$}\NormalTok{coef.new }\OtherTok{=}\NormalTok{ coef.new}
\NormalTok{sigfind.cur}\SpecialCharTok{$}\NormalTok{sign.change }\OtherTok{=} \SpecialCharTok{{-}}\FunctionTok{sign}\NormalTok{(coef.temp)}
\NormalTok{sigfind.cur}\SpecialCharTok{$}\NormalTok{magnitude.change }\OtherTok{=} \FunctionTok{abs}\NormalTok{(coef.new}\SpecialCharTok{/}\NormalTok{coef.orig)}
\NormalTok{sigfind.cur}\SpecialCharTok{$}\NormalTok{source.species }\OtherTok{=} \FunctionTok{gsub}\NormalTok{(}\StringTok{":"}\NormalTok{,}\StringTok{""}\NormalTok{,}
                                  \FunctionTok{gsub}\NormalTok{(}\StringTok{"pH"}\NormalTok{,}\StringTok{""}\NormalTok{,}\FunctionTok{rownames}\NormalTok{(sigfind.cur)))}
\NormalTok{sigfind.cur}\SpecialCharTok{$}\NormalTok{target.species }\OtherTok{=} \StringTok{"JB5"}
\NormalTok{sigfind }\OtherTok{=} \FunctionTok{rbind}\NormalTok{(sigfind, sigfind.cur)}
\end{Highlighting}
\end{Shaded}

\hypertarget{jb7}{%
\subsection{JB7}\label{jb7}}

\begin{Shaded}
\begin{Highlighting}[]
\NormalTok{dat.cur }\OtherTok{=}\NormalTok{ dat.full[dat.full}\SpecialCharTok{$}\NormalTok{pres.JB7}\SpecialCharTok{==}\DecValTok{1}\NormalTok{,]}
\NormalTok{out }\OtherTok{=} \FunctionTok{lm}\NormalTok{(JB7 }\SpecialCharTok{\textasciitilde{}}\NormalTok{ BC9}\SpecialCharTok{*}\NormalTok{pH }\SpecialCharTok{+}\NormalTok{ BC10}\SpecialCharTok{*}\NormalTok{pH }\SpecialCharTok{+}\NormalTok{ JB5}\SpecialCharTok{*}\NormalTok{pH }\SpecialCharTok{+}\NormalTok{ X135E}\SpecialCharTok{*}\NormalTok{pH }\SpecialCharTok{+}\NormalTok{ JBC}\SpecialCharTok{*}\NormalTok{pH }\SpecialCharTok{+}\NormalTok{ JB370}\SpecialCharTok{*}\NormalTok{pH,}
         \AttributeTok{data =}\NormalTok{ dat.cur)}
\FunctionTok{summary}\NormalTok{(out)}
\end{Highlighting}
\end{Shaded}

\begin{verbatim}
## 
## Call:
## lm(formula = JB7 ~ BC9 * pH + BC10 * pH + JB5 * pH + X135E * 
##     pH + JBC * pH + JB370 * pH, data = dat.cur)
## 
## Residuals:
##        Min         1Q     Median         3Q        Max 
## -1.419e+09 -1.352e+08  1.224e+07  1.857e+08  9.974e+08 
## 
## Coefficients:
##               Estimate Std. Error t value Pr(>|t|)    
## (Intercept) -2.063e+07  1.278e+08  -0.161    0.872    
## BC9          7.267e-01  1.015e+01   0.072    0.943    
## pH7          1.897e+09  2.081e+08   9.118 2.28e-12 ***
## BC10         1.611e+01  2.741e+00   5.877 3.01e-07 ***
## JB5          2.850e+01  4.344e+02   0.066    0.948    
## X135E        1.481e+02  1.271e+01  11.658 4.00e-16 ***
## JBC          1.435e-01  1.364e+00   0.105    0.917    
## JB370        3.332e+01  2.893e+01   1.151    0.255    
## BC9:pH7      5.426e-01  1.017e+01   0.053    0.958    
## pH7:BC10    -1.363e+01  2.856e+00  -4.771 1.52e-05 ***
## pH7:JB5     -2.851e+01  4.344e+02  -0.066    0.948    
## pH7:X135E    4.319e+01  2.309e+01   1.871    0.067 .  
## pH7:JBC     -7.384e+00  5.676e+00  -1.301    0.199    
## pH7:JB370   -2.699e+02  1.959e+02  -1.378    0.174    
## ---
## Signif. codes:  0 '***' 0.001 '**' 0.01 '*' 0.05 '.' 0.1 ' ' 1
## 
## Residual standard error: 4.21e+08 on 52 degrees of freedom
## Multiple R-squared:  0.921,  Adjusted R-squared:  0.9012 
## F-statistic: 46.62 on 13 and 52 DF,  p-value: < 2.2e-16
\end{verbatim}

\begin{Shaded}
\begin{Highlighting}[]
\NormalTok{temp }\OtherTok{=} \FunctionTok{Anova}\NormalTok{(out)}
\NormalTok{temp }\OtherTok{=}\NormalTok{ temp[}\FunctionTok{grep}\NormalTok{(}\StringTok{\textquotesingle{}:\textquotesingle{}}\NormalTok{,}\FunctionTok{rownames}\NormalTok{(temp)),]}
\NormalTok{temp }\OtherTok{=}\NormalTok{ temp[temp}\SpecialCharTok{$}\StringTok{\textasciigrave{}}\AttributeTok{Pr(\textgreater{}F)}\StringTok{\textasciigrave{}}\SpecialCharTok{\textless{}}\NormalTok{.}\DecValTok{1}\NormalTok{,]}

\NormalTok{findings}\SpecialCharTok{$}\NormalTok{JB7 }\OtherTok{=} \FunctionTok{list}\NormalTok{(}\AttributeTok{all =} \FunctionTok{Anova}\NormalTok{(out),}
                    \AttributeTok{sigint =}\NormalTok{ temp,}
                    \AttributeTok{meta =} \StringTok{"All terms included"}\NormalTok{)}
\CommentTok{\#adding to overall data frame}
\NormalTok{sigfind.cur }\OtherTok{=}\FunctionTok{as.data.frame}\NormalTok{(temp) }
\DocumentationTok{\#\# calculating the original interaction coefficient sign}
\NormalTok{names.temp }\OtherTok{=} \FunctionTok{gsub}\NormalTok{(}\StringTok{"pH"}\NormalTok{,}\StringTok{""}\NormalTok{,}\FunctionTok{rownames}\NormalTok{(temp))}
\NormalTok{names.temp }\OtherTok{=} \FunctionTok{gsub}\NormalTok{(}\StringTok{":"}\NormalTok{,}\StringTok{""}\NormalTok{, names.temp)}
\NormalTok{coef.orig }\OtherTok{=} \SpecialCharTok{{-}}\FunctionTok{coefficients}\NormalTok{(out)[names.temp]}
\DocumentationTok{\#\# kludgy way of extracting coefficients}
\DocumentationTok{\#\# Note that Anova gives rownames for coefficients, while out gives rownames for}
\DocumentationTok{\#\#   the coefficient LEVELS, so we need to turn pH into pH7}
\NormalTok{coef.temp }\OtherTok{=} \FunctionTok{coefficients}\NormalTok{(out)[}\FunctionTok{gsub}\NormalTok{(}\StringTok{"pH"}\NormalTok{,}\StringTok{"pH7"}\NormalTok{,}\FunctionTok{rownames}\NormalTok{(temp))]}
\NormalTok{coef.new }\OtherTok{=}\NormalTok{coef.orig }\SpecialCharTok{{-}}\NormalTok{coef.temp}
\DocumentationTok{\#\# calculate change in the sign}
\NormalTok{sigfind.cur}\SpecialCharTok{$}\NormalTok{coef.orig }\OtherTok{=}\NormalTok{ coef.orig}
\NormalTok{sigfind.cur}\SpecialCharTok{$}\NormalTok{coef.new }\OtherTok{=}\NormalTok{ coef.new}
\NormalTok{sigfind.cur}\SpecialCharTok{$}\NormalTok{sign.change }\OtherTok{=} \SpecialCharTok{{-}}\FunctionTok{sign}\NormalTok{(coef.temp)}
\NormalTok{sigfind.cur}\SpecialCharTok{$}\NormalTok{magnitude.change }\OtherTok{=} \FunctionTok{abs}\NormalTok{(coef.new}\SpecialCharTok{/}\NormalTok{coef.orig)}
\NormalTok{sigfind.cur}\SpecialCharTok{$}\NormalTok{source.species }\OtherTok{=} \FunctionTok{gsub}\NormalTok{(}\StringTok{":"}\NormalTok{,}\StringTok{""}\NormalTok{,}
                                  \FunctionTok{gsub}\NormalTok{(}\StringTok{"pH"}\NormalTok{,}\StringTok{""}\NormalTok{,}\FunctionTok{rownames}\NormalTok{(sigfind.cur)))}
\NormalTok{sigfind.cur}\SpecialCharTok{$}\NormalTok{target.species }\OtherTok{=} \StringTok{"JB7"}
\NormalTok{sigfind }\OtherTok{=} \FunctionTok{rbind}\NormalTok{(sigfind, sigfind.cur)}
\end{Highlighting}
\end{Shaded}

\hypertarget{x135e}{%
\subsection{X135E}\label{x135e}}

\begin{Shaded}
\begin{Highlighting}[]
\NormalTok{dat.cur }\OtherTok{=}\NormalTok{ dat.full[dat.full}\SpecialCharTok{$}\NormalTok{pres.X135E}\SpecialCharTok{==}\DecValTok{1}\NormalTok{,]}
\NormalTok{out }\OtherTok{=} \FunctionTok{lm}\NormalTok{(X135E }\SpecialCharTok{\textasciitilde{}}\NormalTok{ BC9}\SpecialCharTok{*}\NormalTok{pH }\SpecialCharTok{+}\NormalTok{ BC10}\SpecialCharTok{*}\NormalTok{pH }\SpecialCharTok{+}\NormalTok{ JB5}\SpecialCharTok{*}\NormalTok{pH }\SpecialCharTok{+}\NormalTok{ JB7}\SpecialCharTok{*}\NormalTok{pH }\SpecialCharTok{+}\NormalTok{ JBC}\SpecialCharTok{*}\NormalTok{pH }\SpecialCharTok{+}\NormalTok{ JB370}\SpecialCharTok{*}\NormalTok{pH,}
         \AttributeTok{data =}\NormalTok{ dat.cur)}
\FunctionTok{summary}\NormalTok{(out)}
\end{Highlighting}
\end{Shaded}

\begin{verbatim}
## 
## Call:
## lm(formula = X135E ~ BC9 * pH + BC10 * pH + JB5 * pH + JB7 * 
##     pH + JBC * pH + JB370 * pH, data = dat.cur)
## 
## Residuals:
##       Min        1Q    Median        3Q       Max 
## -16323810  -4004118   -486141   4122702  26253884 
## 
## Coefficients:
##               Estimate Std. Error t value Pr(>|t|)    
## (Intercept)  1.798e+07  2.490e+06   7.221  2.2e-09 ***
## BC9          8.187e-02  1.956e-02   4.186 0.000110 ***
## pH7         -9.148e+06  3.978e+06  -2.299 0.025524 *  
## BC10         1.455e-01  3.900e-02   3.731 0.000474 ***
## JB5         -1.207e-03  2.155e-03  -0.560 0.577895    
## JB7          8.471e-05  1.691e-03   0.050 0.960239    
## JBC         -1.037e-01  3.813e-02  -2.719 0.008868 ** 
## JB370       -1.186e+00  1.176e+00  -1.009 0.317683    
## BC9:pH7     -6.147e-02  2.558e-02  -2.403 0.019851 *  
## pH7:BC10    -1.424e-01  4.212e-02  -3.382 0.001374 ** 
## pH7:JB5      1.745e-03  2.529e-03   0.690 0.493304    
## pH7:JB7      8.276e-04  2.032e-03   0.407 0.685467    
## pH7:JBC     -5.908e-02  1.008e-01  -0.586 0.560282    
## pH7:JB370    2.257e+00  3.102e+00   0.728 0.470127    
## ---
## Signif. codes:  0 '***' 0.001 '**' 0.01 '*' 0.05 '.' 0.1 ' ' 1
## 
## Residual standard error: 8739000 on 52 degrees of freedom
## Multiple R-squared:  0.6103, Adjusted R-squared:  0.5129 
## F-statistic: 6.265 on 13 and 52 DF,  p-value: 6.891e-07
\end{verbatim}

\begin{Shaded}
\begin{Highlighting}[]
\NormalTok{temp }\OtherTok{=} \FunctionTok{Anova}\NormalTok{(out)}
\NormalTok{temp }\OtherTok{=}\NormalTok{ temp[}\FunctionTok{grep}\NormalTok{(}\StringTok{\textquotesingle{}:\textquotesingle{}}\NormalTok{,}\FunctionTok{rownames}\NormalTok{(temp)),]}
\NormalTok{temp }\OtherTok{=}\NormalTok{ temp[temp}\SpecialCharTok{$}\StringTok{\textasciigrave{}}\AttributeTok{Pr(\textgreater{}F)}\StringTok{\textasciigrave{}}\SpecialCharTok{\textless{}}\NormalTok{.}\DecValTok{1}\NormalTok{,]}
\NormalTok{findings}\SpecialCharTok{$}\NormalTok{X135E }\OtherTok{=} \FunctionTok{list}\NormalTok{(}\AttributeTok{all =} \FunctionTok{Anova}\NormalTok{(out),}
                      \AttributeTok{sigint =}\NormalTok{ temp,}
                      \AttributeTok{meta =} \StringTok{"All terms included"}\NormalTok{)}
\CommentTok{\#adding to overall data frame}
\NormalTok{sigfind.cur }\OtherTok{=}\FunctionTok{as.data.frame}\NormalTok{(temp) }
\DocumentationTok{\#\# calculating the original interaction coefficient sign}
\NormalTok{names.temp }\OtherTok{=} \FunctionTok{gsub}\NormalTok{(}\StringTok{"pH"}\NormalTok{,}\StringTok{""}\NormalTok{,}\FunctionTok{rownames}\NormalTok{(temp))}
\NormalTok{names.temp }\OtherTok{=} \FunctionTok{gsub}\NormalTok{(}\StringTok{":"}\NormalTok{,}\StringTok{""}\NormalTok{, names.temp)}
\NormalTok{coef.orig }\OtherTok{=} \SpecialCharTok{{-}}\FunctionTok{coefficients}\NormalTok{(out)[names.temp]}
\DocumentationTok{\#\# kludgy way of extracting coefficients}
\DocumentationTok{\#\# Note that Anova gives rownames for coefficients, while out gives rownames for}
\DocumentationTok{\#\#   the coefficient LEVELS, so we need to turn pH into pH7}
\NormalTok{coef.temp }\OtherTok{=} \FunctionTok{coefficients}\NormalTok{(out)[}\FunctionTok{gsub}\NormalTok{(}\StringTok{"pH"}\NormalTok{,}\StringTok{"pH7"}\NormalTok{,}\FunctionTok{rownames}\NormalTok{(temp))]}
\NormalTok{coef.new }\OtherTok{=}\NormalTok{coef.orig }\SpecialCharTok{{-}}\NormalTok{coef.temp}
\DocumentationTok{\#\# calculate change in the sign}
\NormalTok{sigfind.cur}\SpecialCharTok{$}\NormalTok{coef.orig }\OtherTok{=}\NormalTok{ coef.orig}
\NormalTok{sigfind.cur}\SpecialCharTok{$}\NormalTok{coef.new }\OtherTok{=}\NormalTok{ coef.new}
\NormalTok{sigfind.cur}\SpecialCharTok{$}\NormalTok{sign.change }\OtherTok{=} \SpecialCharTok{{-}}\FunctionTok{sign}\NormalTok{(coef.temp)}
\NormalTok{sigfind.cur}\SpecialCharTok{$}\NormalTok{magnitude.change }\OtherTok{=} \FunctionTok{abs}\NormalTok{(coef.new}\SpecialCharTok{/}\NormalTok{coef.orig)}
\NormalTok{sigfind.cur}\SpecialCharTok{$}\NormalTok{source.species }\OtherTok{=} \FunctionTok{gsub}\NormalTok{(}\StringTok{":"}\NormalTok{,}\StringTok{""}\NormalTok{,}
                                  \FunctionTok{gsub}\NormalTok{(}\StringTok{"pH"}\NormalTok{,}\StringTok{""}\NormalTok{,}\FunctionTok{rownames}\NormalTok{(sigfind.cur)))}
\NormalTok{sigfind.cur}\SpecialCharTok{$}\NormalTok{target.species }\OtherTok{=} \StringTok{"X135E"}

\NormalTok{sigfind }\OtherTok{=} \FunctionTok{rbind}\NormalTok{(sigfind, sigfind.cur)}
\end{Highlighting}
\end{Shaded}

\hypertarget{jbc}{%
\subsection{JBC}\label{jbc}}

\begin{Shaded}
\begin{Highlighting}[]
\NormalTok{dat.cur }\OtherTok{=}\NormalTok{ dat.full[dat.full}\SpecialCharTok{$}\NormalTok{pres.JBC}\SpecialCharTok{==}\DecValTok{1}\NormalTok{,]}
\CommentTok{\# View(dat.cur)}
\NormalTok{out }\OtherTok{=} \FunctionTok{lm}\NormalTok{(JBC }\SpecialCharTok{\textasciitilde{}}\NormalTok{ BC9}\SpecialCharTok{*}\NormalTok{pH }\SpecialCharTok{+}\NormalTok{ BC10}\SpecialCharTok{*}\NormalTok{pH }\SpecialCharTok{+}\NormalTok{ JB5 }\SpecialCharTok{+}\NormalTok{ JB7}\SpecialCharTok{*}\NormalTok{pH }\SpecialCharTok{+}\NormalTok{ X135E }\SpecialCharTok{+}\NormalTok{ JB370}\SpecialCharTok{*}\NormalTok{pH,}
         \AttributeTok{data =}\NormalTok{ dat.cur)}
\FunctionTok{summary}\NormalTok{(out)}
\end{Highlighting}
\end{Shaded}

\begin{verbatim}
## 
## Call:
## lm(formula = JBC ~ BC9 * pH + BC10 * pH + JB5 + JB7 * pH + X135E + 
##     JB370 * pH, data = dat.cur)
## 
## Residuals:
##       Min        1Q    Median        3Q       Max 
## -33439899 -10404664  -1617057   3930022 166560101 
## 
## Coefficients:
##               Estimate Std. Error t value Pr(>|t|)    
## (Intercept)  6.944e+07  6.214e+06  11.174 2.56e-15 ***
## BC9          8.469e-03  6.166e-01   0.014   0.9891    
## pH7         -1.309e+07  9.863e+06  -1.327   0.1905    
## BC10        -5.559e-01  6.349e-01  -0.876   0.3854    
## JB5         -1.739e-02  1.972e-01  -0.088   0.9301    
## JB7          1.558e+01  1.814e+00   8.590 1.76e-11 ***
## X135E       -5.312e+04  2.734e+04  -1.943   0.0576 .  
## JB370       -3.047e+01  2.899e+01  -1.051   0.2982    
## BC9:pH7     -5.867e-02  6.212e-01  -0.094   0.9251    
## pH7:BC10     5.839e-01  6.461e-01   0.904   0.3704    
## pH7:JB7     -1.559e+01  1.814e+00  -8.594 1.74e-11 ***
## pH7:JB370    1.322e+03  9.200e+03   0.144   0.8863    
## ---
## Signif. codes:  0 '***' 0.001 '**' 0.01 '*' 0.05 '.' 0.1 ' ' 1
## 
## Residual standard error: 28320000 on 51 degrees of freedom
## Multiple R-squared:  0.6427, Adjusted R-squared:  0.5657 
## F-statistic: 8.341 on 11 and 51 DF,  p-value: 3.871e-08
\end{verbatim}

\begin{Shaded}
\begin{Highlighting}[]
\NormalTok{temp }\OtherTok{=} \FunctionTok{Anova}\NormalTok{(out)}
\NormalTok{temp }\OtherTok{=}\NormalTok{ temp[}\FunctionTok{grep}\NormalTok{(}\StringTok{\textquotesingle{}:\textquotesingle{}}\NormalTok{,}\FunctionTok{rownames}\NormalTok{(temp)),]}
\NormalTok{temp }\OtherTok{=}\NormalTok{ temp[temp}\SpecialCharTok{$}\StringTok{\textasciigrave{}}\AttributeTok{Pr(\textgreater{}F)}\StringTok{\textasciigrave{}}\SpecialCharTok{\textless{}}\NormalTok{.}\DecValTok{1}\NormalTok{,]}
\NormalTok{findings}\SpecialCharTok{$}\NormalTok{JBC }\OtherTok{=} \FunctionTok{list}\NormalTok{(}\AttributeTok{all =} \FunctionTok{Anova}\NormalTok{(out),}
                    \AttributeTok{sigint =}\NormalTok{ temp,}
                    \AttributeTok{meta =} \StringTok{"Insufficient JB5 and X135E at pH 5 {-} both JB5:pH and X135E:pH were removed."}\NormalTok{)}
\CommentTok{\#adding to overall data frame}
\NormalTok{sigfind.cur }\OtherTok{=} \FunctionTok{as.data.frame}\NormalTok{(temp) }
\DocumentationTok{\#\# calculating the original interaction coefficient sign}
\NormalTok{names.temp }\OtherTok{=} \FunctionTok{gsub}\NormalTok{(}\StringTok{"pH"}\NormalTok{,}\StringTok{""}\NormalTok{,}\FunctionTok{rownames}\NormalTok{(temp))}
\NormalTok{names.temp }\OtherTok{=} \FunctionTok{gsub}\NormalTok{(}\StringTok{":"}\NormalTok{,}\StringTok{""}\NormalTok{, names.temp)}
\NormalTok{coef.orig }\OtherTok{=} \SpecialCharTok{{-}}\FunctionTok{coefficients}\NormalTok{(out)[names.temp]}
\DocumentationTok{\#\# kludgy way of extracting coefficients}
\DocumentationTok{\#\# Note that Anova gives rownames for coefficients, while out gives rownames for}
\DocumentationTok{\#\#   the coefficient LEVELS, so we need to turn pH into pH7}
\NormalTok{coef.temp }\OtherTok{=} \FunctionTok{coefficients}\NormalTok{(out)[}\FunctionTok{gsub}\NormalTok{(}\StringTok{"pH"}\NormalTok{,}\StringTok{"pH7"}\NormalTok{,}\FunctionTok{rownames}\NormalTok{(temp))]}
\NormalTok{coef.new }\OtherTok{=}\NormalTok{coef.orig }\SpecialCharTok{{-}}\NormalTok{coef.temp}
\DocumentationTok{\#\# calculate change in the sign}
\NormalTok{sigfind.cur}\SpecialCharTok{$}\NormalTok{coef.orig }\OtherTok{=}\NormalTok{ coef.orig}
\NormalTok{sigfind.cur}\SpecialCharTok{$}\NormalTok{coef.new }\OtherTok{=}\NormalTok{ coef.new}
\NormalTok{sigfind.cur}\SpecialCharTok{$}\NormalTok{sign.change }\OtherTok{=} \SpecialCharTok{{-}}\FunctionTok{sign}\NormalTok{(coef.temp)}
\NormalTok{sigfind.cur}\SpecialCharTok{$}\NormalTok{magnitude.change }\OtherTok{=} \FunctionTok{abs}\NormalTok{(coef.new}\SpecialCharTok{/}\NormalTok{coef.orig)}
\NormalTok{sigfind.cur}\SpecialCharTok{$}\NormalTok{source.species }\OtherTok{=} \FunctionTok{gsub}\NormalTok{(}\StringTok{":"}\NormalTok{,}\StringTok{""}\NormalTok{,}
                                  \FunctionTok{gsub}\NormalTok{(}\StringTok{"pH"}\NormalTok{,}\StringTok{""}\NormalTok{,}\FunctionTok{rownames}\NormalTok{(sigfind.cur)))}
\NormalTok{sigfind.cur}\SpecialCharTok{$}\NormalTok{target.species }\OtherTok{=} \StringTok{"JBC"}
\NormalTok{sigfind }\OtherTok{=} \FunctionTok{rbind}\NormalTok{(sigfind, sigfind.cur)}
\end{Highlighting}
\end{Shaded}

\hypertarget{jb370}{%
\subsection{JB370}\label{jb370}}

\begin{Shaded}
\begin{Highlighting}[]
\NormalTok{dat.cur }\OtherTok{=}\NormalTok{ dat.full[dat.full}\SpecialCharTok{$}\NormalTok{pres.JB370}\SpecialCharTok{==}\DecValTok{1}\NormalTok{,]}
\NormalTok{out }\OtherTok{=} \FunctionTok{lm}\NormalTok{(JB370 }\SpecialCharTok{\textasciitilde{}}\NormalTok{ BC9}\SpecialCharTok{*}\NormalTok{pH }\SpecialCharTok{+}\NormalTok{ BC10}\SpecialCharTok{*}\NormalTok{pH }\SpecialCharTok{+}\NormalTok{ JB5}\SpecialCharTok{*}\NormalTok{pH }\SpecialCharTok{+}\NormalTok{ JB7}\SpecialCharTok{*}\NormalTok{pH }\SpecialCharTok{+}\NormalTok{ X135E}\SpecialCharTok{*}\NormalTok{pH }\SpecialCharTok{+}\NormalTok{ JBC}\SpecialCharTok{*}\NormalTok{pH,}
         \AttributeTok{data =}\NormalTok{ dat.cur)}
\FunctionTok{summary}\NormalTok{(out)}
\end{Highlighting}
\end{Shaded}

\begin{verbatim}
## 
## Call:
## lm(formula = JB370 ~ BC9 * pH + BC10 * pH + JB5 * pH + JB7 * 
##     pH + X135E * pH + JBC * pH, data = dat.cur)
## 
## Residuals:
##      Min       1Q   Median       3Q      Max 
## -7353045  -458549   -53831   350771  8759462 
## 
## Coefficients:
##               Estimate Std. Error t value Pr(>|t|)    
## (Intercept)  6.451e+06  7.697e+05   8.381 5.03e-11 ***
## BC9          1.445e-02  3.350e-03   4.314 7.75e-05 ***
## pH7         -5.540e+06  1.092e+06  -5.073 6.02e-06 ***
## BC10         1.833e-03  6.383e-03   0.287 0.775224    
## JB5          9.279e-03  1.080e-02   0.859 0.394605    
## JB7          6.086e-03  4.419e-03   1.377 0.174671    
## X135E       -1.973e-01  1.127e-01  -1.751 0.086156 .  
## JBC         -1.277e-01  3.524e-02  -3.624 0.000688 ***
## BC9:pH7     -1.587e-02  6.234e-03  -2.546 0.014081 *  
## pH7:BC10    -3.557e-03  1.000e-02  -0.356 0.723725    
## pH7:JB5     -9.108e-03  1.094e-02  -0.833 0.409072    
## pH7:JB7     -5.782e-03  4.527e-03  -1.277 0.207534    
## pH7:X135E    3.098e-01  1.837e-01   1.686 0.098152 .  
## pH7:JBC      1.144e-01  4.240e-02   2.698 0.009538 ** 
## ---
## Signif. codes:  0 '***' 0.001 '**' 0.01 '*' 0.05 '.' 0.1 ' ' 1
## 
## Residual standard error: 2605000 on 49 degrees of freedom
## Multiple R-squared:  0.7168, Adjusted R-squared:  0.6417 
## F-statistic: 9.541 on 13 and 49 DF,  p-value: 1.963e-09
\end{verbatim}

\begin{Shaded}
\begin{Highlighting}[]
\NormalTok{temp }\OtherTok{=} \FunctionTok{Anova}\NormalTok{(out)}
\NormalTok{temp }\OtherTok{=}\NormalTok{ temp[}\FunctionTok{grep}\NormalTok{(}\StringTok{\textquotesingle{}:\textquotesingle{}}\NormalTok{,}\FunctionTok{rownames}\NormalTok{(temp)),]}
\NormalTok{temp }\OtherTok{=}\NormalTok{ temp[temp}\SpecialCharTok{$}\StringTok{\textasciigrave{}}\AttributeTok{Pr(\textgreater{}F)}\StringTok{\textasciigrave{}}\SpecialCharTok{\textless{}}\NormalTok{.}\DecValTok{1}\NormalTok{,]}
\NormalTok{findings}\SpecialCharTok{$}\NormalTok{JB370 }\OtherTok{=} \FunctionTok{list}\NormalTok{(}\AttributeTok{all =} \FunctionTok{Anova}\NormalTok{(out),}
                      \AttributeTok{sigint =}\NormalTok{ temp,}
                      \AttributeTok{meta =} \StringTok{"All terms included"}\NormalTok{)}
\CommentTok{\#adding to overall data frame}
\NormalTok{sigfind.cur }\OtherTok{=}\FunctionTok{as.data.frame}\NormalTok{(temp) }
\DocumentationTok{\#\# calculating the original interaction coefficient sign}
\NormalTok{names.temp }\OtherTok{=} \FunctionTok{gsub}\NormalTok{(}\StringTok{"pH"}\NormalTok{,}\StringTok{""}\NormalTok{,}\FunctionTok{rownames}\NormalTok{(temp))}
\NormalTok{names.temp }\OtherTok{=} \FunctionTok{gsub}\NormalTok{(}\StringTok{":"}\NormalTok{,}\StringTok{""}\NormalTok{, names.temp)}
\NormalTok{coef.orig }\OtherTok{=} \SpecialCharTok{{-}}\FunctionTok{coefficients}\NormalTok{(out)[names.temp]}
\DocumentationTok{\#\# kludgy way of extracting coefficients}
\DocumentationTok{\#\# Note that Anova gives rownames for coefficients, while out gives rownames for}
\DocumentationTok{\#\#   the coefficient LEVELS, so we need to turn pH into pH7}
\NormalTok{coef.temp }\OtherTok{=} \FunctionTok{coefficients}\NormalTok{(out)[}\FunctionTok{gsub}\NormalTok{(}\StringTok{"pH"}\NormalTok{,}\StringTok{"pH7"}\NormalTok{,}\FunctionTok{rownames}\NormalTok{(temp))]}
\NormalTok{coef.new }\OtherTok{=}\NormalTok{coef.orig }\SpecialCharTok{{-}}\NormalTok{coef.temp}
\DocumentationTok{\#\# calculate change in the sign}
\NormalTok{sigfind.cur}\SpecialCharTok{$}\NormalTok{coef.orig }\OtherTok{=}\NormalTok{ coef.orig}
\NormalTok{sigfind.cur}\SpecialCharTok{$}\NormalTok{coef.new }\OtherTok{=}\NormalTok{ coef.new}
\NormalTok{sigfind.cur}\SpecialCharTok{$}\NormalTok{sign.change }\OtherTok{=} \SpecialCharTok{{-}}\FunctionTok{sign}\NormalTok{(coef.temp)}
\NormalTok{sigfind.cur}\SpecialCharTok{$}\NormalTok{magnitude.change }\OtherTok{=} \FunctionTok{abs}\NormalTok{(coef.new}\SpecialCharTok{/}\NormalTok{coef.orig)}
\NormalTok{sigfind.cur}\SpecialCharTok{$}\NormalTok{source.species }\OtherTok{=} \FunctionTok{gsub}\NormalTok{(}\StringTok{":"}\NormalTok{,}\StringTok{""}\NormalTok{,}
                                  \FunctionTok{gsub}\NormalTok{(}\StringTok{"pH"}\NormalTok{,}\StringTok{""}\NormalTok{,}\FunctionTok{rownames}\NormalTok{(sigfind.cur)))}
\NormalTok{sigfind.cur}\SpecialCharTok{$}\NormalTok{target.species }\OtherTok{=} \StringTok{"JB370"}
\NormalTok{sigfind }\OtherTok{=} \FunctionTok{rbind}\NormalTok{(sigfind, sigfind.cur)}
\end{Highlighting}
\end{Shaded}

\hypertarget{saving}{%
\subsection{Saving}\label{saving}}

\begin{Shaded}
\begin{Highlighting}[]
\FunctionTok{write.csv}\NormalTok{(sigfind,}\FunctionTok{here}\NormalTok{(}\StringTok{"4\_res"}\NormalTok{,}\StringTok{"pH{-}interaction{-}estimates.csv"}\NormalTok{), }\AttributeTok{row.names =} \ConstantTok{FALSE}\NormalTok{)}

\FunctionTok{saveRDS}\NormalTok{(findings, }\AttributeTok{file =} \FunctionTok{here}\NormalTok{(}\StringTok{"4\_res"}\NormalTok{,}\StringTok{"ph{-}interactions{-}list{-}obj.RDS"}\NormalTok{))}
\end{Highlighting}
\end{Shaded}


\end{document}
